\documentclass[10pt]{trumathoutline}

%implicitly uses mathead.png for the header. You can get it from github from https://github.com/mateenshaikh/trucourseoutline/blob/main/masthead.png

\newcommand{\smsp}{\vspace{0.25cm}}

%%% Below are the macros that you are welcome to change/fill in. 
%%% I prefer to only change the macros and leave the body of the document unchanged
%%% But if you want to manually fiddle with things, it's easy to crosslist

\newcommand{\instructorname}     {Kyle Schlitt}
\newcommand{\instructorphone}    {250-828-5253}
\newcommand{\instructoremail}    {kschlitt@tru.ca}
\newcommand{\instructorofficehrs}{T: 1:30-3:20, F: 2:30-4:20}
\newcommand{\instructorofficerm} {S140}

\newcommand{\coursecode}     	 {Math 1070- 03, 06}
\newcommand{\coursetitle}    	 {Mathematics for Business and Economics}
\newcommand{\coursevectoring}	 {(3,1.5,0)}
\newcommand{\term}           	 {Winter 2024}
\newcommand{\courseprerequisites}{Foundations of Math 12 or Pre-calculus 12 with a minimum grade of 67\% (C+), or MATH 0600 with a minimum grade of B-, or at least one of MATH 0610, MATH 0630, MATH 0633, MATH 0650, MATH 1000, or MATH 1001 with a minimum grade of C-.}
\newcommand{\coursecorequisites}{None}
\newcommand{\courseexclusions}{Students can get credit for only one of MATH 1070, MATH 1071, MATH 1090, MATH 1091, MATH 1100, or MATH 1101.}


\newcommand{\calendardescription}{
	This course is designed for Business and Economics students. Topics include linear and non-linear functions and models applied to cost, revenue, profit, demand and supply, systems of equations (linear and nonlinear), matrices, linear programming, difference equations, and mathematics of finance (including simple and compound interest, annuities, mortgages, and loans).}

\newcommand{\coursedescription}{
}

\newcommand{\courseobjectives}{
\begin{enumerate}
\item Interpret graphs and equations of linear, quadratic, exponential, and logarithmic functions to identify their main features (domain, range, intercepts, and other points).
\item Perform basic matrix operations including row reduction.
\item Set up necessary equations and solve applications of linear and non-linear systems.
\item Solve systems of equations by graphing, substitution, elimination, row reduction and matrix inverses.
\item Solve linear programming applications by setting up and graphing inequalities.
\item Use formulas to solve financial mathematics problems (compound interest, present and future value, loans, and annuities).
\item Solve applications of difference equations, using recursion and/or shortcuts.	\end{enumerate}
}

\newcommand{\textsmaterials}{
\begin{itemize}
\item \emph{Introductory Mathematical Analysis for Business, Economics, and the Life and Social Sciences}, Fourteenth Edition, by E.~F.~Haeussler, R.~S.~Paul, and R.~J.~Wood.
\item Incomplete notes, designed to be filled in during lectures, will be posted on the Course Itinerary, which is available on Moodle.
\end{itemize}
}

\newcommand{\evaluation}{
	WeBWorK		\dotfill	7\%\\
	Quizzes		\dotfill 	18\%\\
	Midterms (Jan 31, Feb 28, Mar 27, best two out of three)		\dotfill	35\%\\
	Final Exam 	\dotfill	40\%\\

	\hrule~\\
	Total 		\dotfill 	100\%\\
	
	If there is an academic integrity violation, then all “best” criteria turns into “worst” criteria throughout the entire course’s evaluation.
}


\newcommand{\educationalaids}{
	
	\subsection*{Formula Sheets}
	Students will have a formula sheet for tests and the exam. Further information will be provided closer to the first test.
	
	\subsection*{Calculator}
	A scientific calculator is allowed. Graphing calculators are not permitted on tests or quizzes. Use of mobile/communication devices as calculator is not allowed during evaluations.
}


\newcommand{\coursetopics}{
\begin{enumerate}
\item[] \textbf{Chapter 1: Applications and More Algebra}
\begin{itemize}
\item[] Section 1.1: Applications of Equations
\item[] Section 1.2: Linear Inequalities
\end{itemize}
\smsp

\item[] \textbf{Chapter 2: Functions and Graphs}
\begin{itemize}
\item[] Section 2.1: Functions
\item[] Section 2.2: Special Functions
\item[] Section 2.5: Graphs in Rectangular Coordinates
\end{itemize}
\smsp

\item[] \textbf{Chapter 3: Lines, Parabolas, and Systems}
\begin{itemize}
\item[] Section 3.1: Lines
\item[] Section 3.2: Applications and Linear Functions
\item[] Section 3.3: Quadratic Functions
\item[] Section 3.4: Systems of Linear Equations
\item[] Section 3.5: Nonlinear Systems
\item[] Section 3.6: Applications of Systems of Equations
\end{itemize}
\smsp

\item[] \textbf{Chapter 6: Matrix Algebra}
\begin{itemize}
\item[] Section 6.1: Matrices
\item[] Section 6.2: Matrix Addition and Scalar Multiplication
\item[] Section 6.3: Matrix Multiplication 
\item[] Section 6.4: Solving Systems by Reducing Matrices
\item[] Section 6.5: Solving Systems by Reducing Matrices (Continued)
\item[] Section 6.6: Inverses
\end{itemize}
\smsp

\item[] \textbf{Chapter 7: Linear Programming}
\begin{itemize}
\item []Section 7.1: Linear Inequalities in Two Variables
\item []Section 7.2: Linear Programming
\end{itemize}
\smsp

\item[] \textbf{Chapter 4: Exponential and Logarithmic Functions}
\begin{itemize}
\item []Section 4.1: Exponential Functions
\item []Section 4.2: Logarithmic Functions
\item []Section 4.3: Properties of Logarithms
\item []Section 4.4: Logarithmic and Exponential Functions
\end{itemize}
\smsp

\item[] \textbf{Chapter 5: Mathematics of Finance}
\begin{itemize}
\item []Section 5.1: Compound Interest
\item []Section 5.2: Present Value
\item []Section 5.3: Interest Compounded Continuously
\item []Section 5.4: Annuities
\item []Section 5.5: Amortization of Loans
\item []Section 5.6: Perpetuities
\end{itemize}
\smsp

\item[] \textbf{Chapter 11: Difference Equations and Mathematical Models (Goldstein, Schneider, and \nolinebreak Siegel)}
\begin{itemize}
\item []Section 11.1: Introduction to Difference Equations
\item []Section 11.2: Difference Equations and Interest
\item []Section 11.3: Graphing Difference Equations
\item []Section 11.5: Modelling with Difference Equations
\end{itemize}
\end{enumerate}
}


\begin{document}
	
	~
	
	\vspace{-1em}
	
	\begin{center}
		\bfseries
		{\Large \underline{Course Outline}}\\[0.5em]
		
		\coursecode
		
		\coursetitle\ \coursevectoring
		
		\term
		
	\end{center}
\begin{tabular}{ r l}
	\textbf{Instructor:}	& \instructorname 	\\
	\textbf{Office:} 		& \instructorofficerm \\
	\textbf{Office Hours:} 	& \instructorofficehrs
\end{tabular}\hfill
\begin{tabular}{ r l}
	 \textbf{Phone:} & \instructorphone\\
	 \textbf{email:} & \href{mailto:\instructoremail}{\instructoremail}\\
	~
\end{tabular}

\vspace{1em}


Thompson Rivers University is located on the Tk’emlups te Secwepemc territory that is situated in the Southern interior of British Columbia within the unceded traditional lands of the Secwepemc Nation.

\section*{Calendar Description}
\calendardescription
	
%\section*{course Description}
%\begin{quote}
%	\coursedescription
%\end{quote}


\section*{Educational Objectives}
Upon completion of the course the students will be expected to:
\courseobjectives

\section*{Prequisites}
\courseprerequisites

\section*{Corequisites}
\coursecorequisites

\section*{Exclusions}
\courseexclusions

\section*{Texts/Materials}
\textsmaterials

\section*{Student Evaluation}
\begin{minipage}{\textwidth}
\evaluation
\end{minipage}\\


In the event a student misses an evaluation, a mark of zero will be given unless the student contacts the instructor prior to the evaluation/deadline, informing the instructor of the particular situation. Students are responsible for checking the final examination schedule which shall be posted each semester by the Registrar, and for advising the Registrar of any conflicts within the schedule. Attendance at a scheduled final examination is mandatory, and the responsibility is on the student to seek remedy for a missed final exam. Students can refer to TRU Examination Policy (ED 03-9) for more information.


\section*{Accessibility Services}
All TRU students who require accommodations are encouraged to register with Accessibility Services upon registering with TRU. To determine how test and exam accommodations can be arranged, students are encouraged to contact Accessibility Services early on as it may take a few weeks for the accommodations to be arranged. It is a student's responsibility to contact Accessibility Services and meet with an Accessibility Services Advisor to access services. More information can be found in
\href{https://www.tru.ca/current/academic-supports/as.html}{https://www.tru.ca/current/academic-supports/as.html}.

\section*{Attendance Regulations}
A registered student who does not attend the first two events (e.g., lectures/labs/etc.) of the course and who has not made prior arrangements acceptable to the instructor may, at the discretion of the instructor, be considered to have withdrawn from the course and have his/her course registration deleted. A registered student is expected to attend a minimum of 90\% of lectures and seminars for which he/she is enrolled. In the case of deficient attendance without cause, a student may, on recommendation of the instructor to the instructor’s Dean or Chairperson, be withdrawn from a course. Admission to a lecture or seminar may be refused by the instructor for lateness, class misconduct, or failure to complete required work. 
Academic Integrity Policy

TRU students are required to comply with the standards of academic integrity set out in Student Academic Integrity policy (ED 5-0), available at TRU website. Cheating, academic misconduct, fabrication, and plagiarism could result in failure of course or even suspension from TRU. 

\section*{Prior Learning Assessment and Recognition/Challenges}

Students may receive credit for Prior Learning Assessment and Recognition (PLAR) through a formal process designed by a qualified specialist approved by the Department of Mathematics and Statistics. More information can be obtained from the Office of the Registrar.

\section*{Educational Aids}
\educationalaids

\section*{Science Help Centre}
All students are welcome to consult with a math tutor on a drop-in basis, free of charge, at the Science Help Centre, which is located in Science 201.  More information is available on the following webpage: \href{https://www.tru.ca/science/programs/math/math-help-centre.html}{https://www.tru.ca/science/programs/math/math-help-centre.html}. 

\newpage
\section*{Course Topics}
\coursetopics

\end{document}
