\documentclass[11pt,fleqn,dvipsnames,usenames]{article}

% to keep this file less overwhelming
% packages to include

\usepackage[dvipsnames, table]{xcolor}

\usepackage{
  amsthm,
  amsmath,
  amssymb, 
  arydshln, % for hyphenated lines in block matrices
  fancyhdr, % needed for header at top of each page
  graphicx, % to include pictures
  mathtools, % for a longer arrow
  multicol, % displaying enumerates and itemizes into multiple columns
  multirow, % for tables
  multido, % for TOC
  pgfplots, % for axis environment within tikz pictures
  systeme,
  tikz,
}

\usepackage[utf8]{inputenc}
\usepackage{color,soul}

\usepackage[inline, shortlabels]{enumitem}
\usepackage[hidelinks]{hyperref}


% global constants
\newcommand{\term}{Winter 2024}

% mathbb aliases
\newcommand{\COMPLEX}{\mathbb{C}}
\newcommand{\REAL}{\mathbb{R}}
\newcommand{\NATURAL}{\mathbb{N}}
\newcommand{\INTEGER}{\mathbb{Z}}

% for financial stuff
\newcommand{\dollar}{\mathrm{\$}}

% nicer looking trig functions
\newcommand{\SIN}[1]{\sin\left(#1\right)}
\newcommand{\COS}[1]{\cos\left(#1\right)}
\newcommand{\TAN}[1]{\tan\left(#1\right)}
\newcommand{\CSC}[1]{\csc\left(#1\right)}
\newcommand{\SEC}[1]{\sec\left(#1\right)}
\newcommand{\COT}[1]{\cot\left(#1\right)}

% automatically resize set brackets
\newcommand{\SET}[1]{\left\{#1\right\}}

% sums and products
\newcommand{\SUM}{\displaystyle\sum\limits}
\newcommand{\PROD}{\displaystyle\prod\limits}
\newcommand{\of}{\circ}
\newcommand{\restrict}[1]{\raisebox{-.5ex}{$|$}_{#1}}

% set intersection and union
\newcommand{\CAP}{\displaystyle\bigcap\limits}
\newcommand{\CUP}{\displaystyle\bigcup\limits}

% max and min
\newcommand{\MAX}[1]{\ensuremath{\max\left(#1\right)}}
\newcommand{\MIN}[1]{\ensuremath{\min\left(#1\right)}}

% for writing logic within mathematics environment
\newcommand{\FORALL}{\ensuremath{\text{ for all }}}
\newcommand{\FORSOME}{\ensuremath{\text{ for some }}}

% matrix notation
\newcommand{\MATRIX}[2]{\ensuremath{\left[\begin{array}{#1}#2\end{array}\right]}}
\newcommand{\COLUMN}[1]{\ensuremath{\left[\begin{array}{r}#1\end{array}\right]}}

% vector notation
%\newcommand{\vv}{\overset{\rightharpoonup}}
\newcommand{\vv}[1]{{\bf #1}}
\newcommand{\arr}{\overrightarrow}

% dot product
\newcommand{\dotp}{{\scriptstyle\bullet}}

% Text macros
\newcommand{\KER}[1]{\ensuremath{\text{ker}\left(#1\right)}}
\newcommand{\IMG}[1]{\ensuremath{\text{im}\left(#1\right)}}
\newcommand{\CHAR}[1]{\ensuremath{\text{char}\left(#1\right)}}
\newcommand{\BIGO}[1]{\ensuremath{\mathcal{O}\left(#1\right)}}
\newcommand{\TR}[1]{\ensuremath{\text{tr}\left(#1\right)}}

% abbreviations
\newcommand{\ds}{\displaystyle}
\newcommand{\md}{\mdseries}
\newcommand{\vsp}{\vspace{0.5cm}}
\newcommand{\smsp}{\vspace{0.25cm}}
\newcommand{\hsp}{\hspace{0.25cm}}

% new operators
\DeclareMathOperator\SPAN{Span}
\newcommand{\SPANOF}[1]{\ensuremath{\SPAN\left\{#1\right\}}}
\DeclareMathOperator\PROJ{proj}
\DeclareMathOperator\PERP{perp}

% quick abbreviations to avoid using latex environments
\newcommand{\answer}{\noindent \textbf{Answer:} }
\newcommand{\answers}{\noindent \textbf{Answers:} }
\newcommand{\application}{\noindent \textbf{Application:} }
\newcommand{\caution}{\noindent \textbf{Caution:} }
\newcommand{\conclusion}{\noindent \textbf{Conclusion:} }
\newcommand{\consequence}{\noindent \textbf{Consequence:} }
\newcommand{\defn}{\noindent \textbf{Definition:} }
\newcommand{\details}{\noindent \textbf{Details:} }
\newcommand{\example}{\noindent \textbf{Example:} }
\newcommand{\examples}{\noindent \textbf{Examples:} }
\newcommand{\exception}{\noindent \textbf{Exception:} }
\newcommand{\exercise}{\noindent \textbf{Exercise:} }
\newcommand{\exercises}{\noindent \textbf{Exercises:} }
\newcommand{\fact}{\noindent \textbf{Fact:} }
\newcommand{\facts}{\noindent \textbf{Facts:} }
\newcommand{\formula}{\noindent \textbf{Formula:} }
\newcommand{\goal}{\noindent \textbf{Goal:} }
\newcommand{\goals}{\noindent \textbf{Goals:} }
\newcommand{\hint}{\noindent \textbf{Hint:} }
\newcommand{\idea}{\noindent \textbf{Idea:} }
\newcommand{\illustration}{\noindent \textbf{Illustration:} }
\newcommand{\important}{\noindent \textbf{Important:} }
\newcommand{\midea}{\noindent \textbf{Main Idea:} }
\newcommand{\motivation}{\noindent \textbf{Motivation:} }
\newcommand{\nthm}[1]{\noindent \textbf{Theorem} (\textit{#1}):}
\newcommand{\notation}{\noindent \textbf{Notation:} }
\newcommand{\note}{\noindent \textbf{Note:} }
\newcommand{\notes}{\noindent \textbf{Notes:} }
\newcommand{\observation}{\noindent \textbf{Observation:} }
\newcommand{\observations}{\noindent \textbf{Observations:} }
\newcommand{\pict}{\noindent \textbf{Picture:} }
\newcommand{\plan}{\noindent \textbf{Plan:} }
\newcommand{\prf}{\noindent \textbf{Proof:} }
\newcommand{\problem}{\noindent \textbf{Problem:} }
\newcommand{\properties}{\noindent \textbf{Properties:} }
\newcommand{\question}{\noindent \textbf{Question:} }
\newcommand{\questions}{\noindent \textbf{Questions:} }
\newcommand{\recall}{\noindent \textbf{Recall:} }
\newcommand{\reason}{\noindent \textbf{Reason:} }
\newcommand{\remark}{\noindent \textbf{Remark:} }
\newcommand{\remarks}{\noindent \textbf{Remarks:} }
\newcommand{\reminder}{\noindent \textbf{Reminder:} }
\newcommand{\solution}{\noindent \textbf{Solution:} }
\newcommand{\nsolution}[1]{\noindent \textbf{Solution #1:} }
\newcommand{\strategy}{\noindent \textbf{Strategy:} }
\newcommand{\summary}{\noindent \textbf{Summary:} }
\newcommand{\terminology}{\noindent \textbf{Terminology:} }
\newcommand{\thm}{\noindent \textbf{Theorem:} }
\newcommand{\work}{\noindent \textbf{Work:} }


% Where to look for pngs and jpegs
\graphicspath{{Images//}}

\usepackage[includehead, includefoot, left= 2cm, top =1.5cm, bottom = 1.5cm, textwidth=17.5cm]{geometry}

\usepackage{pifont, amsmath}

\pagestyle{fancy}
\fancyhf{}
\renewcommand{\headrulewidth}{1pt}
%\fancyhead[R]{\bfseries\sffamily\thepage}
\fancyfoot[C]{\thepage}
\fancyhead[L]{\nouppercase{\bfseries\sffamily\leftmark}}

% used when adding fill-in-the-blanks for students
\newcommand{\blank}[1]{\underline{\hspace{#1}}}

% indents annoy me, and so does repeatedly typing \noindent
\newcommand{\p}{\noindent}
\newcommand{\ENDPRF}{\hfill $\blacksquare$}

\begin{document}

\fancyhead[L]{Math 2210}
\fancyhead[C]{\includegraphics[width=5cm, trim= 0 0.4cm 0 0]{TRU_logo}}
\fancyhead[R]{\term}
\renewcommand{\headrulewidth}{0.4pt}

\setulcolor{red}

\setcounter{section}{1}
\section{Congruence in $\INTEGER$ and Modular Arithmetic}
\setcounter{subsection}{0}


\subsection{Congruence and Congruence Classes}

\begin{definition}
Let $a,b,n\in\INTEGER$ with $n > 0$.  We say that $a$ is \DEF{congruent to $b$ modulo $n$} if $n|b-a$.
\end{definition}

\notation In this case we write $a\equiv b$ (mod $n$), or $a\overset{n}{\equiv}b$.  Otherwise we write $a\not\equiv b$ (mod $n$), or $a\not\overset{n}{\equiv}b$.

\example $9\overset{2}{\equiv}5$ and $9\overset{4}\equiv{5}$ but $9\not\overset{2}{\equiv}5$.

\begin{theorem}\label{congruenceproperties}Let $n\in\INTEGER$ with $n > 0$.
\begin{enumerate}[(1)]
\item $a\overset{n}{\equiv} a$ for any $a\in\INTEGER$.
\item If $a,b\in\INTEGER$ such that $a\overset{n}{\equiv} b$, then $b\overset{n}{\equiv} a$.
\item If $a,b,c\in\INTEGER$ such that $a\overset{n}{\equiv} b$ and $b\overset{n}{\equiv} c$, then $a\overset{n}{\equiv} c$.
\end{enumerate}
\end{theorem}
%
\begin{proof}
\begin{enumerate}[(1)]
\item For any $a\in\INTEGER$, we have $n|a-a$, since $a-a = 0$.  Hence $a\overset{n}{\equiv} a$.
\item For any $a,b\in\INTEGER$, if $a\overset{n}{\equiv} b$, then $n|b-a$.  Hence $n|a-b$, so $b\overset{n}{\equiv} a$.
\item For any $a,b,c\in\INTEGER$, if $a\overset{n}{\equiv} b$ and $b\overset{n}{\equiv} c$, then $n$ divides both $b-a$ and $c-b$.  It follows that $n$ also divides $c-a = (c-b) + (b-a)$.  Hence $a\overset{n}{\equiv} c$.\hfill \qedhere
\end{enumerate}
\end{proof}
%
\begin{theorem}\label{congruencereplacement}
If $a,b,c,d\in\INTEGER$ such that $a\overset{n}{\equiv} b$ and $c\overset{n}{\equiv} d$, then $a + c\overset{n}{\equiv} b + d$ and $ac\overset{n}{\equiv} bd$.
\end{theorem}
%
\begin{proof}
By assumption, $n|b-a$ and $n|d-c$.  It follows that
\begin{center}
$(b+d) - (a+c) = (b-a) + (d-c)$
\end{center}
and
\begin{center}
$bd - ac = bd - ad + ad - ac = d(b-a) + a(d-c)$.
\end{center}
are both divisible by $n$.  Hence $a + c\overset{n}{\equiv} b + d$ and $ac\overset{n}{\equiv} bd$.
\end{proof}
\vsp

\p In particular, if $a,b\in\INTEGER$ such that $a\overset{n}{\equiv} b$, then for any $c\in\INTEGER$, $a + c\overset{n}{\equiv} b + c$ and $ac\overset{n}{\equiv} bc$ for any $c\in\INTEGER$.

\begin{example}
Find all solutions to the congruence $2x\overset{5}{\equiv} 3$.
\end{example}

\solution Note that $5|2x - 3$ whenever there exists $s\in\INTEGER$ such that $2x - 3 = 5s$.  In other words, the congruence is satisfied whenever
\begin{center}
$\ds{x = \frac{5s + 3}{2}}$
\end{center}
for some integer $s\in\INTEGER$.
\vsp

\begin{exercises} \phantom{ }
\begin{enumerate}[(1)]
\item Find all solutions to the congruence $3x\overset{7}{\equiv} 4$.
\item Which of the following congruences have solutions?
\begin{enumerate}[(a)]
\item $x^{2}\overset{3}{\equiv} 1$
\item $x^{2}\overset{7}{\equiv} 2$
\item $x^{2}\overset{11}{\equiv} 3$
\end{enumerate}
\item Let $a,b,n\in\INTEGER$ with $n > 0$.  Prove that if $a\overset{n}{\equiv}b$, then $a^2 + b^2\overset{n^2}{\equiv}2ab$.
\item Suppose $a,b\in\INTEGER$ such that $a\overset{p}{\equiv}b$ for every prime $p$.  Prove that $a = b$.
\end{enumerate}
\end{exercises}
%
\begin{definition}\label{congruenceclassmodndefn}
Let $a,n\in\INTEGER$ with $n > 0$.  The \DEF{congruence class of $a$ modulo $n$} is defined by
\begin{center}
$[a] = \SET{b\in\INTEGER : a\overset{n}{\equiv}b} = \SET{a + kn:k\in\INTEGER}$.
\end{center}
\end{definition}
%
\begin{example}
If $n = 5$, then
\begin{itemize}[\ ]
\item $[0] = \SET{0 + 5k:k\in\INTEGER} = \SET{5k:k\in\INTEGER}$
\item $[1] = \SET{1 + 5k:k\in\INTEGER} = \SET{1 + 5k:k\in\INTEGER}$
\item $[2] = \SET{2 + 5k:k\in\INTEGER} = \SET{2 + 5k:k\in\INTEGER}$
\item $[3] = \SET{3 + 5k:k\in\INTEGER} = \SET{3 + 5k:k\in\INTEGER}$
\item $[4] = \SET{4 + 5k:k\in\INTEGER} = \SET{4 + 5k:k\in\INTEGER}$
\item $[5] = \SET{5 + 5k:k\in\INTEGER} = \SET{5(1 + k):k\in\INTEGER} = \SET{5k:k\in\INTEGER} = [0]$
\end{itemize}
\p In a similar fashion, $[6] = [1]$ and $[7] = [2]$, etc.
\end{example}
%
\begin{theorem}\label{congruenceequalitycriterion}
Let $a,b,n\in\INTEGER$ with $n > 0$.  $a\overset{n}{\equiv}b$ if and only if $[a] = [b]$.
\end{theorem}
%
\begin{proof}
$(\Rightarrow)$ It must be verified that if $a\overset{n}{\equiv}b$, then $[a] = [b]$.

$(\subset)$ Suppose $c\in[a]$.  Then $c\overset{n}{\equiv}a$.  By assumption, $a\overset{n}{\equiv}b$.  Hence by Theorem \ref{congruenceproperties} (3), we must have $c\overset{n}{\equiv}b$, i.e, $c\in[b]$.

$(\supset)$ This direction follows from an identical argument.

$(\Leftarrow)$ Suppose $[a] = [b]$.  By Theorem \ref{congruenceproperties}, $a\overset{n}{\equiv}a$ and hence $a\in [a]$.  It follows that $a\in[b]$ and therefore $a\overset{n}{\equiv}b$, as required.
\end{proof}
%
\begin{corollary}\label{disjointorsame}
For any $a,b,n\in\INTEGER$ with $n > 0$, either $[a] = [b]$ or $[a]\cap [b] = \emptyset$.
\end{corollary}
%
\begin{proof}
Suppose $a,b\in\INTEGER$ with $[a]\cap[b]\neq\emptyset$.  Then choose $c\in[a]\cap[b]$.  It follows that $a\overset{n}{\equiv}c$ and $b\overset{n}{\equiv}c$.  Therefore $a\overset{n}{\equiv}c\overset{n}{\equiv}b$.  By Theorem \ref{congruenceequalitycriterion}, it follows that $[a] = [b]$.
\end{proof}
%
\begin{corollary}
Let $n\in\INTEGER$ with $n>0$.
\begin{enumerate}[(1)]
\item If $a\in\INTEGER$ and $r$ is the remainder when $a$ is divided by $n$, then $[a] = [r]$.
\item There are exactly $n$ distinct congruence classes modulo $n$, namely $[0], [1], [2],\ldots, [n-1]$.
\end{enumerate}
\end{corollary}
%
\begin{proof}~
\begin{enumerate}[(1)]
\item Use the Division algorithm to write $a = nq + r$ for some $r,n\in\INTEGER$ with $0\leq r < n$.  Since $a - r = nq$, $n|a-r$ and hence $a\overset{n}{\equiv}r$.  Therefore $[a] = [r]$ by Theorem {congruenceequalitycriterion}.
\item First, note that $[0], [1], [2],\ldots, [n-1]$ are all distinct congruence classes.  Indeed if $[i] = [j]$ for some $i,j\in\INTEGER$ with $0\leq i,j < n$, then $i\overset{n}{\equiv}j$ and hence $n|i-j$.  By the way $i$ and $j$ were chosen, we must have $-n+1 < i-j < n-1$.  Since $i-j$ is divisible by $n$, we must have $i-j = 0$, or equivalently, $i=j$.  On the other hand if $a\in\INTEGER$, then by (1) $[a] = [r]$, where $r$ is the remainder of $a$ when divided by $n$.  Since $0\leq r < n-1$, this completes the proof.\qedhere
\end{enumerate}
\end{proof}
%
\begin{definition}
$\ZN$ is the set of all congruence classes modulo $n$.  That is, $\ZN = \SET{[0], [1], [2],\ldots, [n-1]}$.
\end{definition}
%
\begin{exercises}~
\begin{enumerate}
\item Decide whether each of the following statements are true.  If they are true, then prove them.  Otherwise, provide a counter-example.
\begin{enumerate}[(a)]
\item If $a^2\overset{n}{\equiv}b^2$, then either $a\overset{n}{\equiv}b$ or $a\overset{n}{\equiv}-b$.
\item If $p$ is prime and $a^2\overset{p}{\equiv}b^2$, then either $a\overset{p}{\equiv}b$ or $a\overset{p}{\equiv}-b$.
\item If $[a] = [b]$ in $\ZN$, then $\GCD{a,n} = \GCD{b,n}$.
\end{enumerate}
\item If $[a] = [1]$ in $\ZN$, prove that $\GCD{a,n} = 1$.  Demonstrate also that the converse fails to be true in general.
\end{enumerate}
\end{exercises}
%
\subsection{Modular Arithmetic}
\begin{definition}
The \DEF{sum} and \DEF{product} of $[a]$ and $[b]$ in $\ZN$ is given by
\begin{center}
$[a]\oplus [b] = [a + b]$ and $[a]\odot[b] = [ab]$.
\end{center}
\end{definition}
%
\p Note that in order for these operations to be well-defined, it must be the case that the sum and product of two congruence classes do not depend on the representative chosen from the two congruence classes being added or multiplied.  That is, if $[a] = [a']$ and $[b] = [b']$, then $[a+b] = [a' + b']$ and $[ab] = [a'b']$.
\smsp

\p But this is equivalent to the requirement that $a + b\overset{n}{\equiv} a' + b'$ and $ab\overset{n}{\equiv} a'b'$ whenever $a\overset{n}{\equiv} a'$ and $b\overset{n}{\equiv}b'$.  And this follows immediately from Theorem \ref{congruencereplacement}.
%
\begin{example}In $\Z_{5}$, the addition and multiplication tables are shown below.
\vsmsp

\begin{center}
\bgroup
\def\arraystretch{1.5}
\begin{tabular}{c|ccccc}
$\oplus$ & [0] & [1] & [2] & [3] & [4]\\
\hline
$[0]$ & [0] & [1] & [2] & [3] & [4]\\
$[1]$ & [1] & [2] & [3] & [4] & [0]\\
$[2]$ & [2] & [3] & [4] & [0] & [1]\\
$[3]$ & [3] & [4] & [0] & [1] & [2]\\
$[4]$ & [4] & [0] & [1] & [2] & [3]
\end{tabular}
\hspace{3cm}
\begin{tabular}{c|ccccc}
$\odot$ & [0] & [1] & [2] & [3] & [4]\\
\hline
$[0]$ & [0] & [0] & [0] & [0] & [0]\\
$[1]$ & [0] & [1] & [2] & [3] & [4]\\
$[2]$ & [0] & [2] & [4] & [1] & [3]\\
$[3]$ & [0] & [3] & [1] & [4] & [2]\\
$[4]$ & [0] & [4] & [3] & [2] & [1]
\end{tabular}
\egroup
\end{center}
\end{example}
%
\exercise Write out the addition and multiplication tables for $\Z_6$.
\vsp
%
\begin{definition}
A \DEF{relation} on a set $A$ is a subset of $A\times A$.
\end{definition}
\vsp

\notation If $T$ is a relation on a set $A$, and $a,b\in A$ are such that $(a,b)\in T$, then we write $a\sim b$.
\vsp

\begin{definition}
A relation $T$ on a set $A$ is said to be an \DEF{equivalence relation} if the following conditions are met:
\begin{enumerate}[(1)]
\item $a\sim a$, for all $a,\in A$.
\item $a\sim b \Rightarrow b\sim a$, for all $a,b\in A$.
\item $a\sim b$ and $b\sim c \Rightarrow a\sim c$, for all $a,b,c\in A$.
\end{enumerate}
\end{definition}
%
\begin{examples}\label{equivalencerelationexamples}
\begin{enumerate}[(a)]
\item Define the relation $T$ on $\REAL$ by $T = \SET{(x,x):x\in\REAL}\cup \SET{(x,-x):x\in\REAL}$.  Then $(x,y)\in T$ exactly when $x = y$ or $x = -y$.  That is, $x\sim y$ if and only if $|x| = |y|$.
\item $S = \SET{\Big((x,y), (w,z)\Big):x,y,w,z\in\REAL\text{ with }x = y}$ is a relation on $\REAL^{2}$.  In this case, $(x,y)\sim (w,z)$ if and only if $x = w$.
\item Define another relation on $\REAL^{2}$ by $(x,y)\sim (w,z)$ if and only if both $(x,y)$ and $(w,z)$ are the same distance from the origin.
\item Let $A$ be the set of all students in this class.  Define a relation on $A$ by imposing $a\sim b$ whenever $a$ and $b$ were born in the same month.
\item Let $n\in \INTEGER$ with $n > 0$ and define the relation $T = \SET{(a,b):n|b-a}$.  In other words, $a\sim b$ if and only if $a\overset{n}{\equiv}b$.  By Theorem \ref{congruenceproperties}, $T$ is an equivalence relation.  For this reason, equivalence relations may be thought of as a generalization of congruence modulo $n$.
\end{enumerate}
\end{examples}
%
%\begin{definition}
%A partition of a set $A$ is a collection $\mathcal{T}$ of subsets of $A$ with the properties:
%\begin{enumerate}[(1)]
%\item $A = \CUP_{X\in\mathcal{T}} X$ (i.e, for any $x\in A$, there exists $X\in\mathcal{T}$ such that $x\in X$) 
%\item $X_{1}\cap X_{2} = \emptyset$ for any $X_{1}, X_{2}\in\mathcal{T}$.
%\end{enumerate}
%\end{definition}
%
\begin{definition}
If $T$ is an equivalence relation on a set $A$, then for each $a\in A$, then the \DEF{congruence class of $a$} is given by
\begin{center}
$[a] = \SET{b\in A: a\sim b}$
\end{center}
\end{definition}
%
\begin{theorem}
If $T$ is an equivalence relation on a set $A$, then $\SET{[a]:a\in A}$ forms a partition of $A$, i.e. a disjoint collection of subsets of $A$ that cover $A$.
\end{theorem}
%
\begin{proof}
The fact that $\SET{[a]:a\in A}$ covers $A$ is immediate, since for any $a\in A$, we have $a\in [a]$.  To see that congruence classes are disjoint, copy the proof of
Corollary \ref{disjointorsame}.
\end{proof}
%
\begin{examples}
Identify the congruence classes in each of Examples \ref{equivalencerelationexamples}.
\end{examples}
%
\begin{solution}
\begin{enumerate}[(a)]
\item $[0] = \SET{x\in\REAL: 0 = x\text{ or }0 = -x} = \SET{0}$, and for all $x\in \REAL$ with $x\neq 0$, $[x] = \SET{y\in\REAL: x = y\text{ or }x = -y}= {x,-x}$.
\item For each $(x,y)\in\REAL^{2}$, $[(x,y)] = \SET{(w,z)\in\REAL^{2}:x = y}$ is the vertical line through the point $(x,y)$.
\item For each $(x,y)\in\REAL^{2}$, $[(x,y)] = \SET{(w,z)\in\REAL^{2}: \sqrt{x^2 + y^2} = \sqrt{w^2 + z^2}}$, which is the circle centred at the origin which contains the point $(x,y)$.
\item For each student $a\in A$, $[a] = \SET{b\in A:a\text{ and }b\text{ were born in the same month.}}$.
\item $[a] = \SET{b\in\INTEGER:a\sim b} = \SET{b\in\INTEGER: a\overset{n}{\equiv}b} = \SET{a+kn:k\in\INTEGER}$, which is Definition \ref{congruenceclassmodndefn}.
\end{enumerate}
\end{solution}
%
\begin{theorem}  Let $n\in\INTEGER$ with $n>0$.  For any $[a], [b], [c]\in\Z_n$,
\begin{enumerate}[(1)]
\item $[a] \oplus [b] = [b]\oplus [a]$ \hfill (Commutativity of addition)
\item $[a] \odot [b] = [b]\odot [a]$ \hfill (Commutativity of multiplication)
\item $[a]\oplus ([b]\oplus [c]) = ([a] \oplus [b])\oplus [c]$ \hfill (Associativity of addition)
\item $[a]\odot ([b]\odot [c]) = ([a] \odot [b])\odot [c]$ \hfill (Associativity of multiplication)
\item $[a]\oplus [0] = [a]$ \hfill (Additive Identity)
\item $[a]\odot [1] = [a]$ \hfill (Multiplicative Identity)
\item There exists $[d]\in \Z_n$ such that $[a] \oplus [d] = [0]$ \hfill (Additive inverse)
\item $[a]\odot ([b]\oplus [c]) = [a]\odot [b] \oplus [a]\odot [c]$ \hfill (Distributivity)
\end{enumerate}
\end{theorem}
\vsp

%
\p Up until now, $\oplus$ and $\odot$ have been used to denote addition and multiplication of congruence classes in $\Z_n$, in order to distinguish these operations from ordinary addition and multiplication of integers.  But it is a standard approach to make this distinction using context alone, and there is no harm in writing the sum and product of $[a],[b]\in \Z_n$ as $[a] + [b]$ and $[a]\cdot [b]$, respectively.
\vsp

\p Moreover, unless there is a risk of confusion, the congruence class $[a]\in\Z_n$ will simply be denoted $a$.  For example, calculations in $\Z_5$ such as
\begin{center}
$[13] + [11] = [3] + [1] = [4]$
\end{center}
may be abbreviated as
\begin{center}
$13 + 11 = 3 + 1 = 4$.
\end{center}
On a similar note, it is not incorrect to make statements like $17\in\Z_5$ or $-8\in\Z_5$.  But in this case both $17$ and $-8$ are both labels for the same congruence class $[2]\in\Z_5$.
\vsp

\p In this sense, we may think of $\Z_n$ as simply the set of all integers, but with the understanding that any two integers that differ by a multiple of $n$ are in fact equal.  We make use of this convention in the following example.
\vsp

%
\begin{example}
Prove that $(a+b)^{5} = a^{5} + b^{5}$ in $\Z_5$.
\end{example}
\begin{solution}
Expanding in the usual way, we get
\begin{center}
$(a + b)^5 = a^{5} + \underbrace{5a^4b}_{=0} + \underbrace{10a^3b^2}_{=0} + \underbrace{10a^2b^3}_{=0} + \underbrace{5ab^4}_{=0} + b^5 = a^5 + b^5$,
\end{center}
where we have used the fact that $5a^4b, 10a^3b^2, 10a^2b^3$, and $5ab^4$ are all multiples of $5$.
\end{solution}
%

\subsection{The Structure of $\Z_p$ when $p$ is Prime}
\p Unless otherwise stated, assume $n\in\INTEGER$ with $n > 0$.  
\begin{definition}
A non-zero $a\in\Z_n$ is a \DEF{zero-divisor} if there exists a non-zero $b\in\Z_n$ such that $ab = 0$.
\end{definition}
%
\begin{examples}~
\begin{enumerate}[(a)]
\item $\Z_2$ and $\Z_3$ have no zero-divisors.
\item $2\in\Z_4$ is a zero-divisor, since $2\cdot 2 = 4 = 0$.
\item $2,4,6\in\Z_8$ are all zero-divisors, since $2\cdot 4 = 8 = 0$ and $4\cdot 6 = 24 = 0$.
\end{enumerate}
\end{examples}
\vsp

\question What are the zero-divisors of $\Z_n$?
\vsp

\begin{theorem}\label{primepnozerodivisors}
$p\in\INTEGER$ is prime if and only if $\Z_p$ has no zero-divisors.
\end{theorem}
%
\begin{proof}\phantom{-}

$(\Rightarrow)$ Suppose $p$ is prime.  If $a,b\in\Z_p$ such that $ab = 0$, then $p|ab$.  Since $p$ is prime, we have $p|a$ or $p|b$.  Hence either $a=0$ or $b=0$.

$(\Leftarrow)$ If $p$ is not prime, then there exists $m,n\in\INTEGER$ with $1\leq m\leq n < p$ such that $p = mn$.  It follows that $p\ndiv m$ and $p\ndiv n$.  Hence $mn = 0$ but $m\neq 0$ and $n\neq 0$.
\end{proof}
%
\begin{theorem}
Let $p$ be prime.  An element $\Z_{p^2}$ is a zero-divisor if and only if it is a non-zero multiple of $p$.
\end{theorem}
%
\begin{proof}\phantom{-}

$(\Leftarrow)$ If $a\in\Z_{p^2}$ is a non-zero multiple of $p$, then $a = kp$ for some $k\in\INTEGER$.  Since $p^2\ndiv p$, $p\neq 0$ and $ap = kp^2 = k(0) = 0$.  Hence $a$ is a zero-divisor.

$(\Rightarrow)$ Suppose $a,b\in\Z_{p^2}$ with $a\neq0$, $b\neq0$, and $ab = 0$.  Then $p^2|ab$, but $p^2\ndiv a$ and $p^2\ndiv b$.  The only way this can be true is if $p|a$ and $p|b$.  Hence both $a$ and $b$ are multiples of $p$.
\end{proof}
\vsp

\begin{examples}\phantom{-}

\begin{enumerate}[(a)]
\item The zero-divisors in $\Z_{25}$ are $5, 10, 15$, and $20$.
\item The zero-divisors in $\Z_9$ are $3$ and $6$.
\end{enumerate}
\end{examples}
%
\begin{definition}
An element $a\in\Z_n$ is called a \DEF{unit} if there exists $b\in \Z_n$ such that $ab = 1$.  In this case $b$ is called the \DEF{inverse} of $a$ and we write $a^{-1} = b$.
\end{definition}
%
\newpage
%
\begin{example}
The multiplication table for $\Z_6$ is shown below.  
\bgroup
\begin{center}
\def\arraystretch{1.5}
\begin{tabular}{c|cccccc}
$\odot$ & [0] & [1] & [2] & [3] & [4] & [5]\\
\hline
$[0]$ & $[0]$ & $[0]$ & $[0]$ & $[0]$ & $[0]$ & $[0]$\\
$[1]$ & $[0]$ & $[1]$ & $[2]$ & $[3]$ & $[4]$ & $[5]$\\
$[2]$ & $[0]$ & $[2]$ & $[4]$ & $[0]$ & $[2]$ & $[4]$\\
$[3]$ & $[0]$ & $[3]$ & $[0]$ & $[3]$ & $[0]$ & $[3]$\\
$[4]$ & $[0]$ & $[4]$ & $[2]$ & $[0]$ & $[4]$ & $[2]$\\
$[5]$ & $[0]$ & $[5]$ & $[4]$ & $[3]$ & $[2]$ & $[1]$ 
\end{tabular}
\end{center}
\egroup
\p It can be seen that $1$ and $5$ are units, but $0,2,3$, and $4$ are not.
\end{example}

\begin{exercise} Show that a zero-divisor in $\Z_n$ cannot be a unit.\end{exercise}

\begin{example}
$3\in\Z_9$ is not a unit, because it is a zero-divisor.
\end{example}
%
\begin{theorem}
A positive integer $p$ is prime if and only if every non-zero element of $\Z_p$ is a unit.
\end{theorem}
%
\begin{lemma}\label{gcdunitlemma}
$a\in\Z_n$ is a unit if and only if $a$ and $n$ are co-prime.
\end{lemma}
%
\begin{proof}\phantom{-}

$(\Rightarrow)$ If $a$ and $n$ are not co-prime, then let $d>1$ be a common divisor of $a$ and $n$.  It follows that
\begin{center}
$a(n/d) = d(a/d)(n/d) = (a/d)n = 0$
\end{center}
in $\Z_n$.  Since $n\ndiv n/d$, $a$ is a zero-divisor and thus cannot be a unit.

$(\Leftarrow)$ If $a$ and $n$ are co-prime, then $\GCD{a,n} = 1$ and we may choose $s,t\in\INTEGER$ satisfying $sa + tn = 1$.  Then $sa\overset{n}{\equiv} 1$ and hence $a$ is a unit with $a^{-1} = s$.
\end{proof}
%
\begin{theorem}
A positive integer $p$ is prime if and only if every non-zero element of $\Z_p$ is a unit.
\end{theorem}
%
\begin{proof}\phantom{-}

$(\Rightarrow)$ Suppose $p$ is prime, and let $a\in\Z_p$ be non-zero.  Then $p\ndiv a$ and hence $\GCD{a,p} = 1$.  By Lemma \ref{gcdunitlemma}, $a$ is a unit.

$(\Leftarrow)$ Suppose every non-zero element of $\Z_p$ is a unit.  Then $\Z_p$ cannot contain a zero-divisor.  By Theorem \ref{primepnozerodivisors}, $p$ is prime.
\end{proof}

\begin{corollary}
Let $a,b\in\Z_n$.  If $a$ is a unit, then the equation $ax = b$ has a unique solution in $\Z_n$.
\end{corollary}
%
\begin{proof}\phantom{-}

(Existence) Let $a,b\in\Z_n$ with $a\neq 0$.  Since $a(a^{-1}b) = (aa^{-1})b = 1b = b$, $x = a^{-1}b$ is a solution.

(Uniqueness) Suppose for a contradiction that $x_{1},x_{2}\in\Z_n$ satisfy $ax_{1} = b$ and $ax_{2} = b$ with $x_{1}\neq x_{2}$.  Then $x_{1} - x_{2}\neq 0$, and $a(x_{1} - x_{2}) = ax_{1} - ax_{2} = b - b = 0$.  Hence $a$ is a zero-divisor which is a contradiction, since $a$ is a unit.
\end{proof}
%
\begin{corollary}
Let $p$ be a prime.  For any $a,b\in\Z_p$ with $a\neq 0$, the equation $ax = b$ has a unique solution in $\Z_p$.
\end{corollary}
%
\begin{corollary}
For any $a,b\in\Z_n$, if $a$ and $n$ are co-prime, the equation $ax = b$ has a unique solution in $\Z_n$.
\end{corollary}
%
\begin{remark}
The inverse of a unit is unique.  That is, if $a\in\Z_n$ is a unit with $ab = 1$ and $ac = 1$, then $b=c$.  To see why this is true, compute $b = (ac)b = (ca)b = c(ab) = c\cdot 1 = c$.
\end{remark}
%
\begin{example}
Solve $24x = 5$ in $\Z_{95}$.
\end{example}
\begin{solution}
The Euclidean Algorithm may be used to show that $\GCD{24,95} = 1$ and that $4\cdot 24\nolinebreak + \nolinebreak(-1)95 =\nolinebreak 1$.  Therefore in $\Z_{95}$, $24^{-1} = 4$ and we may compute $x = 24^{-1}\cdot 5 = 4\cdot 5 = 20$.
\end{solution}
%
\begin{theorem}\label{numbersolutionsinZn}
Let $a,b,n\in\INTEGER$ with $n > 1$ and set $d = \GCD{a,n}$.
\begin{enumerate}[(1)]
\item The equation $ax = b$ has solutions in $\Z_n$ if and only if $d|b$.
\item If $d|b$, then $ax = b$ has d distinct solutions in $\Z_n$.
\end{enumerate}
\end{theorem}
%
\begin{example}
Find all solutions to $18x = 12$ in $\Z_{24}$.
\end{example}
\begin{solution}
Note first that $\GCD{18,12} = 6$, and $6|12$.  Therefore by Theorem \ref{numbersolutionsinZn}, there are $6$ solutions to $18x=12$ in $\Z_{24}$.  To find them, observe that for any $x\in\Z_{24}$,
\begin{align*}
18x = 12\text{ in }\Z_{24} \Leftrightarrow 18x\overset{24}{\equiv}12 \Leftrightarrow 24|18x - 12 &\Leftrightarrow 18x - 12 = 24k\text{ for some }k\in\INTEGER\\
&\Leftrightarrow 3x - 2 = 4k\text{ for some }k\in\INTEGER\\
&\Leftrightarrow 4|3x - 2\\
&\Leftrightarrow 3x\overset{4}{\equiv}2\\
&\Leftrightarrow 3x = 2\text{ in }\Z_{4} 
\end{align*}
Note that $3$ is a unit and is its own inverse in $\Z_4$ since $3\cdot 3 = 9 = 1$ in $\Z_4$.  Hence in $\Z_4$, the equation $3x=2$ is equivalent to $x = 2$.  Of all the possible values of $x$ in the set
\begin{center}
$Z_{24} = \SET{0,1,2,3,4,5,6,7,8,9,10,11,12,13,14,15,16,17,18,19,20,21,22,23}$,
\end{center}
the ones that satisfy $x = 2$ in $\Z_4$ are $2,6,10,14,18$, and $22$.  These are the solutions we seek.
\end{solution}
\end{document}

