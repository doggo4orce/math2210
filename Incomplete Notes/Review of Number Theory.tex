\documentclass[11pt,fleqn,dvipsnames,usenames]{article}

% to keep this file less overwhelming
% packages to include

\usepackage[dvipsnames, table]{xcolor}

\usepackage{
  amsthm,
  amsmath,
  amssymb, 
  arydshln, % for hyphenated lines in block matrices
  fancyhdr, % needed for header at top of each page
  graphicx, % to include pictures
  mathtools, % for a longer arrow
  multicol, % displaying enumerates and itemizes into multiple columns
  multirow, % for tables
  multido, % for TOC
  pgfplots, % for axis environment within tikz pictures
  systeme,
  tikz,
}

\usepackage[utf8]{inputenc}
\usepackage{color,soul}

\usepackage[inline, shortlabels]{enumitem}
\usepackage[hidelinks]{hyperref}


% global constants
\newcommand{\term}{Winter 2024}

% mathbb aliases
\newcommand{\COMPLEX}{\mathbb{C}}
\newcommand{\REAL}{\mathbb{R}}
\newcommand{\NATURAL}{\mathbb{N}}
\newcommand{\INTEGER}{\mathbb{Z}}

% for financial stuff
\newcommand{\dollar}{\mathrm{\$}}

% nicer looking trig functions
\newcommand{\SIN}[1]{\sin\left(#1\right)}
\newcommand{\COS}[1]{\cos\left(#1\right)}
\newcommand{\TAN}[1]{\tan\left(#1\right)}
\newcommand{\CSC}[1]{\csc\left(#1\right)}
\newcommand{\SEC}[1]{\sec\left(#1\right)}
\newcommand{\COT}[1]{\cot\left(#1\right)}

% automatically resize set brackets
\newcommand{\SET}[1]{\left\{#1\right\}}

% sums and products
\newcommand{\SUM}{\displaystyle\sum\limits}
\newcommand{\PROD}{\displaystyle\prod\limits}
\newcommand{\of}{\circ}
\newcommand{\restrict}[1]{\raisebox{-.5ex}{$|$}_{#1}}

% set intersection and union
\newcommand{\CAP}{\displaystyle\bigcap\limits}
\newcommand{\CUP}{\displaystyle\bigcup\limits}

% max and min
\newcommand{\MAX}[1]{\ensuremath{\max\left(#1\right)}}
\newcommand{\MIN}[1]{\ensuremath{\min\left(#1\right)}}

% for writing logic within mathematics environment
\newcommand{\FORALL}{\ensuremath{\text{ for all }}}
\newcommand{\FORSOME}{\ensuremath{\text{ for some }}}

% matrix notation
\newcommand{\MATRIX}[2]{\ensuremath{\left[\begin{array}{#1}#2\end{array}\right]}}
\newcommand{\COLUMN}[1]{\ensuremath{\left[\begin{array}{r}#1\end{array}\right]}}

% vector notation
%\newcommand{\vv}{\overset{\rightharpoonup}}
\newcommand{\vv}[1]{{\bf #1}}
\newcommand{\arr}{\overrightarrow}

% dot product
\newcommand{\dotp}{{\scriptstyle\bullet}}

% Text macros
\newcommand{\KER}[1]{\ensuremath{\text{ker}\left(#1\right)}}
\newcommand{\IMG}[1]{\ensuremath{\text{im}\left(#1\right)}}
\newcommand{\CHAR}[1]{\ensuremath{\text{char}\left(#1\right)}}
\newcommand{\BIGO}[1]{\ensuremath{\mathcal{O}\left(#1\right)}}
\newcommand{\TR}[1]{\ensuremath{\text{tr}\left(#1\right)}}

% abbreviations
\newcommand{\ds}{\displaystyle}
\newcommand{\md}{\mdseries}
\newcommand{\vsp}{\vspace{0.5cm}}
\newcommand{\smsp}{\vspace{0.25cm}}
\newcommand{\hsp}{\hspace{0.25cm}}

% new operators
\DeclareMathOperator\SPAN{Span}
\newcommand{\SPANOF}[1]{\ensuremath{\SPAN\left\{#1\right\}}}
\DeclareMathOperator\PROJ{proj}
\DeclareMathOperator\PERP{perp}

% quick abbreviations to avoid using latex environments
\newcommand{\answer}{\noindent \textbf{Answer:} }
\newcommand{\answers}{\noindent \textbf{Answers:} }
\newcommand{\application}{\noindent \textbf{Application:} }
\newcommand{\caution}{\noindent \textbf{Caution:} }
\newcommand{\conclusion}{\noindent \textbf{Conclusion:} }
\newcommand{\consequence}{\noindent \textbf{Consequence:} }
\newcommand{\defn}{\noindent \textbf{Definition:} }
\newcommand{\details}{\noindent \textbf{Details:} }
\newcommand{\example}{\noindent \textbf{Example:} }
\newcommand{\examples}{\noindent \textbf{Examples:} }
\newcommand{\exception}{\noindent \textbf{Exception:} }
\newcommand{\exercise}{\noindent \textbf{Exercise:} }
\newcommand{\exercises}{\noindent \textbf{Exercises:} }
\newcommand{\fact}{\noindent \textbf{Fact:} }
\newcommand{\facts}{\noindent \textbf{Facts:} }
\newcommand{\formula}{\noindent \textbf{Formula:} }
\newcommand{\goal}{\noindent \textbf{Goal:} }
\newcommand{\goals}{\noindent \textbf{Goals:} }
\newcommand{\hint}{\noindent \textbf{Hint:} }
\newcommand{\idea}{\noindent \textbf{Idea:} }
\newcommand{\illustration}{\noindent \textbf{Illustration:} }
\newcommand{\important}{\noindent \textbf{Important:} }
\newcommand{\midea}{\noindent \textbf{Main Idea:} }
\newcommand{\motivation}{\noindent \textbf{Motivation:} }
\newcommand{\nthm}[1]{\noindent \textbf{Theorem} (\textit{#1}):}
\newcommand{\notation}{\noindent \textbf{Notation:} }
\newcommand{\note}{\noindent \textbf{Note:} }
\newcommand{\notes}{\noindent \textbf{Notes:} }
\newcommand{\observation}{\noindent \textbf{Observation:} }
\newcommand{\observations}{\noindent \textbf{Observations:} }
\newcommand{\pict}{\noindent \textbf{Picture:} }
\newcommand{\plan}{\noindent \textbf{Plan:} }
\newcommand{\prf}{\noindent \textbf{Proof:} }
\newcommand{\problem}{\noindent \textbf{Problem:} }
\newcommand{\properties}{\noindent \textbf{Properties:} }
\newcommand{\question}{\noindent \textbf{Question:} }
\newcommand{\questions}{\noindent \textbf{Questions:} }
\newcommand{\recall}{\noindent \textbf{Recall:} }
\newcommand{\reason}{\noindent \textbf{Reason:} }
\newcommand{\remark}{\noindent \textbf{Remark:} }
\newcommand{\remarks}{\noindent \textbf{Remarks:} }
\newcommand{\reminder}{\noindent \textbf{Reminder:} }
\newcommand{\solution}{\noindent \textbf{Solution:} }
\newcommand{\nsolution}[1]{\noindent \textbf{Solution #1:} }
\newcommand{\strategy}{\noindent \textbf{Strategy:} }
\newcommand{\summary}{\noindent \textbf{Summary:} }
\newcommand{\terminology}{\noindent \textbf{Terminology:} }
\newcommand{\thm}{\noindent \textbf{Theorem:} }
\newcommand{\work}{\noindent \textbf{Work:} }


\usepackage[version=4]{mhchem}

% Where to look for pngs and jpegs
\graphicspath{{Images//}}

\usepackage[includehead, includefoot, left= 2cm, top =1.5cm, bottom = 1.5cm, textwidth=17.5cm]{geometry}

\usepackage{pifont, amsmath}


\pagestyle{fancy}
\fancyhf{}
\renewcommand{\headrulewidth}{1pt}
%\fancyhead[R]{\bfseries\sffamily\thepage}
%\fancyfoot[C]{\bfseries\sffamily\thepage}
\fancyhead[L]{\nouppercase{\bfseries\sffamily\leftmark}}

% used when adding fill-in-the-blanks for students
\newcommand{\blank}[1]{\underline{\hspace{#1}}}

% indents annoy me, and so does repeatedly typing \noindent
\newcommand{\p}{\noindent}

\begin{document}

\fancyhead[L]{\course}
\fancyhead[C]{\includegraphics[width=5cm, trim= 0 0.4cm 0 0]{TRU_logo}}
\fancyhead[R]{\term}
\renewcommand{\headrulewidth}{0.4pt}

\setulcolor{red}

\setcounter{section}{0}
\section{Arithmetic in \texorpdfstring{$\INTEGER, \RATIONAL, \REAL$}{Z, Q, R} and \texorpdfstring{$\COMPLEX$}{C}}
\setcounter{subsection}{0}
\subsection{The Division Algorithm}

\p The set of \DEF{natural numbers} is written as
\begin{center}
$\NATURAL = \SET{1,2,3,\dots}$
\end{center}
and the set of \DEF{integers} is written as
\begin{center}
\item $\INTEGER = \SET{0, \pm 1, \pm 2,\pm 3,\dots}$
\end{center}
\vsp

\p New sets may be written concisely using \DEF{set-builder notation}.
\vsp

\begin{examples*}~
\begin{enumerate}[(a)]
\item $\Z^{+} := \SET{n\in \Z: n > 0}$ is the set of all positive integers.
\item The set of all multiples of $5$ may be written as $\SET{5k:k\in\Z}$.
\end{enumerate}
\end{examples*}
\vsp

\begin{theorem*}[The Division Algorithm] Let $a,b\in\Z^{+}$.  There exists unique $q,r\in\INTEGER$ such that
\begin{center}
$a = bq + r$ and $0\leq r < b$.
\end{center}
\end{theorem*}
%

\begin{example*}~

\vspace{2cm}

\end{example*}

\p The division algorithm essentially tells us that performing division between two integers will yield a unique quotient and remainder.
\vsp

\subsection{Divisibility}

\begin{definition*} Let $a,b\in\INTEGER$ and $b\neq 0$.  We say that $b$ \DEF{divides} $a$ (or that $b$ is a \DEF{divisor} of $a$) if there exists $k\in\INTEGER$ such that $a = kb$.
\end{definition*}
\vsp

\notation If $b$ divides $a$, we write $b | a$.  Otherwise we write $b\ndiv a$.
\vsp

\begin{examples*}~
\vspace{2cm}

\end{examples*}
\newpage

\begin{example*}
Prove that if $b|a$ and $b|c$, then $b|a+c$.
\end{example*}
%
\begin{solution}~
\vspace{3cm}

\end{solution}
%
\begin{definition*} If $a,b\in\Z$ are not both zero, then the \DEF{greatest common divisor} of $a$ and $b$ is the largest \nolinebreak of all integers which divide both $a$ and $b$.  In other words, $d\in\INTEGER$ is the greatest common divisor of $a$ and \nolinebreak $b$ \nolinebreak if:
\begin{enumerate}[(1)]
\item $d$ is a common divisor of both $a$ and $b$.
\item If $d'$ is also a common divisor of both $a$ and $b$, then $d'\leq d$.
\end{enumerate}
\end{definition*}
\vsp

\notation The greatest common divisor of $a,b\in\INTEGER$ is denoted by $\GCD{a,b}$.
\vsp

\begin{example*}~
\vspace{1cm}

\end{example*}
%
\begin{theorem*}[Euclidean Algorithm]
Let $a,b\in\INTEGER$ be positive.  If $b|a$ then $\GCD{a,b} = b$.  If $b\ndiv a$, then apply the Division Algorithm repeatedly as follows.

\begin{itemize}[\ ]
\item $a = q_{0}b + r_{0}$ for some $q_{0},r_{0}\in\INTEGER$ with $0 \leq r_{0} < b$.
\item $b = q_{1}r_{0} + r_{1}$ for some $q_{1},r_{1}\in\INTEGER$ with $0 \leq r_{1} < r_{0}$.
\item $r_{0} = q_{2}r_{1} + r_{2}$ for some $q_{2},r_{2}\in\INTEGER$ with $0 \leq r_{2} < r_{1}$.
\item $r_{1} = q_{3}r_{2} + r_{3}$ for some $q_{3},r_{3}\in\INTEGER$ with $0 \leq r_{3} < r_{2}$.
\item \phantom{$r_{0} = q_{2}r_{1} + r_{2}$ for some}$\vdots$
\item $r_{t-2} = q_{t}r_{t-1} + r_{t}$ for some $q_{t},r_{t}\in\INTEGER$ with $0 \leq r_{t} < r_{t-1}$.
\item $r_{t-1} = q_{t+1}r_{t} + 0$ for some $q_{t+1}\in\INTEGER$,
\end{itemize}
with the process terminating on the $(t+2)$-nd iteration, where $r_{t}$ divides $r_{t-1}$.  In this case, the final non-zero remainder is the greatest common divisor of $a$ and $b$.  That is, $\GCD{a,b} = r_{t}$.
\end{theorem*}
%
\begin{corollary*}[B\'{e}zout's Identity]\label{bezout}
Let $a,b\in\INTEGER$ be positive.  There exists $s,t\in\INTEGER$ such that $\GCD{a,b} = sa + tb$.
\end{corollary*}
%
\begin{example*}~
\begin{enumerate}[(a)]
\item Use the Euclidean algorithm to show that $\GCD{84,60} = 12$.
\item Find $s,t\in\Z$ such that $12 = 84s + 60t$.
\end{enumerate}
\end{example*}
%
\begin{solution}~
\vspace{10cm}

\end{solution}
%
\begin{definition*}
Let $a,b\in\INTEGER$.  The \DEF{least common multiple} of $a$ and $b$ is the smallest positive integer $m$ such that $a|m$ and $b|m$.
\end{definition*}
%
\notation The least common multiple of $a,b\in\Z$ is denoted as $\LCM{a,b}$.
\vsp

\begin{example*}~
\vspace{2cm}

\end{example*}

\subsection{Primes and Unique Factorization}

\begin{definition}
An integer $p > 1$ is \DEF{prime} if it has exactly two positive divisors.  Otherwise $p$ is said to be \DEF{composite}.
\end{definition}

\begin{example*}~
\vspace{1cm}

\end{example*}

\begin{theorem*}[Euclid's Lemma]\label{euclidslemma}
Suppose $b,c\in\Z$.  If $p$ is a prime such that $p|bc$, then $p|b$ or $p|c$.
\end{theorem*}
\vsp

\begin{remark}
In fact, this property entirely characterizes prime numbers!
\end{remark}
\newpage

\begin{theorem}[Fundamental Theorem of Arithmetic]
Every integer $n > 1$ is either prime or is a product of primes.  Moreover, its factorization as a product of primes is unique up to the order in which the primes are multiplied.
\end{theorem}
%
\begin{examples*}~
\vspace{5cm}

\end{examples*}


\end{document}





