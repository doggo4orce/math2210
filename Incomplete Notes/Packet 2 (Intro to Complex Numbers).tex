\documentclass[11pt,fleqn,dvipsnames,usenames]{article}

% to keep this file less overwhelming
% packages to include

\usepackage[dvipsnames, table]{xcolor}

\usepackage{
  amsthm,
  amsmath,
  amssymb, 
  arydshln, % for hyphenated lines in block matrices
  fancyhdr, % needed for header at top of each page
  graphicx, % to include pictures
  mathtools, % for a longer arrow
  multicol, % displaying enumerates and itemizes into multiple columns
  multirow, % for tables
  multido, % for TOC
  pgfplots, % for axis environment within tikz pictures
  systeme,
  tikz,
}

\usepackage[utf8]{inputenc}
\usepackage{color,soul}

\usepackage[inline, shortlabels]{enumitem}
\usepackage[hidelinks]{hyperref}


% global constants
\newcommand{\term}{Winter 2024}

% mathbb aliases
\newcommand{\COMPLEX}{\mathbb{C}}
\newcommand{\REAL}{\mathbb{R}}
\newcommand{\NATURAL}{\mathbb{N}}
\newcommand{\INTEGER}{\mathbb{Z}}

% for financial stuff
\newcommand{\dollar}{\mathrm{\$}}

% nicer looking trig functions
\newcommand{\SIN}[1]{\sin\left(#1\right)}
\newcommand{\COS}[1]{\cos\left(#1\right)}
\newcommand{\TAN}[1]{\tan\left(#1\right)}
\newcommand{\CSC}[1]{\csc\left(#1\right)}
\newcommand{\SEC}[1]{\sec\left(#1\right)}
\newcommand{\COT}[1]{\cot\left(#1\right)}

% automatically resize set brackets
\newcommand{\SET}[1]{\left\{#1\right\}}

% sums and products
\newcommand{\SUM}{\displaystyle\sum\limits}
\newcommand{\PROD}{\displaystyle\prod\limits}
\newcommand{\of}{\circ}
\newcommand{\restrict}[1]{\raisebox{-.5ex}{$|$}_{#1}}

% set intersection and union
\newcommand{\CAP}{\displaystyle\bigcap\limits}
\newcommand{\CUP}{\displaystyle\bigcup\limits}

% max and min
\newcommand{\MAX}[1]{\ensuremath{\max\left(#1\right)}}
\newcommand{\MIN}[1]{\ensuremath{\min\left(#1\right)}}

% for writing logic within mathematics environment
\newcommand{\FORALL}{\ensuremath{\text{ for all }}}
\newcommand{\FORSOME}{\ensuremath{\text{ for some }}}

% matrix notation
\newcommand{\MATRIX}[2]{\ensuremath{\left[\begin{array}{#1}#2\end{array}\right]}}
\newcommand{\COLUMN}[1]{\ensuremath{\left[\begin{array}{r}#1\end{array}\right]}}

% vector notation
%\newcommand{\vv}{\overset{\rightharpoonup}}
\newcommand{\vv}[1]{{\bf #1}}
\newcommand{\arr}{\overrightarrow}

% dot product
\newcommand{\dotp}{{\scriptstyle\bullet}}

% Text macros
\newcommand{\KER}[1]{\ensuremath{\text{ker}\left(#1\right)}}
\newcommand{\IMG}[1]{\ensuremath{\text{im}\left(#1\right)}}
\newcommand{\CHAR}[1]{\ensuremath{\text{char}\left(#1\right)}}
\newcommand{\BIGO}[1]{\ensuremath{\mathcal{O}\left(#1\right)}}
\newcommand{\TR}[1]{\ensuremath{\text{tr}\left(#1\right)}}

% abbreviations
\newcommand{\ds}{\displaystyle}
\newcommand{\md}{\mdseries}
\newcommand{\vsp}{\vspace{0.5cm}}
\newcommand{\smsp}{\vspace{0.25cm}}
\newcommand{\hsp}{\hspace{0.25cm}}

% new operators
\DeclareMathOperator\SPAN{Span}
\newcommand{\SPANOF}[1]{\ensuremath{\SPAN\left\{#1\right\}}}
\DeclareMathOperator\PROJ{proj}
\DeclareMathOperator\PERP{perp}

% quick abbreviations to avoid using latex environments
\newcommand{\answer}{\noindent \textbf{Answer:} }
\newcommand{\answers}{\noindent \textbf{Answers:} }
\newcommand{\application}{\noindent \textbf{Application:} }
\newcommand{\caution}{\noindent \textbf{Caution:} }
\newcommand{\conclusion}{\noindent \textbf{Conclusion:} }
\newcommand{\consequence}{\noindent \textbf{Consequence:} }
\newcommand{\defn}{\noindent \textbf{Definition:} }
\newcommand{\details}{\noindent \textbf{Details:} }
\newcommand{\example}{\noindent \textbf{Example:} }
\newcommand{\examples}{\noindent \textbf{Examples:} }
\newcommand{\exception}{\noindent \textbf{Exception:} }
\newcommand{\exercise}{\noindent \textbf{Exercise:} }
\newcommand{\exercises}{\noindent \textbf{Exercises:} }
\newcommand{\fact}{\noindent \textbf{Fact:} }
\newcommand{\facts}{\noindent \textbf{Facts:} }
\newcommand{\formula}{\noindent \textbf{Formula:} }
\newcommand{\goal}{\noindent \textbf{Goal:} }
\newcommand{\goals}{\noindent \textbf{Goals:} }
\newcommand{\hint}{\noindent \textbf{Hint:} }
\newcommand{\idea}{\noindent \textbf{Idea:} }
\newcommand{\illustration}{\noindent \textbf{Illustration:} }
\newcommand{\important}{\noindent \textbf{Important:} }
\newcommand{\midea}{\noindent \textbf{Main Idea:} }
\newcommand{\motivation}{\noindent \textbf{Motivation:} }
\newcommand{\nthm}[1]{\noindent \textbf{Theorem} (\textit{#1}):}
\newcommand{\notation}{\noindent \textbf{Notation:} }
\newcommand{\note}{\noindent \textbf{Note:} }
\newcommand{\notes}{\noindent \textbf{Notes:} }
\newcommand{\observation}{\noindent \textbf{Observation:} }
\newcommand{\observations}{\noindent \textbf{Observations:} }
\newcommand{\pict}{\noindent \textbf{Picture:} }
\newcommand{\plan}{\noindent \textbf{Plan:} }
\newcommand{\prf}{\noindent \textbf{Proof:} }
\newcommand{\problem}{\noindent \textbf{Problem:} }
\newcommand{\properties}{\noindent \textbf{Properties:} }
\newcommand{\question}{\noindent \textbf{Question:} }
\newcommand{\questions}{\noindent \textbf{Questions:} }
\newcommand{\recall}{\noindent \textbf{Recall:} }
\newcommand{\reason}{\noindent \textbf{Reason:} }
\newcommand{\remark}{\noindent \textbf{Remark:} }
\newcommand{\remarks}{\noindent \textbf{Remarks:} }
\newcommand{\reminder}{\noindent \textbf{Reminder:} }
\newcommand{\solution}{\noindent \textbf{Solution:} }
\newcommand{\nsolution}[1]{\noindent \textbf{Solution #1:} }
\newcommand{\strategy}{\noindent \textbf{Strategy:} }
\newcommand{\summary}{\noindent \textbf{Summary:} }
\newcommand{\terminology}{\noindent \textbf{Terminology:} }
\newcommand{\thm}{\noindent \textbf{Theorem:} }
\newcommand{\work}{\noindent \textbf{Work:} }


\usepackage[version=4]{mhchem}

% Where to look for pngs and jpegs
\graphicspath{{Images//}}

\usepackage[includehead, includefoot, left= 2cm, top =1.5cm, bottom = 1.5cm, textwidth=17.5cm]{geometry}

\usepackage{pifont, amsmath}


\pagestyle{fancy}
\fancyhf{}
\renewcommand{\headrulewidth}{1pt}
%\fancyhead[R]{\bfseries\sffamily\thepage}
%\fancyfoot[C]{\bfseries\sffamily\thepage}
\fancyhead[L]{\nouppercase{\bfseries\sffamily\leftmark}}

% used when adding fill-in-the-blanks for students
\newcommand{\blank}[1]{\underline{\hspace{#1}}}

% indents annoy me, and so does repeatedly typing \noindent
\newcommand{\p}{\noindent}

\begin{document}

\fancyhead[L]{\course}
\fancyhead[C]{\includegraphics[width=5cm, trim= 0 0.4cm 0 0]{TRU_logo}}
\fancyhead[R]{\term}
\renewcommand{\headrulewidth}{0.4pt}

\setulcolor{red}

\setcounter{section}{0}
\section{Arithmetic in \texorpdfstring{$\INTEGER, \RATIONAL$, and $\REAL$}{Z, Q, and R}}
\setcounter{subsection}{3}
\subsection{Rational Numbers}

\recall A real number $r\in\REAL$ is said to be \DEF{rational} if it may be written in the form $r = m/n$, where $m,n\in\INTEGER$ with $\GCD{m, n} = 1$.  If $r$ is not rational it is said to be \DEF{irrational}.
\vsp

\notation The set of all rational numbers is denoted by $\RATIONAL = \SET{m/n:m,n\in\INTEGER\text{ with } n\neq 0}$.

\begin{example}
Show that $\sqrt{2} \notin \RATIONAL$.
\end{example}
%
\begin{solution}
\vspace{6cm}

\end{solution}

\subsection{Arithmetic in \texorpdfstring{$\C$}{C}}

\recall $\REAL^{2}$ is the set of all ordered pairs $(a,b)$, where $a,b\in\REAL$.

\begin{definition}
A \DEF{complex number} is
\end{definition}
\vsp

\notation \blank{1cm} is the set of all complex numbers.
\vsp

\begin{remark}
We may view $\REAL$ as a subset of $\COMPLEX$ by identifying a real number $x\in\REAL$ with the complex number $(x,0)\in\COMPLEX$.  For example,
\begin{itemize}
\item $3\in\REAL$ is identified with $(3,0)\in\COMPLEX$, i.e. $3 = (3,0)$.
\item $-4\in\REAL$ is identified with $(-4,0)\in\COMPLEX$, i.e. $-4 = (-4,0)$.
\item $0\in\REAL$ is identified with $(0,0)\in\COMPLEX$, i.e. $0 = (0,0)$.
\end{itemize}
\end{remark}
\newpage


\begin{definition}
If $z_{1} = (a_{1},b_{1}), z_{2} = (a_{2},b_{2})\in\COMPLEX$, we define their \DEF{sum} and \DEF{difference} using the rules:
\begin{itemize}
\item $z_{1} + z_{2} =$
\vsmsp

\item $z_{1} - z_{2} =$
\end{itemize}
\end{definition}

\begin{example}
If $z_{1} = (3,4)$ and $z_{2} = (1,-1)$, then
\end{example}
\vspace{3cm}

\begin{definition}\label{complexmultiplication}
The \DEF{product} of two complex numbers $z_{1} = (a_{1},b_{1}), z_{2} = (a_{2},b_{2})\in\COMPLEX$, is defined using the rule
\vspace{2cm}
\end{definition}

\begin{example}\label{examplecomplexmultiplication}
If $z_{1} = (1,2)$ and $z_{2} = (-1,3)$, then
\end{example}
\vspace{3cm}

\notation For any $z\in\COMPLEX$, $-z = (-a,-b)$ is the \DEF{negation} of $z$.
\vsp

\properties For any $z,w\in\COMPLEX$,
\begin{multicols}{2}
\begin{enumerate}[(a)]
\item $1z = z$
\item $(-1)z = -z$
\item $z - z = 0$
\item $z + (-w) = z - w$
\item $z + w = w + z$
\item $z + (w + \lambda) = (z + w) + \lambda$
\item $wz = zw$
\item $w(z\lambda) = (wz)\lambda$
\end{enumerate}
\end{multicols}
\newpage

\notation As usual, we use exponents to express repeated multiplication.  For any $z\in\COMPLEX$ and any integer $n\geq 0$,
\begin{center}
$z^{n} = \begin{cases} 1 & \text{ if }n = 0\\z\cdot z^{n-1} &\text{ if }n \geq 1\end{cases}$
\end{center}

\note Negative powers will also be defined in the usual way, but not until division is defined in $\COMPLEX$!

\begin{example}
If $z = (1,1)$, then
\vsp

$z^3 =$
\vsp

\end{example}
%
\observation Recall that there exists no $z\in\REAL$ such that $z^2 = -1$.  But note that in $\COMPLEX$, we have
\vspace{2cm}

\terminology \blank{3cm} is called the \DEF{imaginary unit}.
\vsp

\notation For any real numbers $a,b\in\REAL$, we may write the complex number $z = (a,b)$ as
\vspace{2cm}

\p With this convention, Definition \ref{complexmultiplication} is equivalent to using the distributive property, treating $i$ as a variable, and collecting like terms.  For example,
\vsp

$(1 + 2i)(-1 + 3i) =$
\vsp

\begin{definition}
If $z = a + bi$ is a complex number,
\vfill

\end{definition}

\note A complex number is characterized entirely by its real and imaginary parts.  That is, $z_{1},z_{2}\in\COMPLEX$ are equal if and only if
\begin{center}
$\RE{z_{1}} = \RE{z_{2}}$ and $\IM{z_{1}} = \IM{z_{2}}$.
\end{center}
\newpage

%
\begin{definition}
Let $z_{1} = a+bi,z_{2} = c+di\in\COMPLEX$ with $z_{2}\neq 0$.  The \DEF{quotient} of $z_{1}$ and $z_{2}$ is defined to be the complex number $z_{1}/z_{2}$ with real and imaginary parts given by
\begin{center}
$\RE{z_{1}/z_{2}} = \dfrac{ac + bd}{c^2 + d^2}$ and $\IM{z_{1}/z_{2}} = \dfrac{ad - bc}{c^2 + d^2}$.
\end{center}
\end{definition}
\vsmsp

\GRAYLINE
%

\begin{definition}
Let $z = a + bi\in\COMPLEX$.  The \DEF{complex conjugate} of $z$, denoted $\CC{z}$, is defined by $\CC{z} = a - bi$.
\end{definition}
\vsp

\p Note that for any $z = a + bi\in\COMPLEX$, we have
\vspace{2cm}

\p In particular, $z\CC{z}\in\REAL$.  This is the idea behind the definition of complex division, which is really an analogue of the technique of {multiplication by the conjugate}:
\vfill

\begin{definition}
The \DEF{complex modulus} or \DEF{absolute value} of $z = a+bi\in\COMPLEX$ is given by
\end{definition}
\newpage

\p With division available, negative powers may be defined using the familiar rule
\vspace{2cm}

\p whenever $n$ is an integer with $n\geq 1$.
\vsp

\properties For any $z\in\COMPLEX$ and $m,n\in\INTEGER$,
\begin{enumerate}[(a)]
\item $z^{m}z^{n} = z^{m+n}$
\item $z^{m}/z^{n} = z^{m-n}$
\item $\ds{\left(z_{1}z_{2}\right)^{n} = z_{1}^{n}\cdot z_{2}^{n}}$
\item $\ds{\left(\frac{z_{1}}{z_{2}}\right)^{n} = \frac{z_{1}^{n}}{z_{2}^{n}}}$
\item $\left(z^{m}\right)^{n} = z^{mn}$
\end{enumerate}
\vsp

\p Thanks to the existence of $i\in\C$, we may extend the square root function to the entire real line!
\vsp

\begin{definition}
If $x\in\R$ with $x < 0$, then the \DEF{square root} of $x$ is
\end{definition}
\vsp

\begin{examples}~
\begin{enumerate}[(a)]
\item Compute $\sqrt{-4}$.
\item Find all solutions to the quadratic equation $x^2 + x + 1 = 0$.
\end{enumerate}
\end{examples}
%
\begin{solution}
\newpage

\end{solution}
\end{document}





