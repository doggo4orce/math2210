\documentclass[11pt,fleqn,dvipsnames,usenames]{article}

% to keep this file less overwhelming
% packages to include

\usepackage[dvipsnames, table]{xcolor}

\usepackage{
  amsthm,
  amsmath,
  amssymb, 
  arydshln, % for hyphenated lines in block matrices
  fancyhdr, % needed for header at top of each page
  graphicx, % to include pictures
  mathtools, % for a longer arrow
  multicol, % displaying enumerates and itemizes into multiple columns
  multirow, % for tables
  multido, % for TOC
  pgfplots, % for axis environment within tikz pictures
  systeme,
  tikz,
}

\usepackage[utf8]{inputenc}
\usepackage{color,soul}

\usepackage[inline, shortlabels]{enumitem}
\usepackage[hidelinks]{hyperref}


% global constants
\newcommand{\term}{Winter 2024}

% mathbb aliases
\newcommand{\COMPLEX}{\mathbb{C}}
\newcommand{\REAL}{\mathbb{R}}
\newcommand{\NATURAL}{\mathbb{N}}
\newcommand{\INTEGER}{\mathbb{Z}}

% for financial stuff
\newcommand{\dollar}{\mathrm{\$}}

% nicer looking trig functions
\newcommand{\SIN}[1]{\sin\left(#1\right)}
\newcommand{\COS}[1]{\cos\left(#1\right)}
\newcommand{\TAN}[1]{\tan\left(#1\right)}
\newcommand{\CSC}[1]{\csc\left(#1\right)}
\newcommand{\SEC}[1]{\sec\left(#1\right)}
\newcommand{\COT}[1]{\cot\left(#1\right)}

% automatically resize set brackets
\newcommand{\SET}[1]{\left\{#1\right\}}

% sums and products
\newcommand{\SUM}{\displaystyle\sum\limits}
\newcommand{\PROD}{\displaystyle\prod\limits}
\newcommand{\of}{\circ}
\newcommand{\restrict}[1]{\raisebox{-.5ex}{$|$}_{#1}}

% set intersection and union
\newcommand{\CAP}{\displaystyle\bigcap\limits}
\newcommand{\CUP}{\displaystyle\bigcup\limits}

% max and min
\newcommand{\MAX}[1]{\ensuremath{\max\left(#1\right)}}
\newcommand{\MIN}[1]{\ensuremath{\min\left(#1\right)}}

% for writing logic within mathematics environment
\newcommand{\FORALL}{\ensuremath{\text{ for all }}}
\newcommand{\FORSOME}{\ensuremath{\text{ for some }}}

% matrix notation
\newcommand{\MATRIX}[2]{\ensuremath{\left[\begin{array}{#1}#2\end{array}\right]}}
\newcommand{\COLUMN}[1]{\ensuremath{\left[\begin{array}{r}#1\end{array}\right]}}

% vector notation
%\newcommand{\vv}{\overset{\rightharpoonup}}
\newcommand{\vv}[1]{{\bf #1}}
\newcommand{\arr}{\overrightarrow}

% dot product
\newcommand{\dotp}{{\scriptstyle\bullet}}

% Text macros
\newcommand{\KER}[1]{\ensuremath{\text{ker}\left(#1\right)}}
\newcommand{\IMG}[1]{\ensuremath{\text{im}\left(#1\right)}}
\newcommand{\CHAR}[1]{\ensuremath{\text{char}\left(#1\right)}}
\newcommand{\BIGO}[1]{\ensuremath{\mathcal{O}\left(#1\right)}}
\newcommand{\TR}[1]{\ensuremath{\text{tr}\left(#1\right)}}

% abbreviations
\newcommand{\ds}{\displaystyle}
\newcommand{\md}{\mdseries}
\newcommand{\vsp}{\vspace{0.5cm}}
\newcommand{\smsp}{\vspace{0.25cm}}
\newcommand{\hsp}{\hspace{0.25cm}}

% new operators
\DeclareMathOperator\SPAN{Span}
\newcommand{\SPANOF}[1]{\ensuremath{\SPAN\left\{#1\right\}}}
\DeclareMathOperator\PROJ{proj}
\DeclareMathOperator\PERP{perp}

% quick abbreviations to avoid using latex environments
\newcommand{\answer}{\noindent \textbf{Answer:} }
\newcommand{\answers}{\noindent \textbf{Answers:} }
\newcommand{\application}{\noindent \textbf{Application:} }
\newcommand{\caution}{\noindent \textbf{Caution:} }
\newcommand{\conclusion}{\noindent \textbf{Conclusion:} }
\newcommand{\consequence}{\noindent \textbf{Consequence:} }
\newcommand{\defn}{\noindent \textbf{Definition:} }
\newcommand{\details}{\noindent \textbf{Details:} }
\newcommand{\example}{\noindent \textbf{Example:} }
\newcommand{\examples}{\noindent \textbf{Examples:} }
\newcommand{\exception}{\noindent \textbf{Exception:} }
\newcommand{\exercise}{\noindent \textbf{Exercise:} }
\newcommand{\exercises}{\noindent \textbf{Exercises:} }
\newcommand{\fact}{\noindent \textbf{Fact:} }
\newcommand{\facts}{\noindent \textbf{Facts:} }
\newcommand{\formula}{\noindent \textbf{Formula:} }
\newcommand{\goal}{\noindent \textbf{Goal:} }
\newcommand{\goals}{\noindent \textbf{Goals:} }
\newcommand{\hint}{\noindent \textbf{Hint:} }
\newcommand{\idea}{\noindent \textbf{Idea:} }
\newcommand{\illustration}{\noindent \textbf{Illustration:} }
\newcommand{\important}{\noindent \textbf{Important:} }
\newcommand{\midea}{\noindent \textbf{Main Idea:} }
\newcommand{\motivation}{\noindent \textbf{Motivation:} }
\newcommand{\nthm}[1]{\noindent \textbf{Theorem} (\textit{#1}):}
\newcommand{\notation}{\noindent \textbf{Notation:} }
\newcommand{\note}{\noindent \textbf{Note:} }
\newcommand{\notes}{\noindent \textbf{Notes:} }
\newcommand{\observation}{\noindent \textbf{Observation:} }
\newcommand{\observations}{\noindent \textbf{Observations:} }
\newcommand{\pict}{\noindent \textbf{Picture:} }
\newcommand{\plan}{\noindent \textbf{Plan:} }
\newcommand{\prf}{\noindent \textbf{Proof:} }
\newcommand{\problem}{\noindent \textbf{Problem:} }
\newcommand{\properties}{\noindent \textbf{Properties:} }
\newcommand{\question}{\noindent \textbf{Question:} }
\newcommand{\questions}{\noindent \textbf{Questions:} }
\newcommand{\recall}{\noindent \textbf{Recall:} }
\newcommand{\reason}{\noindent \textbf{Reason:} }
\newcommand{\remark}{\noindent \textbf{Remark:} }
\newcommand{\remarks}{\noindent \textbf{Remarks:} }
\newcommand{\reminder}{\noindent \textbf{Reminder:} }
\newcommand{\solution}{\noindent \textbf{Solution:} }
\newcommand{\nsolution}[1]{\noindent \textbf{Solution #1:} }
\newcommand{\strategy}{\noindent \textbf{Strategy:} }
\newcommand{\summary}{\noindent \textbf{Summary:} }
\newcommand{\terminology}{\noindent \textbf{Terminology:} }
\newcommand{\thm}{\noindent \textbf{Theorem:} }
\newcommand{\work}{\noindent \textbf{Work:} }


\usepackage[version=4]{mhchem}

% Where to look for pngs and jpegs
\graphicspath{{Images//}}

\usepackage[includehead, includefoot, left= 2cm, top =1.5cm, bottom = 1.5cm, textwidth=17.5cm]{geometry}

\usepackage{pifont, amsmath}


\pagestyle{fancy}
\fancyhf{}
\renewcommand{\headrulewidth}{1pt}
%\fancyhead[R]{\bfseries\sffamily\thepage}
%\fancyfoot[C]{\bfseries\sffamily\thepage}
\fancyhead[L]{\nouppercase{\bfseries\sffamily\leftmark}}

% used when adding fill-in-the-blanks for students
\newcommand{\blank}[1]{\underline{\hspace{#1}}}

% indents annoy me, and so does repeatedly typing \noindent
\newcommand{\p}{\noindent}

\begin{document}

\fancyhead[L]{\course}
\fancyhead[C]{\includegraphics[width=5cm, trim= 0 0.4cm 0 0]{TRU_logo}}
\fancyhead[R]{\term}
\renewcommand{\headrulewidth}{0.4pt}

\setulcolor{red}

\setcounter{section}{1}
\section{Complex Numbers}
\setcounter{subsection}{1}
\subsection{Euler's Formula}

\remark Intuition about real powers demands that any good definition satisfy
\vspace{2cm}


\p But this still leaves the meaning of $e^{bi}$ undetermined.  To fix this, we recall the Maclaurin series expansion of the following real-valued functions:
\begin{center}
$e^{x} = \SUM_{n=0}^{\infty}\frac{x^{n}}{n!}$\hspace{2cm}
$\sin(x) = \SUM_{n=0}^{\infty}(-1)^{n}\frac{x^{2n+1}}{(2n+1)!}$\hspace{2cm}
$\cos(x) = \SUM_{n=0}^{\infty}(-1)^{n}\frac{x^{2n}}{(2n)!}$
\end{center}
\vsp

\defn
For $z\in\C$, we define
\vspace{2cm}

\p From this, a beautiful consequence follows!
\vsp

\nthm{Euler's Formula}
For any $\theta\in\REAL$, $e^{i\theta} = \cos(\theta) + i\sin(\theta)$.
\vspace{2cm}

\prf

\begin{align*}
e^{\theta i} = \SUM_{n=0}^{\infty}\frac{(\theta i)^{n}}{n!} = \SUM_{n=0}^{\infty}\frac{\theta^{n}i^{n}}{n!} = \SUM_{n=0}^{\infty}\frac{\theta^{2n}i^{2n}}{(2n)!} + \SUM_{n=0}^{\infty}\frac{\theta^{2n+1}i^{2n+1}}{(2n+1)!} &= \SUM_{n=0}^{\infty}\frac{\theta^{2n}(-1)^n}{(2n)!} + \SUM_{n=0}^{\infty}(-1)^{n}i\frac{\theta^{2n+1}}{(2n+1)!}\\
&= \SUM_{n=0}^{\infty}(-1)^{n}\frac{\theta^{2n}}{(2n)!} + \SUM_{n=0}^{\infty}(-1)^{n}i\frac{\theta^{2n+1}}{(2n+1)!}\\
&= \SUM_{n=0}^{\infty}(-1)^{n}\frac{\theta^{2n}}{(2n)!} + i\left[\SUM_{n=0}^{\infty}(-1)^{n}\frac{\theta^{2n+1}}{(2n+1)!}\right]\\
&= \cos(\theta) + i\sin(\theta)
\end{align*}
\vsp

\example If $z = 2 + \big(\pi/2\big)i$, then
\vspace{3cm}

\properties For any $z_{1},z_{2}\in\COMPLEX$,
\begin{multicols}{2}
\begin{enumerate}[(1)]
\item $e^{z_{1}}e^{z_{2}} = e^{z_{1}z_{2}}$
\item $\ds{\frac{e^{z_{1}}}{e^{z_{2}}} = e^{z_{1} - z_{2}}}$
\end{enumerate}
\end{multicols}

\subsection{The Complex Plane}

\p As elements of $\REAL^{2}$, complex numbers may be identified with points in the Cartesian plane.  For example, if $z_{1} = 2 + 4i$, $z_{2} = 3$, and $z_{3} = -1 -3i$, then they may be sketched:
\vfill

\p A complex number $z = a + bi$ may also be geometrically represented as an arrow pointing from the origin to the point $(a,b)$ in $\REAL^{2}$.  Any such arrow has length given by the complex modulus of $z$, and forms an angle $\theta$ of inclination, measured counter-clockwise from the positive half of the real axis.
\vfill

\defn An ordered pair $(r,\theta)$ is a polar coordinate representation of $z$ if $r = |z|$ and $\theta$ is the angle of inclination of $z$, measured counter-clockwise from the positive half of the real axis.  In this case $\theta$ is called an \DEF{argument} of $z$.
\newpage

\begin{examples*}~
\begin{enumerate}[(a)]
\item $(2,0)$ is a polar coordinate representation of $2$.
\item $(1,\pi/2)$ is a polar coordinate representation of $i$.
\item $(3,-\pi/2)$ is a polar coordinate representation of $-3i$.
\end{enumerate}
\end{examples*}
\vfill

%
\begin{remarks}~
\begin{enumerate}[(1)]
\item Since arguments of complex numbers are not unique, neither are polar coordinate representations!  For example both $\pi$ and $3\pi$ are arguments of the real number $-1$, which therefore has both $(1,\pi)$ and $(1,3\pi)$ as polar coordinate representations.
\item If $(r,\theta)$ is a polar coordinate representation of a complex number $z$, the Pythagorean Theorem forces real and imaginary parts of $z$ to be given by
\vspace{2cm}

\p It follows that $z = r\cos(\theta) + ir\sin(\theta) = r(\cos(\theta) + i\sin(\theta)) = re^{i\theta}$, which is referred to as the \DEF{polar form} of $z$.
\end{enumerate}
\end{remarks}
\vsp

\observation If $z_{1} = r_{1}e^{i\theta_{1}}$ and $z_{2} = r_{2}e^{i\theta_{2}}$ are written in polar form, then
\vspace{2cm}

and
\vspace{2cm}
\vsp

\p Consequently, to multiply complex numbers in polar form, we \emph{multiply their lengths and add their arguments}.  To divide them, we \emph{divide their lengths, and subtract their arguments}.
\newpage

\p This is helpful to make use of when computing powers.  Indeed if $z = re^{i\theta}$ is in polar form and $n\geq 0$, then
\begin{center}
$z^{n} = \left(re^{i\theta}\right)^{n} = r^{n}\left(e^{i\theta}\right)^{n} = r^{n}e^{n(i\theta)} = r^{n}e^{i(n\theta)}$.
\end{center}
\vsp

\p In the special case when $r = 1$,
\vspace{3cm}
~

\nthm{DeMoivre's Formula}
\vspace{2cm}

\begin{examples}~
\begin{enumerate}[(a)]
\item Let $z_{1} = 1 + i$ and $z_{2}  = \sqrt{3} - i$.  Find $z_{1}z_{2}$ and $z_{1}/z_{2}$ in polar form.
\item Find $(1 + i)^{16}$.
\end{enumerate}
\end{examples}
\newpage


\end{document}





