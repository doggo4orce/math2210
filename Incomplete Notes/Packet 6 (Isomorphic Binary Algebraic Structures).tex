\documentclass[11pt,fleqn,dvipsnames,usenames]{article}

% to keep this file less overwhelming
% packages to include

\usepackage[dvipsnames, table]{xcolor}

\usepackage{
  amsthm,
  amsmath,
  amssymb, 
  arydshln, % for hyphenated lines in block matrices
  fancyhdr, % needed for header at top of each page
  graphicx, % to include pictures
  mathtools, % for a longer arrow
  multicol, % displaying enumerates and itemizes into multiple columns
  multirow, % for tables
  multido, % for TOC
  pgfplots, % for axis environment within tikz pictures
  systeme,
  tikz,
}

\usepackage[utf8]{inputenc}
\usepackage{color,soul}

\usepackage[inline, shortlabels]{enumitem}
\usepackage[hidelinks]{hyperref}


% global constants
\newcommand{\term}{Winter 2024}

% mathbb aliases
\newcommand{\COMPLEX}{\mathbb{C}}
\newcommand{\REAL}{\mathbb{R}}
\newcommand{\NATURAL}{\mathbb{N}}
\newcommand{\INTEGER}{\mathbb{Z}}

% for financial stuff
\newcommand{\dollar}{\mathrm{\$}}

% nicer looking trig functions
\newcommand{\SIN}[1]{\sin\left(#1\right)}
\newcommand{\COS}[1]{\cos\left(#1\right)}
\newcommand{\TAN}[1]{\tan\left(#1\right)}
\newcommand{\CSC}[1]{\csc\left(#1\right)}
\newcommand{\SEC}[1]{\sec\left(#1\right)}
\newcommand{\COT}[1]{\cot\left(#1\right)}

% automatically resize set brackets
\newcommand{\SET}[1]{\left\{#1\right\}}

% sums and products
\newcommand{\SUM}{\displaystyle\sum\limits}
\newcommand{\PROD}{\displaystyle\prod\limits}
\newcommand{\of}{\circ}
\newcommand{\restrict}[1]{\raisebox{-.5ex}{$|$}_{#1}}

% set intersection and union
\newcommand{\CAP}{\displaystyle\bigcap\limits}
\newcommand{\CUP}{\displaystyle\bigcup\limits}

% max and min
\newcommand{\MAX}[1]{\ensuremath{\max\left(#1\right)}}
\newcommand{\MIN}[1]{\ensuremath{\min\left(#1\right)}}

% for writing logic within mathematics environment
\newcommand{\FORALL}{\ensuremath{\text{ for all }}}
\newcommand{\FORSOME}{\ensuremath{\text{ for some }}}

% matrix notation
\newcommand{\MATRIX}[2]{\ensuremath{\left[\begin{array}{#1}#2\end{array}\right]}}
\newcommand{\COLUMN}[1]{\ensuremath{\left[\begin{array}{r}#1\end{array}\right]}}

% vector notation
%\newcommand{\vv}{\overset{\rightharpoonup}}
\newcommand{\vv}[1]{{\bf #1}}
\newcommand{\arr}{\overrightarrow}

% dot product
\newcommand{\dotp}{{\scriptstyle\bullet}}

% Text macros
\newcommand{\KER}[1]{\ensuremath{\text{ker}\left(#1\right)}}
\newcommand{\IMG}[1]{\ensuremath{\text{im}\left(#1\right)}}
\newcommand{\CHAR}[1]{\ensuremath{\text{char}\left(#1\right)}}
\newcommand{\BIGO}[1]{\ensuremath{\mathcal{O}\left(#1\right)}}
\newcommand{\TR}[1]{\ensuremath{\text{tr}\left(#1\right)}}

% abbreviations
\newcommand{\ds}{\displaystyle}
\newcommand{\md}{\mdseries}
\newcommand{\vsp}{\vspace{0.5cm}}
\newcommand{\smsp}{\vspace{0.25cm}}
\newcommand{\hsp}{\hspace{0.25cm}}

% new operators
\DeclareMathOperator\SPAN{Span}
\newcommand{\SPANOF}[1]{\ensuremath{\SPAN\left\{#1\right\}}}
\DeclareMathOperator\PROJ{proj}
\DeclareMathOperator\PERP{perp}

% quick abbreviations to avoid using latex environments
\newcommand{\answer}{\noindent \textbf{Answer:} }
\newcommand{\answers}{\noindent \textbf{Answers:} }
\newcommand{\application}{\noindent \textbf{Application:} }
\newcommand{\caution}{\noindent \textbf{Caution:} }
\newcommand{\conclusion}{\noindent \textbf{Conclusion:} }
\newcommand{\consequence}{\noindent \textbf{Consequence:} }
\newcommand{\defn}{\noindent \textbf{Definition:} }
\newcommand{\details}{\noindent \textbf{Details:} }
\newcommand{\example}{\noindent \textbf{Example:} }
\newcommand{\examples}{\noindent \textbf{Examples:} }
\newcommand{\exception}{\noindent \textbf{Exception:} }
\newcommand{\exercise}{\noindent \textbf{Exercise:} }
\newcommand{\exercises}{\noindent \textbf{Exercises:} }
\newcommand{\fact}{\noindent \textbf{Fact:} }
\newcommand{\facts}{\noindent \textbf{Facts:} }
\newcommand{\formula}{\noindent \textbf{Formula:} }
\newcommand{\goal}{\noindent \textbf{Goal:} }
\newcommand{\goals}{\noindent \textbf{Goals:} }
\newcommand{\hint}{\noindent \textbf{Hint:} }
\newcommand{\idea}{\noindent \textbf{Idea:} }
\newcommand{\illustration}{\noindent \textbf{Illustration:} }
\newcommand{\important}{\noindent \textbf{Important:} }
\newcommand{\midea}{\noindent \textbf{Main Idea:} }
\newcommand{\motivation}{\noindent \textbf{Motivation:} }
\newcommand{\nthm}[1]{\noindent \textbf{Theorem} (\textit{#1}):}
\newcommand{\notation}{\noindent \textbf{Notation:} }
\newcommand{\note}{\noindent \textbf{Note:} }
\newcommand{\notes}{\noindent \textbf{Notes:} }
\newcommand{\observation}{\noindent \textbf{Observation:} }
\newcommand{\observations}{\noindent \textbf{Observations:} }
\newcommand{\pict}{\noindent \textbf{Picture:} }
\newcommand{\plan}{\noindent \textbf{Plan:} }
\newcommand{\prf}{\noindent \textbf{Proof:} }
\newcommand{\problem}{\noindent \textbf{Problem:} }
\newcommand{\properties}{\noindent \textbf{Properties:} }
\newcommand{\question}{\noindent \textbf{Question:} }
\newcommand{\questions}{\noindent \textbf{Questions:} }
\newcommand{\recall}{\noindent \textbf{Recall:} }
\newcommand{\reason}{\noindent \textbf{Reason:} }
\newcommand{\remark}{\noindent \textbf{Remark:} }
\newcommand{\remarks}{\noindent \textbf{Remarks:} }
\newcommand{\reminder}{\noindent \textbf{Reminder:} }
\newcommand{\solution}{\noindent \textbf{Solution:} }
\newcommand{\nsolution}[1]{\noindent \textbf{Solution #1:} }
\newcommand{\strategy}{\noindent \textbf{Strategy:} }
\newcommand{\summary}{\noindent \textbf{Summary:} }
\newcommand{\terminology}{\noindent \textbf{Terminology:} }
\newcommand{\thm}{\noindent \textbf{Theorem:} }
\newcommand{\work}{\noindent \textbf{Work:} }


\usepackage[version=4]{mhchem}

% Where to look for pngs and jpegs
\graphicspath{{Images//}}

\usepackage[includehead, includefoot, left= 2cm, top =1.5cm, bottom = 1.5cm, textwidth=17.5cm]{geometry}

\usepackage{pifont, amsmath}


\pagestyle{fancy}
\fancyhf{}
\renewcommand{\headrulewidth}{1pt}
%\fancyhead[R]{\bfseries\sffamily\thepage}
%\fancyfoot[C]{\bfseries\sffamily\thepage}
\fancyhead[L]{\nouppercase{\bfseries\sffamily\leftmark}}

% used when adding fill-in-the-blanks for students
\newcommand{\blank}[1]{\underline{\hspace{#1}}}

% indents annoy me, and so does repeatedly typing \noindent
\newcommand{\p}{\noindent}

\begin{document}

\fancyhead[L]{\course}
\fancyhead[C]{\includegraphics[width=5cm, trim= 0 0.4cm 0 0]{TRU_logo}}
\fancyhead[R]{\term}
\renewcommand{\headrulewidth}{0.4pt}

\setulcolor{red}

\setcounter{section}{3}
\section{Rings}
\setcounter{subsection}{1}
\subsection{Isomorphic Binary Algebraic Structures}


\recall If $X$ and $Y$ are any sets, a map $f:X\to Y$ is said to be
\vsmsp

\begin{itemize}
\item \DEF{injective} if
\vsp

\item \DEF{surjective} if 
\vsp

\item \DEF{bijective} if
\end{itemize}
\vsp

%
\begin{definition}
Let $(S,*)$ and $(S',*')$ be binary algebraic structures.  A bijection $\varphi:S\to S'$ is an \DEF{isomorphism} if
\vspace{2cm}

\p In this case we say that $(S,*)$ and $(S',*')$ are \DEF{isomorphic}.
\end{definition}
\vsp

\notation If $(S,*)$ and $(S',*')$ are isomorphic, we write
\vsp

%
\begin{example}\label{firstisomorphismexample}
Let $S = \SET{a,b,c}$ and $S' = \SET{1,2,3}$ be binary algebraic structures under the operations $*$ and $*'$ respectively defined by the following tables:
\begin{center}
\bgroup
\begin{center}
\def\arraystretch{1.5}
\begin{tabular}{c|ccc}
$*$ & $a$ & $b$ & $c$\\
\hline
$a$ & $b$ & $a$ & $c$\\
$b$ & $a$ & $c$ & $b$\\
$c$ & $c$ & $b$ & $a$\\
\end{tabular}
\hspace{3cm}
\begin{tabular}{c|ccc}
$*'$ & $1$ & $2$ & $3$\\
\hline
$1$ & $2$ & $1$ & $3$\\
$2$ & $1$ & $3$ & $2$\\
$3$ & $3$ & $2$ & $1$\\
\end{tabular}
\end{center}
\egroup
\end{center}
\end{example}
\vspace{8cm}

%
\notation For $a,n\in\Z$ with $n\neq 0$, let $a \% n$ denote the remainder when $a$ is divided by $n$.  For example, $15 \% 4 = 3$ and $8 \% 2 = 0$.
%
\begin{examples}~
\begin{enumerate}[(a)]
\item Recall that $\ZN = \SET{[a]:a\in\INTEGER}$ of congruence classes modulo $n$ and is equipped with addition and multiplication operations defined by
\begin{center}
$[a]\oplus [b] = [a + b]$ and $[a]\odot [b] = [ab]$, for all $[a],[b]\in\ZN$.
\end{center}
\p Note that $(\ZN,\oplus)$ and $(\ZN,\odot)$ are both binary algebraic structures.
\vsmsp

\p Consider now the set $\overline{\Z}_{n} = \SET{0,1,2,\ldots, n-1}$, which is also a binary algebraic structure under either of the addition $+_{n}$ and multiplication $\cdot_{n}$ operations defined by
\begin{center}
$a+_{n}b = (a + b)\% n$,
\end{center}
and
\begin{center}
$a\cdot_{n}b = (ab)\% n$, for all $a,b\in\overline{\Z}_{n}$.
\end{center}
\vsmsp

\begin{enumerate}[(i)]
\item Prove that $(\ZN,\oplus)$ is isomorphic to $(\overline{\Z}_{n},+_{n})$.
\newpage

\item Prove that $(\ZN,\odot)$ is isomorphic to $(\overline{\Z}_{n},\cdot_{n})$.
\end{enumerate}
\vsp

%
\item The set $\REAL^{+} = \SET{x\in\REAL:x>0}$ is a binary algebraic structure under multiplication.
\vspace{6cm}

\end{enumerate}
\newpage
\end{examples}
%

\begin{lemma}\label{selfsquarelemma}
Suppose $(S,*)$ and $(S',*')$ are isomorphic binary algebraic structures, and that there exists a unique $x\in S$ such that $x*x = x$.  Prove that there exists a unique $y'\in S'$ such that $y'*'y' = y'$.
\end{lemma}
%
\begin{proof}\phantom{-}
\end{proof}
\vspace{8cm}

%
\begin{theorem}
$(\Z,+)$ and $(\Z,\cdot)$ are not isomorphic.
\end{theorem}
%
\begin{proof}
\end{proof}
\vspace{6cm}

%
\begin{definition}
Let $(S,*)$ be a binary algebraic structure.  $e\in S$ is an \DEF{identity element} for $*$ if $e*x = x$ and $x*e = x$ for all $x\in S$.
\end{definition}
%
\begin{examples}~
\begin{enumerate}[(a)]
\item The identity element of $(\Z,+)$ is
\vspace{2cm}

\item The identity element of $(\REAL,\cdot)$ is
\vspace{2cm}

\item The identity element of $D_3$ is
\vspace{2cm}

\item The identity element of $M_{2}(\REAL)$ is
\vspace{2cm}

\item The binary algebraic structure $(S,*)$ with $S = \SET{a,b,c}$ and multiplication $*$ defined by
\begin{center}
\begin{tabular}{c|ccc}
$*$ & $a$ & $b$ & $c$\\
\hline
$a$ & $b$ & $a$ & $c$\\
$b$ & $a$ & $c$ & $b$\\
$c$ & $c$ & $b$ & $a$\\
\end{tabular}
\end{center}
has no identity element.
\end{enumerate}
\end{examples}
%
\begin{remark}\label{uniqueidentities}
A binary algebraic structure $(S,*)$ can have at most one identity element $e$.
\end{remark}
\vspace{3cm}

%
\begin{exercise}
Suppose $(S,*)$ and $(S',*')$ are two isomorphic binary algebraic structures.  Prove that if $S$ has an identity element for $*$, that $S'$ has an identity element for $*'$.
\end{exercise}

\end{document}





