% Exam Template using Philip Hirschhorn's exam.cls: http://www-math.mit.edu/~psh/#ExamCls
%
% run pdflatex on a finished exam at least three times to do the grading table on front page.
%
%%%%%%%%%%%%%%%%%%%%%%%%%%%%%%%%%%%%%%%%%%%%%%%%%%%%%%%%%%%%%%%%%%%%%%%%

% These lines can probably stay unchanged, although you can remove the last two packages if you're not making pictures with tikz.
\documentclass[11pt]{exam}
\RequirePackage{amssymb, amsfonts, amsmath, latexsym, verbatim, xspace, setspace, graphicx, pgfplots, paralist, tikz, xcolor}

% By default LaTeX uses large margins.  This doesn't work well on exams; problems end up in the "middle" of the page, reducing the amount of space for students to work on them.
\usepackage[margin=1in]{geometry}

\graphicspath{{Extra//}}

% Here's where you edit the Class, Exam, Date, etc.
\newcommand{\class}{Math 1070}
\newcommand{\term}{Winter 2024}
\newcommand{\examnum}{Quiz 1}
\newcommand{\ds}{\displaystyle}
\newcommand{\vsp}{\vspace{0.5cm}}

% For an exam, single spacing is most appropriate
\singlespacing
% \onehalfspacing
% \doublespacing

% For an exam, we generally want to turn off paragraph indentation
\parindent 0ex

\begin{document} 

% These commands set up the running header on the top of the exam pages
\pagestyle{head}
\firstpageheader{}{}{}
\runningheader{\class}{ }{\examnum\ - Page \thepage\ of \numpages}
\runningheadrule

\begin{flushright}
\begin{tabular}{p{3.8in} r l}
\textbf{\class} & \textbf{Full Name:} & \makebox[2in]{\hrulefill}\\
\textbf{\term} & \textbf{ID:} & \makebox[2in]{\hrulefill}\\
\textbf{\examnum} &&\\
\end{tabular}\\
\end{flushright}
\rule[1ex]{\textwidth}{.1pt}

%%%%%%%%%%%%%%%%%%%%%%%%%%%%%%%%%%%%%%%%%%%%%%%%%%%%%%%%%%%%%%%%%%%%%%%%%%%%%%%%%%%%%
%
% See http://www-math.mit.edu/~psh/#examCls for full documentation, but the questions
% below give an idea of how to write questions [with parts] and have the points
% tracked automatically on the cover page.
%
%
%%%%%%%%%%%%%%%%%%%%%%%%%%%%%%%%%%%%%%%%%%%%%%%%%%%%%%%%%%%%%%%%%%%%%%%%%%%%%%%%%%%%%

\begin{questions}

\setlength\answerskip{0pt}

\addpoints
% Question
\question[3] Let $z = re^{i\theta}$, where $\theta = 2\pi/n$ for some integer $n\geq 1$.

\end{questions}
\end{document}