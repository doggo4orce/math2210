% Exam Template using Philip Hirschhorn's exam.cls: http://www-math.mit.edu/~psh/#ExamCls
%
% run pdflatex on a finished exam at least three times to do the grading table on front page.
%
%%%%%%%%%%%%%%%%%%%%%%%%%%%%%%%%%%%%%%%%%%%%%%%%%%%%%%%%%%%%%%%%%%%%%%%%

% These lines can probably stay unchanged, although you can remove the last two packages if you're not making pictures with tikz.
\documentclass[11pt]{exam}
\RequirePackage{amssymb, amsfonts, amsmath, latexsym, verbatim, xspace, setspace, graphicx, pgfplots, paralist, tikz, xcolor}

% By default LaTeX uses large margins.  This doesn't work well on exams; problems end up in the "middle" of the page, reducing the amount of space for students to work on them.
\usepackage[margin=1in]{geometry}

\graphicspath{{Extra//}}

% Here's where you edit the Class, Exam, Date, etc.
\newcommand{\class}{Math 1070}
\newcommand{\term}{Winter 2024}
\newcommand{\examnum}{Quiz 1}
\newcommand{\ds}{\displaystyle}
\newcommand{\vsp}{\vspace{0.5cm}}

% For an exam, single spacing is most appropriate
\singlespacing
% \onehalfspacing
% \doublespacing

% For an exam, we generally want to turn off paragraph indentation
\parindent 0ex

\begin{document} 

% These commands set up the running header on the top of the exam pages
\pagestyle{head}
\firstpageheader{}{}{}
\runningheader{\class}{ }{\examnum\ - Page \thepage\ of \numpages}
\runningheadrule

\begin{flushright}
\begin{tabular}{p{3.8in} r l}
\textbf{\class} & \textbf{Full Name:} & \makebox[2in]{\hrulefill}\\
\textbf{\term} & \textbf{ID:} & \makebox[2in]{\hrulefill}\\
\textbf{\examnum} &&\\
\end{tabular}\\
\end{flushright}
\rule[1ex]{\textwidth}{.1pt}

%%%%%%%%%%%%%%%%%%%%%%%%%%%%%%%%%%%%%%%%%%%%%%%%%%%%%%%%%%%%%%%%%%%%%%%%%%%%%%%%%%%%%
%
% See http://www-math.mit.edu/~psh/#examCls for full documentation, but the questions
% below give an idea of how to write questions [with parts] and have the points
% tracked automatically on the cover page.
%
%
%%%%%%%%%%%%%%%%%%%%%%%%%%%%%%%%%%%%%%%%%%%%%%%%%%%%%%%%%%%%%%%%%%%%%%%%%%%%%%%%%%%%%
\noindent
{\large \textbf{Important Rules:}}

\begin{itemize}

\item \textbf{Organize your work}, in a reasonably neat and coherent way, in
the space provided. Work scattered all over the page without a clear ordering will 
receive very little credit.  

\item \textbf{Mysterious or unsupported answers may not receive full
credit}.  A correct answer, unsupported by calculations, explanation,
or algebraic work may receive little or no credit; an incorrect answer supported
by some correct calculations will likely receive partial credit.

\item \textbf{Phones should be turned off or in silent mode} -- and they should not be on your desk.  Put them away in a backpack or bag.  The same goes for smart watches.

\item \textbf{You may use a non-programmable scientific calculator.}  All you should have with you at your desk is a pencil (or pen), an eraser, and a calculator.  Beverages are fine, but nothing else should be nearby.  

\item \textbf{No scrap paper or formula sheets are allowed.}  If you need extra space, raise your hand and I will bring you extra paper.

\end{itemize}

\begin{questions}

\setlength\answerskip{0pt}

\addpoints
% Question
\question[3] Find the domain of the function
\begin{center}
$\ds{f(x) = \frac{\sqrt{2-5x}}{x - 10}}$
\end{center}
\vfill

\question[3] Find the equation of the line that passes through the points $(-5,9)$ and $(-4,5)$.
\vspace{3cm}
\newpage

\question[4] An electronics retailer selling computers has fixed costs of \$11,000 per month.  For each desktop computer ordered from the supplier, they are billed for the following:
\begin{itemize}
\item \$15 for the keyboard,
\item \$5 for the mouse,
\item \$120 for the monitor, 
\item \$80 for the speakers, and
\item \$400 for the tower.
\end{itemize}
\vsp

\noindent The retailer sells these computers for \$2500 each.  Let $q$ be the number of computers ordered and sold.

\begin{enumerate}[(a)]
\item Express the total cost as a function of $q$.
\vfill

\item Express the total revenue as a function of $q$.
\vfill

\item How much does it cost the retailer to order 33 computers from the supplier?
\vfill

\item What is the total profit if $20$ computers are ordered and sold in a month?
\vfill
\end{enumerate}
\end{questions}
\end{document}