% Exam Template using Philip Hirschhorn's exam.cls: http://www-math.mit.edu/~psh/#ExamCls
%
% run pdflatex on a finished exam at least three times to do the grading table on front page.
%
%%%%%%%%%%%%%%%%%%%%%%%%%%%%%%%%%%%%%%%%%%%%%%%%%%%%%%%%%%%%%%%%%%%%%%%%

% These lines can probably stay unchanged, although you can remove the last two packages if you're not making pictures with tikz.
\documentclass[11pt, legalpaper]{exam}
\RequirePackage{amssymb, amsfonts, amsmath, latexsym, verbatim, xspace, setspace, graphicx, enumerate, paralist, array}
\newcolumntype{L}[1]{>{\raggedright\let\newline\\\arraybackslash\hspace{0pt}}m{#1}}
\newcolumntype{C}[1]{>{\centering\let\newline\\\arraybackslash\hspace{0pt}}m{#1}}
\newcolumntype{R}[1]{>{\raggedleft\let\newline\\\arraybackslash\hspace{0pt}}m{#1}}

% By default LaTeX uses large margins.  This doesn't work well on exams; problems end up in the "middle" of the page, reducing the amount of space for students to work on them.
\usepackage[margin=1in]{geometry}


% Here's where you edit the Class, Exam, Date, etc.
\newcommand{\class}{Math 1070}
\newcommand{\term}{Winter 2023}
\newcommand{\examnum}{Midterm Exam 2}
\newcommand{\examdate}{March 10th}
\newcommand{\timelimit}{50 Minutes}


% For an exam, single spacing is most appropriate
\singlespacing
% \onehalfspacing
% \doublespacing

% For an exam, we generally want to turn off paragraph indentation
\parindent 0ex

% packages to include

\usepackage[dvipsnames, table]{xcolor}

\usepackage{
  amsthm,
  amsmath,
  amssymb, 
  arydshln, % for hyphenated lines in block matrices
  fancyhdr, % needed for header at top of each page
  graphicx, % to include pictures
  mathtools, % for a longer arrow
  multicol, % displaying enumerates and itemizes into multiple columns
  multirow, % for tables
  multido, % for TOC
  pgfplots, % for axis environment within tikz pictures
  systeme,
  tikz,
}

\usepackage[utf8]{inputenc}
\usepackage{color,soul}

\usepackage[inline, shortlabels]{enumitem}
\usepackage[hidelinks]{hyperref}



% global constants
\newcommand{\term}{Winter 2024}

% mathbb aliases
\newcommand{\COMPLEX}{\mathbb{C}}
\newcommand{\REAL}{\mathbb{R}}
\newcommand{\NATURAL}{\mathbb{N}}
\newcommand{\INTEGER}{\mathbb{Z}}

% for financial stuff
\newcommand{\dollar}{\mathrm{\$}}

% nicer looking trig functions
\newcommand{\SIN}[1]{\sin\left(#1\right)}
\newcommand{\COS}[1]{\cos\left(#1\right)}
\newcommand{\TAN}[1]{\tan\left(#1\right)}
\newcommand{\CSC}[1]{\csc\left(#1\right)}
\newcommand{\SEC}[1]{\sec\left(#1\right)}
\newcommand{\COT}[1]{\cot\left(#1\right)}

% automatically resize set brackets
\newcommand{\SET}[1]{\left\{#1\right\}}

% sums and products
\newcommand{\SUM}{\displaystyle\sum\limits}
\newcommand{\PROD}{\displaystyle\prod\limits}
\newcommand{\of}{\circ}
\newcommand{\restrict}[1]{\raisebox{-.5ex}{$|$}_{#1}}

% set intersection and union
\newcommand{\CAP}{\displaystyle\bigcap\limits}
\newcommand{\CUP}{\displaystyle\bigcup\limits}

% max and min
\newcommand{\MAX}[1]{\ensuremath{\max\left(#1\right)}}
\newcommand{\MIN}[1]{\ensuremath{\min\left(#1\right)}}

% for writing logic within mathematics environment
\newcommand{\FORALL}{\ensuremath{\text{ for all }}}
\newcommand{\FORSOME}{\ensuremath{\text{ for some }}}

% matrix notation
\newcommand{\MATRIX}[2]{\ensuremath{\left[\begin{array}{#1}#2\end{array}\right]}}
\newcommand{\COLUMN}[1]{\ensuremath{\left[\begin{array}{r}#1\end{array}\right]}}

% vector notation
%\newcommand{\vv}{\overset{\rightharpoonup}}
\newcommand{\vv}[1]{{\bf #1}}
\newcommand{\arr}{\overrightarrow}

% dot product
\newcommand{\dotp}{{\scriptstyle\bullet}}

% Text macros
\newcommand{\KER}[1]{\ensuremath{\text{ker}\left(#1\right)}}
\newcommand{\IMG}[1]{\ensuremath{\text{im}\left(#1\right)}}
\newcommand{\CHAR}[1]{\ensuremath{\text{char}\left(#1\right)}}
\newcommand{\BIGO}[1]{\ensuremath{\mathcal{O}\left(#1\right)}}
\newcommand{\TR}[1]{\ensuremath{\text{tr}\left(#1\right)}}

% abbreviations
\newcommand{\ds}{\displaystyle}
\newcommand{\md}{\mdseries}
\newcommand{\vsp}{\vspace{0.5cm}}
\newcommand{\smsp}{\vspace{0.25cm}}
\newcommand{\hsp}{\hspace{0.25cm}}

% new operators
\DeclareMathOperator\SPAN{Span}
\newcommand{\SPANOF}[1]{\ensuremath{\SPAN\left\{#1\right\}}}
\DeclareMathOperator\PROJ{proj}
\DeclareMathOperator\PERP{perp}

% quick abbreviations to avoid using latex environments
\newcommand{\answer}{\noindent \textbf{Answer:} }
\newcommand{\answers}{\noindent \textbf{Answers:} }
\newcommand{\application}{\noindent \textbf{Application:} }
\newcommand{\caution}{\noindent \textbf{Caution:} }
\newcommand{\conclusion}{\noindent \textbf{Conclusion:} }
\newcommand{\consequence}{\noindent \textbf{Consequence:} }
\newcommand{\defn}{\noindent \textbf{Definition:} }
\newcommand{\details}{\noindent \textbf{Details:} }
\newcommand{\example}{\noindent \textbf{Example:} }
\newcommand{\examples}{\noindent \textbf{Examples:} }
\newcommand{\exception}{\noindent \textbf{Exception:} }
\newcommand{\exercise}{\noindent \textbf{Exercise:} }
\newcommand{\exercises}{\noindent \textbf{Exercises:} }
\newcommand{\fact}{\noindent \textbf{Fact:} }
\newcommand{\facts}{\noindent \textbf{Facts:} }
\newcommand{\formula}{\noindent \textbf{Formula:} }
\newcommand{\goal}{\noindent \textbf{Goal:} }
\newcommand{\goals}{\noindent \textbf{Goals:} }
\newcommand{\hint}{\noindent \textbf{Hint:} }
\newcommand{\idea}{\noindent \textbf{Idea:} }
\newcommand{\illustration}{\noindent \textbf{Illustration:} }
\newcommand{\important}{\noindent \textbf{Important:} }
\newcommand{\midea}{\noindent \textbf{Main Idea:} }
\newcommand{\motivation}{\noindent \textbf{Motivation:} }
\newcommand{\nthm}[1]{\noindent \textbf{Theorem} (\textit{#1}):}
\newcommand{\notation}{\noindent \textbf{Notation:} }
\newcommand{\note}{\noindent \textbf{Note:} }
\newcommand{\notes}{\noindent \textbf{Notes:} }
\newcommand{\observation}{\noindent \textbf{Observation:} }
\newcommand{\observations}{\noindent \textbf{Observations:} }
\newcommand{\pict}{\noindent \textbf{Picture:} }
\newcommand{\plan}{\noindent \textbf{Plan:} }
\newcommand{\prf}{\noindent \textbf{Proof:} }
\newcommand{\problem}{\noindent \textbf{Problem:} }
\newcommand{\properties}{\noindent \textbf{Properties:} }
\newcommand{\question}{\noindent \textbf{Question:} }
\newcommand{\questions}{\noindent \textbf{Questions:} }
\newcommand{\recall}{\noindent \textbf{Recall:} }
\newcommand{\reason}{\noindent \textbf{Reason:} }
\newcommand{\remark}{\noindent \textbf{Remark:} }
\newcommand{\remarks}{\noindent \textbf{Remarks:} }
\newcommand{\reminder}{\noindent \textbf{Reminder:} }
\newcommand{\solution}{\noindent \textbf{Solution:} }
\newcommand{\nsolution}[1]{\noindent \textbf{Solution #1:} }
\newcommand{\strategy}{\noindent \textbf{Strategy:} }
\newcommand{\summary}{\noindent \textbf{Summary:} }
\newcommand{\terminology}{\noindent \textbf{Terminology:} }
\newcommand{\thm}{\noindent \textbf{Theorem:} }
\newcommand{\work}{\noindent \textbf{Work:} }


\begin{document} 

% These commands set up the running header on the top of the exam pages
\pagestyle{head}
\firstpageheader{}{}{}
\runningheader{\class}{\examnum\ - Page \thepage\ of \numpages}{\examdate}
\runningheadrule

\begin{flushright}
\begin{tabular}{p{3.8in} r l}
\textbf{\class} & \textbf{Full Name:} & \makebox[2in]{\hrulefill}\\
\textbf{\term} & \textbf{ID:} & \makebox[2in]{\hrulefill}\\
\textbf{\examnum} &&\\
\textbf{\examdate} &&\\
\textbf{Time Limit: \timelimit} &
\end{tabular}\\
\end{flushright}
\rule[1ex]{\textwidth}{.1pt}

\textbf{Reminders:} 
\begin{itemize}

\item \textbf{Organize your work}, in a reasonably neat and coherent way, in the space provided. Work scattered all over the page without a clear ordering will 
receive very little credit.

\item \textbf{Clearly identify your final answer}, by circling it or enclosing it in a box.

\item \textbf{Unsupported answers may not receive full credit}.  A correct answer, unsupported by calculations, explanation,
or algebraic work may receive little or no credit; an incorrect answer supported by some correct calculations will likely receive partial credit.

\item \textbf{Phones should be turned off or in silent mode} -- and they should not be on your desk.  Put them away in a backpack or bag.  The same goes for smart watches.

\item \textbf{You may use a non-programmable scientific calculator.}  All you should have with you at your desk is a pencil (or pen), an eraser, and a calculator.  Beverages are fine, but nothing else should be nearby.  

\item \textbf{No scrap paper or formula sheets are allowed.}  If you need extra space, raise your hand and I will bring you extra paper.

\item \textbf{There is to be absolutely no talking}, whispering, or communicating in any way with anyone (except your instructor) during the test at any time.
\end{itemize}
\vsp

\addpoints

\noindent
\begin{center}
\gradetablestretch{2}
\vqword{Page}
\gradetable[v][pages]  % Use [pages] to have grading table by page instead of question
\end{center}


\newpage % End of cover page

%%%%%%%%%%%%%%%%%%%%%%%%%%%%%%%%%%%%%%%%%%%%%%%%%%%%%%%%%%%%%%%%%%%%%%%%%%%%%%%%%%%%%
%
% See http://www-math.mit.edu/~psh/#examCls for full documentation, but the questions
% below give an idea of how to write questions [with parts] and have the points
% tracked automatically on the cover page.
%
%
%%%%%%%%%%%%%%%%%%%%%%%%%%%%%%%%%%%%%%%%%%%%%%%%%%%%%%%%%%%%%%%%%%%%%%%%%%%%%%%%%%%%%

\begin{questions}
\question[4] Evaluate each of the matrix expressions if they are defined.  Otherwise, say that they are undefined.
\begin{enumerate}[(a)]
\item $\MATRIX{rrr}{1 & 5 & 2\\-1 & 0 & 1\\3 & 2 & 4} - 2\MATRIX{rrr}{6 & 1 & 3\\-1 & 1 & 2\\4 & 1 & 3}$
\vfill

\item $\MATRIX{rr}{4 & -1\\0 & 2}\MATRIX{rrr}{1 & 4 & 2\\3 & 1 & 5}$
\vfill
\end{enumerate}

\question[2] Express $\MATRIX{rrr}{5 & 6 & -7\\-1 & -2 & 3\\0 & 4 & -1}\COLUMN{u\\v\\w} = \COLUMN{2\\0\\3}$ as a linear system.

\textbf{Do not solve the system.}
\vfill

\question[2] Express \systeme{2a - 3b + 5c = 7,9a - b + c = -1,a + 5b + 4c = 0} as a matrix equation.

\textbf{Do not solve the system.}
\vfill

\newpage

\question[4] Solve the linear system \systeme{x + y - z = 7, 2x - 3y - 2z = 4,x - y - 5z = 23}
\vfill
\vspace{6cm}

\question[4] Show that the matrix $A = \MATRIX{rrr}{1 & 6 & 4\\2 & 4 & -1\\-1 & 2 & 5}$ is not invertible.
\vfill

\newpage

\question[3] Expand the logarithm $\ds{\ln\left[\frac{(x+1)^2(x+2)}{\sqrt{x}}\right]}$ as much as possible.
\vfill

\question[3] Combine all of the logarithms in the expression
\begin{center}
$\ds{7\log_{3}(5) + 4\log_{3}(17) - \frac{1}{2}\log_{3}(x)}$
\end{center}
into a single logarithm.\\
\vfill

\newpage

\fullwidth{\textbf{Please read all of the parts of this question before you start, and structure your solution accordingly.}}

\question[8] Consider the following linear programming problem.

\begin{quote}
\textbf{Minimize} the objective function
\[
C(x,y)=3x+5y
\]
subject to the constraints
\[
\begin{cases}
3x+y\geq 45\\
x+2y\geq 60\\
x,y\geq 0
\end{cases}
\]
\end{quote}
Use the space below to draw the feasible set and find all corner points of the feasible set.

\bigskip

\hfill
\begin{tikzpicture}[scale=1.2]
\begin{axis}[
	scale only axis,
	grid=both,
	grid style={line width=0.5pt, draw=gray!30},
    axis equal image,
    axis lines=middle,
    x axis line style={<->},
    y axis line style={<->},
    ticklabel style={font=\tiny},
    xtick distance=10,
    ytick distance=10,
    xmin=-15,
    xmax=85,
    ymin=-15,
    ymax=85,
    samples=50
]
\end{axis}
\end{tikzpicture}

\begin{enumerate}[(a)]
\item Use the Existence Theorem to argue that this problem has a solution.
\vspace{4cm}

\item List \textbf{all} corner points of the feasible set.
\vspace{4cm}

\item Find the \textbf{minimum} value of $C(x,y)$ subject to the given constraints.
\end{enumerate}
\newpage

\fullwidth{\textbf{Set up but do not solve the following linear programming problem.}}
\question[5] Bob is paying extra close attention to his vitamin intake.  Each day, he wants to make sure he takes:
\begin{itemize}
\item 60mg of Vitamin C,
\item 1,000mg of Calcium,
\item 18mg of Iron,
\item 20mg of Niacin, and
\item 360mg of Magnesium.
\end{itemize}
He is considering two different nutritional supplements: Vega Vita, which costs \$0.20/tablet, and Happy Health, which costs \$0.30/tablet.  Each tablet of Vega Vita contains 20mg of Vitamin C, 500mg of Calcium, 9mg of Iron, 2mg of Niacin, and 60mg of Magnesium.  Each tablet of  Happy Health contains 30mg of Vitamin C, 250mg of Calcium, 2mg of Iron, 10mg of Niacin, and 90mg of Magnesium.  What should his daily pill regiment be so that he achieves all of his nutritional needs, while spending as little money as possible?
\end{questions}
\end{document}