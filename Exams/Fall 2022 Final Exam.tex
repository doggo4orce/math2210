% Exam Template using Philip Hirschhorn's exam.cls: http://www-math.mit.edu/~psh/#ExamCls
%
% run pdflatex on a finished exam at least three times to do the grading table on front page.
%
%%%%%%%%%%%%%%%%%%%%%%%%%%%%%%%%%%%%%%%%%%%%%%%%%%%%%%%%%%%%%%%%%%%%%%%%

% These lines can probably stay unchanged, although you can remove the last two packages if you're not making pictures with tikz.
\documentclass[12pt,legalpaper]{exam}
\RequirePackage{amssymb, amsfonts, amsmath, latexsym, verbatim, xspace, setspace, graphicx, enumerate, paralist, array}
\newcolumntype{L}[1]{>{\raggedright\let\newline\\\arraybackslash\hspace{0pt}}m{#1}}
\newcolumntype{C}[1]{>{\centering\let\newline\\\arraybackslash\hspace{0pt}}m{#1}}
\newcolumntype{R}[1]{>{\raggedleft\let\newline\\\arraybackslash\hspace{0pt}}m{#1}}

% By default LaTeX uses large margins.  This doesn't work well on exams; problems end up in the "middle" of the page, reducing the amount of space for students to work on them.
\usepackage[margin=1in]{geometry}


% Here's where you edit the Class, Exam, Date, etc.
\newcommand{\class}{Math 1070}
\newcommand{\term}{}
\newcommand{\examnum}{Final Exam}
\newcommand{\examdate}{}
\newcommand{\timelimit}{3 hours}


% For an exam, single spacing is most appropriate
\singlespacing
% \onehalfspacing
% \doublespacing

% For an exam, we generally want to turn off paragraph indentation
\parindent 0ex

% packages to include

\usepackage[dvipsnames, table]{xcolor}

\usepackage{
  amsthm,
  amsmath,
  amssymb, 
  arydshln, % for hyphenated lines in block matrices
  fancyhdr, % needed for header at top of each page
  graphicx, % to include pictures
  mathtools, % for a longer arrow
  multicol, % displaying enumerates and itemizes into multiple columns
  multirow, % for tables
  multido, % for TOC
  pgfplots, % for axis environment within tikz pictures
  systeme,
  tikz,
}

\usepackage[utf8]{inputenc}
\usepackage{color,soul}

\usepackage[inline, shortlabels]{enumitem}
\usepackage[hidelinks]{hyperref}



% global constants
\newcommand{\term}{Winter 2024}

% mathbb aliases
\newcommand{\COMPLEX}{\mathbb{C}}
\newcommand{\REAL}{\mathbb{R}}
\newcommand{\NATURAL}{\mathbb{N}}
\newcommand{\INTEGER}{\mathbb{Z}}

% for financial stuff
\newcommand{\dollar}{\mathrm{\$}}

% nicer looking trig functions
\newcommand{\SIN}[1]{\sin\left(#1\right)}
\newcommand{\COS}[1]{\cos\left(#1\right)}
\newcommand{\TAN}[1]{\tan\left(#1\right)}
\newcommand{\CSC}[1]{\csc\left(#1\right)}
\newcommand{\SEC}[1]{\sec\left(#1\right)}
\newcommand{\COT}[1]{\cot\left(#1\right)}

% automatically resize set brackets
\newcommand{\SET}[1]{\left\{#1\right\}}

% sums and products
\newcommand{\SUM}{\displaystyle\sum\limits}
\newcommand{\PROD}{\displaystyle\prod\limits}
\newcommand{\of}{\circ}
\newcommand{\restrict}[1]{\raisebox{-.5ex}{$|$}_{#1}}

% set intersection and union
\newcommand{\CAP}{\displaystyle\bigcap\limits}
\newcommand{\CUP}{\displaystyle\bigcup\limits}

% max and min
\newcommand{\MAX}[1]{\ensuremath{\max\left(#1\right)}}
\newcommand{\MIN}[1]{\ensuremath{\min\left(#1\right)}}

% for writing logic within mathematics environment
\newcommand{\FORALL}{\ensuremath{\text{ for all }}}
\newcommand{\FORSOME}{\ensuremath{\text{ for some }}}

% matrix notation
\newcommand{\MATRIX}[2]{\ensuremath{\left[\begin{array}{#1}#2\end{array}\right]}}
\newcommand{\COLUMN}[1]{\ensuremath{\left[\begin{array}{r}#1\end{array}\right]}}

% vector notation
%\newcommand{\vv}{\overset{\rightharpoonup}}
\newcommand{\vv}[1]{{\bf #1}}
\newcommand{\arr}{\overrightarrow}

% dot product
\newcommand{\dotp}{{\scriptstyle\bullet}}

% Text macros
\newcommand{\KER}[1]{\ensuremath{\text{ker}\left(#1\right)}}
\newcommand{\IMG}[1]{\ensuremath{\text{im}\left(#1\right)}}
\newcommand{\CHAR}[1]{\ensuremath{\text{char}\left(#1\right)}}
\newcommand{\BIGO}[1]{\ensuremath{\mathcal{O}\left(#1\right)}}
\newcommand{\TR}[1]{\ensuremath{\text{tr}\left(#1\right)}}

% abbreviations
\newcommand{\ds}{\displaystyle}
\newcommand{\md}{\mdseries}
\newcommand{\vsp}{\vspace{0.5cm}}
\newcommand{\smsp}{\vspace{0.25cm}}
\newcommand{\hsp}{\hspace{0.25cm}}

% new operators
\DeclareMathOperator\SPAN{Span}
\newcommand{\SPANOF}[1]{\ensuremath{\SPAN\left\{#1\right\}}}
\DeclareMathOperator\PROJ{proj}
\DeclareMathOperator\PERP{perp}

% quick abbreviations to avoid using latex environments
\newcommand{\answer}{\noindent \textbf{Answer:} }
\newcommand{\answers}{\noindent \textbf{Answers:} }
\newcommand{\application}{\noindent \textbf{Application:} }
\newcommand{\caution}{\noindent \textbf{Caution:} }
\newcommand{\conclusion}{\noindent \textbf{Conclusion:} }
\newcommand{\consequence}{\noindent \textbf{Consequence:} }
\newcommand{\defn}{\noindent \textbf{Definition:} }
\newcommand{\details}{\noindent \textbf{Details:} }
\newcommand{\example}{\noindent \textbf{Example:} }
\newcommand{\examples}{\noindent \textbf{Examples:} }
\newcommand{\exception}{\noindent \textbf{Exception:} }
\newcommand{\exercise}{\noindent \textbf{Exercise:} }
\newcommand{\exercises}{\noindent \textbf{Exercises:} }
\newcommand{\fact}{\noindent \textbf{Fact:} }
\newcommand{\facts}{\noindent \textbf{Facts:} }
\newcommand{\formula}{\noindent \textbf{Formula:} }
\newcommand{\goal}{\noindent \textbf{Goal:} }
\newcommand{\goals}{\noindent \textbf{Goals:} }
\newcommand{\hint}{\noindent \textbf{Hint:} }
\newcommand{\idea}{\noindent \textbf{Idea:} }
\newcommand{\illustration}{\noindent \textbf{Illustration:} }
\newcommand{\important}{\noindent \textbf{Important:} }
\newcommand{\midea}{\noindent \textbf{Main Idea:} }
\newcommand{\motivation}{\noindent \textbf{Motivation:} }
\newcommand{\nthm}[1]{\noindent \textbf{Theorem} (\textit{#1}):}
\newcommand{\notation}{\noindent \textbf{Notation:} }
\newcommand{\note}{\noindent \textbf{Note:} }
\newcommand{\notes}{\noindent \textbf{Notes:} }
\newcommand{\observation}{\noindent \textbf{Observation:} }
\newcommand{\observations}{\noindent \textbf{Observations:} }
\newcommand{\pict}{\noindent \textbf{Picture:} }
\newcommand{\plan}{\noindent \textbf{Plan:} }
\newcommand{\prf}{\noindent \textbf{Proof:} }
\newcommand{\problem}{\noindent \textbf{Problem:} }
\newcommand{\properties}{\noindent \textbf{Properties:} }
\newcommand{\question}{\noindent \textbf{Question:} }
\newcommand{\questions}{\noindent \textbf{Questions:} }
\newcommand{\recall}{\noindent \textbf{Recall:} }
\newcommand{\reason}{\noindent \textbf{Reason:} }
\newcommand{\remark}{\noindent \textbf{Remark:} }
\newcommand{\remarks}{\noindent \textbf{Remarks:} }
\newcommand{\reminder}{\noindent \textbf{Reminder:} }
\newcommand{\solution}{\noindent \textbf{Solution:} }
\newcommand{\nsolution}[1]{\noindent \textbf{Solution #1:} }
\newcommand{\strategy}{\noindent \textbf{Strategy:} }
\newcommand{\summary}{\noindent \textbf{Summary:} }
\newcommand{\terminology}{\noindent \textbf{Terminology:} }
\newcommand{\thm}{\noindent \textbf{Theorem:} }
\newcommand{\work}{\noindent \textbf{Work:} }


\begin{document} 

% These commands set up the running header on the top of the exam pages
\pagestyle{head}
\firstpageheader{}{}{}
\runningheader{\class}{\examnum\ - Page \thepage\ of \numpages}{\examdate}
\runningheadrule

\begin{flushright}
\begin{tabular}{p{3.8in} r l}
\textbf{\class} & \textbf{Full Name:} & \makebox[2in]{\hrulefill}\\
\textbf{\term} & \textbf{ID:} & \makebox[2in]{\hrulefill}\\
\textbf{\examnum} &&\\
\textbf{\examdate} &&\\
\textbf{Time Limit: \timelimit} &
\end{tabular}\\
\end{flushright}
\rule[1ex]{\textwidth}{.1pt}

\textbf{Reminders:} 
\begin{itemize}

\item \textbf{Organize your work}, in a reasonably neat and coherent way, in the space provided. Work scattered all over the page without a clear ordering will receive very little credit.

\item \textbf{Clearly identify your final answer}, by circling it or enclosing it in a box.

\item \textbf{Unsupported answers may not receive full credit}.  A correct answer, unsupported by calculations, explanation,
or algebraic work may receive little or no credit; an incorrect answer supported by some correct calculations will likely receive partial credit.

\item \textbf{Phones should be turned off or in silent mode} -- and they should not be on your desk.  Put them away in a backpack or bag.  The same goes for smart watches.

\item \textbf{You may use a non-programmable scientific calculator.}  All you should have with you at your desk is a pencil (or pen), an eraser, and a calculator.  Beverages are fine, but nothing else should be nearby.  

\item \textbf{No scrap paper or formula sheets are allowed.}  If you need extra space, raise your hand and I will bring you extra paper.

\end{itemize}
\vsp

\addpoints

\noindent
\begin{center}
\gradetablestretch{2}
\vqword{Page}
\gradetable[v][pages]  % Use [pages] to have grading table by page instead of question
\end{center}


\newpage % End of cover page

%%%%%%%%%%%%%%%%%%%%%%%%%%%%%%%%%%%%%%%%%%%%%%%%%%%%%%%%%%%%%%%%%%%%%%%%%%%%%%%%%%%%%
%
% See http://www-math.mit.edu/~psh/#examCls for full documentation, but the questions
% below give an idea of how to write questions [with parts] and have the points
% tracked automatically on the cover page.
%
%
%%%%%%%%%%%%%%%%%%%%%%%%%%%%%%%%%%%%%%%%%%%%%%%%%%%%%%%%%%%%%%%%%%%%%%%%%%%%%%%%%%%%%

\begin{questions}
\question[4] Solve the linear system \systeme{x + 2y + z = 5,x + y + 3z = 0,2x - 3y - z = 9}.
\newpage

\question[3] Let $A = \MATRIX{rrr}{0 & 1 & -1\\1 & 0 & -1\\3 & 2 & -6}$.
\begin{compactenum}[(a)]
\item Using multiplication, verify that $A^{-1} = \MATRIX{rrr}{2 & 4 & -1\\3 & 3 & -1\\2 & 3 & -1}$.
\vspace{8cm}

\item Write the linear system
\systeme{y - z = 8,x - z = 3,3x + 2y - 6z = -5} as a matrix equation.
\vspace{6cm}

\item Use $A^{-1}$ from part (a) to solve the matrix equation in part (c).
\end{compactenum}
\newpage

\question[2] Solve for $x$ in each of the following equations:
\begin{compactenum}[(a)]
\item $3 + 2^{3x + 5} = 35$
\vspace{8cm}

\item $\log(x) + \log(x - 3) = \log(18)$
\vspace{8cm}
\end{compactenum}

\question[2] Completely expand the logarithm $\ds{\log\left(\frac{6x^3}{z(y - 1)}\right)}$.
\newpage

\question[4] Given that $g(x) = 5 - \sqrt{x+4}$.
\begin{compactenum}[(a)]
\item What is the domain of $g$?
\vspace{5cm}

\item Find the $y$-intercept of $g$ (if any).
\vspace{5cm}

\item Find the $x$-intercept(s) of $g$ (if any).
\vspace{5cm}
\end{compactenum}

\question[3] Given that $f(x) = 1 - 3x$, find and simplify the difference quotient $\ds{\frac{f(x + h) - f(x)}{h}}$.
\newpage

\question[7] Consider the system of inequalities $\begin{cases}3x - 2y \geq 6\\x + 2y \leq 10\\x \leq 5\\y \geq 0\end{cases}$.
\vsp

\begin{compactenum}[(a)]
\item Graph its solution set.
\vsp

\begin{center}
\begin{tikzpicture}[scale=2.3]
\begin{axis}[
    grid=both,
    grid style={line width=0.5pt, draw=gray!30},
    x=0.2cm,
    y=0.2cm,
    axis lines=middle,
    x axis line style={<->},
    y axis line style={<->},
    ticklabel style={font=\tiny},
    xtick distance=5,
    ytick distance=5,
    minor tick num = 4,
    ymin=-10,
    ymax=10,
    xmin=-10,
    xmax=10,
    samples=50
]
\end{axis}
\end{tikzpicture}
\end{center}
\vspace{3cm}

\item Find the coordinates of each corner point of the region sketched in part (a).
\vspace{4cm}

\item Use the method of corners to maximize the objective function $P = 5x + 8y$ on the region sketched in part (a).
\end{compactenum}
\newpage

\question[5] The Slipperytread Tire Company makes its tires from a combination of rubber, wire, and mesh.  The company has fixed costs of \$15,000 per day.  In addition, each tire uses \$35 in labour and materials.  The tires sell for \$55 each.  Let $q$ be the number of tires produced and sold daily.
\begin{compactenum}[(a)]
\item Find each of the revenue, cost, and profit functions.
\vspace{1.5cm}
\begin{itemize}
\item[] $R(q) =$
\vspace{1.5cm}

\item[] $C(q) =$
\vspace{1.5cm}

\item[] $P(q) =$
\end{itemize}
\vspace{1.5cm}

\item How many tires must be sold each day for the company to break even?
\vspace{5cm}

\item How many must be sold to make a profit of \$2,500 per day?
\end{compactenum}
\vspace{5cm}

\question[3] Find the inverse of the matrix $A = \MATRIX{rr}{1 & -3\\2 & 5}$.
\newpage

\question[4] The supply and demand equations for boxes of napkins are given by
\begin{center}
$\text{(Supply) }p - \sqrt{q} = 5$ and $\text{(Demand) }q + p = 11$.
\end{center}
Find the equilibrium price $p_{0}$ and the equilibrium quantity $q_{0}$.
\vspace{15cm}

\question[3] Calculate the effective annual rate for each of the following interest plans.
\begin{compactenum}[(a)]
\item 5\% compounded continuously
\vspace{5cm}

\item 5.2\% compounded semi-annually
\vspace{5cm}
\end{compactenum}
\newpage

\question[5] A merchant observes on average, that
\begin{itemize}
\item 200 daily t-shirt sales at \$26 each, and
\smsp

\item 230 daily t-shirt sales at \$20 each.
\end{itemize}
\vsp

\begin{compactenum}[(a)]
\item Find the demand equation $p = D(q)$, assuming that $p$ and $q$ have a linear relationship.
\vspace{5cm}

\item Find the revenue function $R(q)$ as a function of quantity $q$ sold.
\vspace{8cm}

\item Find the number of t-shirts that should be sold daily to maximize the revenue.
\end{compactenum}
\newpage

\question[5] Consider the difference equation $y_{n} = -0.5y_{n-1} + 7$ with $y_{0} = 6$.
\begin{compactenum}[(a)]
\item State whether this determines a monotonic or oscillating sequence, and why.
\vspace{5cm}

\item What is the equilibrium value of this sequence?
\vspace{5cm}

\item State whether this sequence is attracted to, or repelled by its equilibrium value.
\vspace{5cm}

\item Sketch the sequence.
\vsp

\hfill
\begin{tikzpicture}[scale=1]
\begin{axis}[
%	grid=both,
%	axis equal image,
	%grid style={line width=0.5pt, draw=gray!30},
    scale only axis,
    axis lines=middle,
    x axis line style={->},
    y axis line style={<->},
    xtick distance=1,
    xticklabels={},
    yticklabels={},
    ymin=-7.5,
    ymax=22.5,
    xmin=0,
    xmax=6.5,
    samples=50
]
\end{axis}
\end{tikzpicture}
\end{compactenum}
\newpage

\question[4] A company has borrowed \$150,000.  The loan is to be repaid in monthly payments over 10 years.  Assume that the interest rate is 4\%, compounded semi-annually.
\begin{compactenum}[(a)]
\item Find the amount of each payment.
\vspace{10cm}

\item How much interest will be paid over the lifetime of this loan?
\vspace{8cm}

\item What will be the outstanding balance of of the loan after 2 years?
\end{compactenum}
\newpage

\question[3] A family wishes to have \$30,000 saved for their newborn child when they turn 18.  If the interest rate is 3.6\% compounded quarterly, how much should they invest at the end of each quarter for 18 years?
\vspace{13cm}

\question[3] A loan of \$8,000 is to be paid in 2 payments, the first in the amount of \$4,000 made at the end of the first year and the remaining balance paid off at the end of the third year.  If the interest rate is charged at 3\% and compounded monthly, what must the second payment be?
\newpage

\question[4] A scholarship fund currently has \$50,000 in it.  It earns interest at an effective rate of 3\%, and  8\% of the fund is withdrawn to pay out scholarships every year.  At the same time, \$5,000 is added to the fund each year from contributions.
\begin{compactenum}[(a)]
\item Model this problem with a difference equation.  Let $y_{n}$ be the balance of the fund after $n$ years.
\vspace{5cm}

\item Solve the difference equation.
\vspace{8cm}

\item What will the balance of the fund be after 5 years?
\end{compactenum}
\newpage
\end{questions}
\large
\textbf{Formulas:}
\begin{itemize}
\item $\ds{S = P(1 + rt)}$
\item $\ds{S = P(1 + r)^t}$
\item $\ds{S = P(1 + i)^n}$
\item $\ds{S = P(1 + r_{e})^{t}}$
\item $\ds{S = Pe^{rt}}$
\item $\ds{i = \frac{r}{m}}$
\item $\ds{n = mt}$
\item $\ds{r_{e} = (1 + \frac{r}{m})^{m} - 1}$
\item $\ds{r_{e} = e^{r} - 1}$
\item $\ds{P_{A} = \frac{R[1 - (1 + i)^{-n}]}{i}}$
\item $\ds{S_{A} = \frac{R[(1 + i)^{n} - 1]}{i}}$
\item $\ds{x = \frac{-b \pm \sqrt{b^2 - 4ac}}{2a}}$ 
\end{itemize}
\end{document}