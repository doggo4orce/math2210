% Exam Template using Philip Hirschhorn's exam.cls: http://www-math.mit.edu/~psh/#ExamCls
%
% run pdflatex on a finished exam at least three times to do the grading table on front page.
%
%%%%%%%%%%%%%%%%%%%%%%%%%%%%%%%%%%%%%%%%%%%%%%%%%%%%%%%%%%%%%%%%%%%%%%%%

% These lines can probably stay unchanged, although you can remove the last two packages if you're not making pictures with tikz.
\documentclass[11pt]{exam}
\RequirePackage{amssymb, amsfonts, amsmath, latexsym, verbatim, xspace, setspace, graphicx, enumerate, paralist, array}
\newcolumntype{L}[1]{>{\raggedright\let\newline\\\arraybackslash\hspace{0pt}}m{#1}}
\newcolumntype{C}[1]{>{\centering\let\newline\\\arraybackslash\hspace{0pt}}m{#1}}
\newcolumntype{R}[1]{>{\raggedleft\let\newline\\\arraybackslash\hspace{0pt}}m{#1}}

% By default LaTeX uses large margins.  This doesn't work well on exams; problems end up in the "middle" of the page, reducing the amount of space for students to work on them.
\usepackage[margin=1in]{geometry}


% Here's where you edit the Class, Exam, Date, etc.
\newcommand{\class}{Math 1070}
\newcommand{\term}{Fall 2024}
\newcommand{\examnum}{Midterm Exam 3}
\newcommand{\examdate}{March 27th}
\newcommand{\timelimit}{75 Minutes}


% For an exam, single spacing is most appropriate
\singlespacing
% \onehalfspacing
% \doublespacing

% For an exam, we generally want to turn off paragraph indentation
\parindent 0ex

% packages to include

\usepackage[dvipsnames, table]{xcolor}

\usepackage{
  amsthm,
  amsmath,
  amssymb, 
  arydshln, % for hyphenated lines in block matrices
  fancyhdr, % needed for header at top of each page
  graphicx, % to include pictures
  mathtools, % for a longer arrow
  multicol, % displaying enumerates and itemizes into multiple columns
  multirow, % for tables
  multido, % for TOC
  pgfplots, % for axis environment within tikz pictures
  systeme,
  tikz,
}

\usepackage[utf8]{inputenc}
\usepackage{color,soul}

\usepackage[inline, shortlabels]{enumitem}
\usepackage[hidelinks]{hyperref}



% global constants
\newcommand{\term}{Winter 2024}

% mathbb aliases
\newcommand{\COMPLEX}{\mathbb{C}}
\newcommand{\REAL}{\mathbb{R}}
\newcommand{\NATURAL}{\mathbb{N}}
\newcommand{\INTEGER}{\mathbb{Z}}

% for financial stuff
\newcommand{\dollar}{\mathrm{\$}}

% nicer looking trig functions
\newcommand{\SIN}[1]{\sin\left(#1\right)}
\newcommand{\COS}[1]{\cos\left(#1\right)}
\newcommand{\TAN}[1]{\tan\left(#1\right)}
\newcommand{\CSC}[1]{\csc\left(#1\right)}
\newcommand{\SEC}[1]{\sec\left(#1\right)}
\newcommand{\COT}[1]{\cot\left(#1\right)}

% automatically resize set brackets
\newcommand{\SET}[1]{\left\{#1\right\}}

% sums and products
\newcommand{\SUM}{\displaystyle\sum\limits}
\newcommand{\PROD}{\displaystyle\prod\limits}
\newcommand{\of}{\circ}
\newcommand{\restrict}[1]{\raisebox{-.5ex}{$|$}_{#1}}

% set intersection and union
\newcommand{\CAP}{\displaystyle\bigcap\limits}
\newcommand{\CUP}{\displaystyle\bigcup\limits}

% max and min
\newcommand{\MAX}[1]{\ensuremath{\max\left(#1\right)}}
\newcommand{\MIN}[1]{\ensuremath{\min\left(#1\right)}}

% for writing logic within mathematics environment
\newcommand{\FORALL}{\ensuremath{\text{ for all }}}
\newcommand{\FORSOME}{\ensuremath{\text{ for some }}}

% matrix notation
\newcommand{\MATRIX}[2]{\ensuremath{\left[\begin{array}{#1}#2\end{array}\right]}}
\newcommand{\COLUMN}[1]{\ensuremath{\left[\begin{array}{r}#1\end{array}\right]}}

% vector notation
%\newcommand{\vv}{\overset{\rightharpoonup}}
\newcommand{\vv}[1]{{\bf #1}}
\newcommand{\arr}{\overrightarrow}

% dot product
\newcommand{\dotp}{{\scriptstyle\bullet}}

% Text macros
\newcommand{\KER}[1]{\ensuremath{\text{ker}\left(#1\right)}}
\newcommand{\IMG}[1]{\ensuremath{\text{im}\left(#1\right)}}
\newcommand{\CHAR}[1]{\ensuremath{\text{char}\left(#1\right)}}
\newcommand{\BIGO}[1]{\ensuremath{\mathcal{O}\left(#1\right)}}
\newcommand{\TR}[1]{\ensuremath{\text{tr}\left(#1\right)}}

% abbreviations
\newcommand{\ds}{\displaystyle}
\newcommand{\md}{\mdseries}
\newcommand{\vsp}{\vspace{0.5cm}}
\newcommand{\smsp}{\vspace{0.25cm}}
\newcommand{\hsp}{\hspace{0.25cm}}

% new operators
\DeclareMathOperator\SPAN{Span}
\newcommand{\SPANOF}[1]{\ensuremath{\SPAN\left\{#1\right\}}}
\DeclareMathOperator\PROJ{proj}
\DeclareMathOperator\PERP{perp}

% quick abbreviations to avoid using latex environments
\newcommand{\answer}{\noindent \textbf{Answer:} }
\newcommand{\answers}{\noindent \textbf{Answers:} }
\newcommand{\application}{\noindent \textbf{Application:} }
\newcommand{\caution}{\noindent \textbf{Caution:} }
\newcommand{\conclusion}{\noindent \textbf{Conclusion:} }
\newcommand{\consequence}{\noindent \textbf{Consequence:} }
\newcommand{\defn}{\noindent \textbf{Definition:} }
\newcommand{\details}{\noindent \textbf{Details:} }
\newcommand{\example}{\noindent \textbf{Example:} }
\newcommand{\examples}{\noindent \textbf{Examples:} }
\newcommand{\exception}{\noindent \textbf{Exception:} }
\newcommand{\exercise}{\noindent \textbf{Exercise:} }
\newcommand{\exercises}{\noindent \textbf{Exercises:} }
\newcommand{\fact}{\noindent \textbf{Fact:} }
\newcommand{\facts}{\noindent \textbf{Facts:} }
\newcommand{\formula}{\noindent \textbf{Formula:} }
\newcommand{\goal}{\noindent \textbf{Goal:} }
\newcommand{\goals}{\noindent \textbf{Goals:} }
\newcommand{\hint}{\noindent \textbf{Hint:} }
\newcommand{\idea}{\noindent \textbf{Idea:} }
\newcommand{\illustration}{\noindent \textbf{Illustration:} }
\newcommand{\important}{\noindent \textbf{Important:} }
\newcommand{\midea}{\noindent \textbf{Main Idea:} }
\newcommand{\motivation}{\noindent \textbf{Motivation:} }
\newcommand{\nthm}[1]{\noindent \textbf{Theorem} (\textit{#1}):}
\newcommand{\notation}{\noindent \textbf{Notation:} }
\newcommand{\note}{\noindent \textbf{Note:} }
\newcommand{\notes}{\noindent \textbf{Notes:} }
\newcommand{\observation}{\noindent \textbf{Observation:} }
\newcommand{\observations}{\noindent \textbf{Observations:} }
\newcommand{\pict}{\noindent \textbf{Picture:} }
\newcommand{\plan}{\noindent \textbf{Plan:} }
\newcommand{\prf}{\noindent \textbf{Proof:} }
\newcommand{\problem}{\noindent \textbf{Problem:} }
\newcommand{\properties}{\noindent \textbf{Properties:} }
\newcommand{\question}{\noindent \textbf{Question:} }
\newcommand{\questions}{\noindent \textbf{Questions:} }
\newcommand{\recall}{\noindent \textbf{Recall:} }
\newcommand{\reason}{\noindent \textbf{Reason:} }
\newcommand{\remark}{\noindent \textbf{Remark:} }
\newcommand{\remarks}{\noindent \textbf{Remarks:} }
\newcommand{\reminder}{\noindent \textbf{Reminder:} }
\newcommand{\solution}{\noindent \textbf{Solution:} }
\newcommand{\nsolution}[1]{\noindent \textbf{Solution #1:} }
\newcommand{\strategy}{\noindent \textbf{Strategy:} }
\newcommand{\summary}{\noindent \textbf{Summary:} }
\newcommand{\terminology}{\noindent \textbf{Terminology:} }
\newcommand{\thm}{\noindent \textbf{Theorem:} }
\newcommand{\work}{\noindent \textbf{Work:} }


\begin{document} 

% These commands set up the running header on the top of the exam pages
\pagestyle{head}
\firstpageheader{}{}{}
\runningheader{\class}{\examnum\ - Page \thepage\ of \numpages}{\examdate}
\runningheadrule

\begin{flushright}
\begin{tabular}{p{3.8in} r l}
\textbf{\class} & \textbf{Full Name:} & \makebox[2in]{\hrulefill}\\
\textbf{\term} & \textbf{ID:} & \makebox[2in]{\hrulefill}\\
\textbf{\examnum} &&\\
\textbf{\examdate} &&\\
\textbf{Time Limit: \timelimit} &
\end{tabular}\\
\end{flushright}
\rule[1ex]{\textwidth}{.1pt}

\textbf{Reminders:} 
\begin{itemize}

\item \textbf{Organize your work}, in a reasonably neat and coherent way, in the space provided. Work scattered all over the page without a clear ordering will 
receive very little credit.

\item \textbf{Clearly identify your final answer}, by circling it or enclosing it in a box.

\item \textbf{Unsupported answers may not receive full credit}.  A correct answer, unsupported by calculations, explanation,
or algebraic work may receive little or no credit; an incorrect answer supported by some correct calculations will likely receive partial credit.

\item \textbf{Phones should be turned off or in silent mode} -- and they should not be on your desk.  Put them away in a backpack or bag.  The same goes for smart watches.

\item \textbf{You may use a non-programmable scientific calculator.}  All you should have with you at your desk is a pencil (or pen), an eraser, and a calculator.  Beverages are fine, but nothing else should be nearby.  

\item \textbf{No scrap paper or formula sheets are allowed.}  If you need extra space, raise your hand and I will bring you extra paper.

\end{itemize}
\vsp

\addpoints

\noindent
\begin{center}
\gradetablestretch{2}
\vqword{Page}
\gradetable[v][pages]  % Use [pages] to have grading table by page instead of question
\end{center}


\newpage % End of cover page

%%%%%%%%%%%%%%%%%%%%%%%%%%%%%%%%%%%%%%%%%%%%%%%%%%%%%%%%%%%%%%%%%%%%%%%%%%%%%%%%%%%%%
%
% See http://www-math.mit.edu/~psh/#examCls for full documentation, but the questions
% below give an idea of how to write questions [with parts] and have the points
% tracked automatically on the cover page.
%
%
%%%%%%%%%%%%%%%%%%%%%%%%%%%%%%%%%%%%%%%%%%%%%%%%%%%%%%%%%%%%%%%%%%%%%%%%%%%%%%%%%%%%%

\begin{questions}
\question[3] Suppose that \$600 is invested for $7$ years.  Find the accumulated amount if interest is earned at...
\begin{compactenum}[(a)]
<<<<<<< HEAD
\item an annual rate of 7\% compounded weekly.
\vspace{7cm}

\item an annual rate of 2\% compounded daily:
\vspace{7cm}

\item an effective rate of 9.4\%:
\end{compactenum}
\newpage

\question[3] Solve for $x$ in the equation $\log_2(5-x)=7$.
=======
\item an annual rate of 6\% compounded monthly.
\vspace{7cm}

\item an annual rate of 3\% compounded quarterly:
\vspace{7cm}

\item an effective rate of 10.5\%:
\end{compactenum}
\newpage

\question[3] Solve for $y$ in the equation $\log_2(1-y)=3$.
>>>>>>> 4a38f385b8735155eeb6ad3a18e1e75337f3f30e
\vspace{12cm}

\question[3] Solve for $x$ in the equation $4 + 4^{3+x} = 20$.
\newpage

\question[2] If an initial investment of \$500 accumulated to \$529.58 in 13 weeks (with weekly compounding), then what was the annual rate?
\vspace{10cm}

<<<<<<< HEAD
\question[2] How long will it take for an initial principal $P$ to triple at an effective annual rate of $3.7\%$?
\newpage

\question[3] A debt of \$2000 due four years from now and a debt of \$3000 due six years from now are to be paid off by a single payment two years from now.  If interest is collected at an annual rate of $4.8\%$, compounded monthly, find the required payment.
\newpage

\question[3] Some expensive equipment can be purchased with a \$3000 down payment and $26$ monthly payments of \$135.  If interest is collected at 4.1\%, compounded monthly, what is the purchase price of the car?
=======
\question[2] How long will it take for an initial principal $P$ to triple at an effective annual rate of $4.1\%$?
\newpage

\question[3] A debt of \$2000 due four years from now and a debt of \$3000 due six years from now are to be paid off immediately with a single payment.  If interest is collected at an annual rate of $4.8\%$, compounded monthly, find the required payment.
\newpage

\question[3] Some expensive equipment can be purchased with a \$3000 down payment and $26$ monthly payments of \$135.  If interest is collected at 4.1\%, compounded monthly, what is the purchase price of the equipment?
>>>>>>> 4a38f385b8735155eeb6ad3a18e1e75337f3f30e

\textbf{Extra space has been left to ensure that you have enough room to work out this problem.  But do not be alarmed if you do not need all of this space!}
\newpage

\question[3] A family opens up a new savings account in order to start saving up \$20,000 for a down payment on a home, which they plan to purchase five years from now.  If their savings account earns interest at an annual rate of 2.4\% compounded monthly, what amount should they deposit at the end of each month for the next five years in order to accumulate the required \$20,000?

\textbf{Extra space has been left to ensure that you have enough room to work out this problem.  But do not be alarmed if you do not need all of this space!}
\newpage

\question[4] A family purchases a home for \$183,000 with a down payment of \$25,000.  To cover the remaining balance, they obtain a $10$-year mortgage at a 3.6\% annual rate, compounded monthly.
\begin{compactenum}[(a)]
\item What monthly mortgage payment will they need to make?
\vspace{10cm}

\item How much will they pay in interest over the entire mortgage?
\end{compactenum}
\newpage

\question[2] Use the laws of logarithms to completely expand the following:
\begin{center}
$\ds{\log_{2}\left[\frac{(2y+1)^5}{(y-5)^2}\right]}$
\end{center}
\vspace{8cm}

\question[2] Use the laws of logarithms to combine the following into a single logarithm:
\begin{center}
$-\ds{\log_{3}(z - 1) + 2\log_{3}(z + 1)}$
\end{center}
\end{questions}
\end{document}