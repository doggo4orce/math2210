\documentclass[11pt,fleqn,dvipsnames,usenames]{article}

% to keep this file less overwhelming
% packages to include

\usepackage[dvipsnames, table]{xcolor}

\usepackage{
  amsthm,
  amsmath,
  amssymb, 
  arydshln, % for hyphenated lines in block matrices
  fancyhdr, % needed for header at top of each page
  graphicx, % to include pictures
  mathtools, % for a longer arrow
  multicol, % displaying enumerates and itemizes into multiple columns
  multirow, % for tables
  multido, % for TOC
  pgfplots, % for axis environment within tikz pictures
  systeme,
  tikz,
}

\usepackage[utf8]{inputenc}
\usepackage{color,soul}

\usepackage[inline, shortlabels]{enumitem}
\usepackage[hidelinks]{hyperref}


% global constants
\newcommand{\term}{Winter 2024}

% mathbb aliases
\newcommand{\COMPLEX}{\mathbb{C}}
\newcommand{\REAL}{\mathbb{R}}
\newcommand{\NATURAL}{\mathbb{N}}
\newcommand{\INTEGER}{\mathbb{Z}}

% for financial stuff
\newcommand{\dollar}{\mathrm{\$}}

% nicer looking trig functions
\newcommand{\SIN}[1]{\sin\left(#1\right)}
\newcommand{\COS}[1]{\cos\left(#1\right)}
\newcommand{\TAN}[1]{\tan\left(#1\right)}
\newcommand{\CSC}[1]{\csc\left(#1\right)}
\newcommand{\SEC}[1]{\sec\left(#1\right)}
\newcommand{\COT}[1]{\cot\left(#1\right)}

% automatically resize set brackets
\newcommand{\SET}[1]{\left\{#1\right\}}

% sums and products
\newcommand{\SUM}{\displaystyle\sum\limits}
\newcommand{\PROD}{\displaystyle\prod\limits}
\newcommand{\of}{\circ}
\newcommand{\restrict}[1]{\raisebox{-.5ex}{$|$}_{#1}}

% set intersection and union
\newcommand{\CAP}{\displaystyle\bigcap\limits}
\newcommand{\CUP}{\displaystyle\bigcup\limits}

% max and min
\newcommand{\MAX}[1]{\ensuremath{\max\left(#1\right)}}
\newcommand{\MIN}[1]{\ensuremath{\min\left(#1\right)}}

% for writing logic within mathematics environment
\newcommand{\FORALL}{\ensuremath{\text{ for all }}}
\newcommand{\FORSOME}{\ensuremath{\text{ for some }}}

% matrix notation
\newcommand{\MATRIX}[2]{\ensuremath{\left[\begin{array}{#1}#2\end{array}\right]}}
\newcommand{\COLUMN}[1]{\ensuremath{\left[\begin{array}{r}#1\end{array}\right]}}

% vector notation
%\newcommand{\vv}{\overset{\rightharpoonup}}
\newcommand{\vv}[1]{{\bf #1}}
\newcommand{\arr}{\overrightarrow}

% dot product
\newcommand{\dotp}{{\scriptstyle\bullet}}

% Text macros
\newcommand{\KER}[1]{\ensuremath{\text{ker}\left(#1\right)}}
\newcommand{\IMG}[1]{\ensuremath{\text{im}\left(#1\right)}}
\newcommand{\CHAR}[1]{\ensuremath{\text{char}\left(#1\right)}}
\newcommand{\BIGO}[1]{\ensuremath{\mathcal{O}\left(#1\right)}}
\newcommand{\TR}[1]{\ensuremath{\text{tr}\left(#1\right)}}

% abbreviations
\newcommand{\ds}{\displaystyle}
\newcommand{\md}{\mdseries}
\newcommand{\vsp}{\vspace{0.5cm}}
\newcommand{\smsp}{\vspace{0.25cm}}
\newcommand{\hsp}{\hspace{0.25cm}}

% new operators
\DeclareMathOperator\SPAN{Span}
\newcommand{\SPANOF}[1]{\ensuremath{\SPAN\left\{#1\right\}}}
\DeclareMathOperator\PROJ{proj}
\DeclareMathOperator\PERP{perp}

% quick abbreviations to avoid using latex environments
\newcommand{\answer}{\noindent \textbf{Answer:} }
\newcommand{\answers}{\noindent \textbf{Answers:} }
\newcommand{\application}{\noindent \textbf{Application:} }
\newcommand{\caution}{\noindent \textbf{Caution:} }
\newcommand{\conclusion}{\noindent \textbf{Conclusion:} }
\newcommand{\consequence}{\noindent \textbf{Consequence:} }
\newcommand{\defn}{\noindent \textbf{Definition:} }
\newcommand{\details}{\noindent \textbf{Details:} }
\newcommand{\example}{\noindent \textbf{Example:} }
\newcommand{\examples}{\noindent \textbf{Examples:} }
\newcommand{\exception}{\noindent \textbf{Exception:} }
\newcommand{\exercise}{\noindent \textbf{Exercise:} }
\newcommand{\exercises}{\noindent \textbf{Exercises:} }
\newcommand{\fact}{\noindent \textbf{Fact:} }
\newcommand{\facts}{\noindent \textbf{Facts:} }
\newcommand{\formula}{\noindent \textbf{Formula:} }
\newcommand{\goal}{\noindent \textbf{Goal:} }
\newcommand{\goals}{\noindent \textbf{Goals:} }
\newcommand{\hint}{\noindent \textbf{Hint:} }
\newcommand{\idea}{\noindent \textbf{Idea:} }
\newcommand{\illustration}{\noindent \textbf{Illustration:} }
\newcommand{\important}{\noindent \textbf{Important:} }
\newcommand{\midea}{\noindent \textbf{Main Idea:} }
\newcommand{\motivation}{\noindent \textbf{Motivation:} }
\newcommand{\nthm}[1]{\noindent \textbf{Theorem} (\textit{#1}):}
\newcommand{\notation}{\noindent \textbf{Notation:} }
\newcommand{\note}{\noindent \textbf{Note:} }
\newcommand{\notes}{\noindent \textbf{Notes:} }
\newcommand{\observation}{\noindent \textbf{Observation:} }
\newcommand{\observations}{\noindent \textbf{Observations:} }
\newcommand{\pict}{\noindent \textbf{Picture:} }
\newcommand{\plan}{\noindent \textbf{Plan:} }
\newcommand{\prf}{\noindent \textbf{Proof:} }
\newcommand{\problem}{\noindent \textbf{Problem:} }
\newcommand{\properties}{\noindent \textbf{Properties:} }
\newcommand{\question}{\noindent \textbf{Question:} }
\newcommand{\questions}{\noindent \textbf{Questions:} }
\newcommand{\recall}{\noindent \textbf{Recall:} }
\newcommand{\reason}{\noindent \textbf{Reason:} }
\newcommand{\remark}{\noindent \textbf{Remark:} }
\newcommand{\remarks}{\noindent \textbf{Remarks:} }
\newcommand{\reminder}{\noindent \textbf{Reminder:} }
\newcommand{\solution}{\noindent \textbf{Solution:} }
\newcommand{\nsolution}[1]{\noindent \textbf{Solution #1:} }
\newcommand{\strategy}{\noindent \textbf{Strategy:} }
\newcommand{\summary}{\noindent \textbf{Summary:} }
\newcommand{\terminology}{\noindent \textbf{Terminology:} }
\newcommand{\thm}{\noindent \textbf{Theorem:} }
\newcommand{\work}{\noindent \textbf{Work:} }


\usepackage[version=4]{mhchem}

% Where to look for pngs and jpegs
\graphicspath{{Images//}}

\usepackage[includehead, includefoot, left= 2cm, top =1.5cm, bottom = 1.5cm, textwidth=17.5cm]{geometry}

\usepackage{pifont, amsmath}

\pagestyle{fancy}
\fancyhf{}
\renewcommand{\headrulewidth}{1pt}
%\fancyhead[R]{\bfseries\sffamily\thepage}
%\fancyfoot[C]{\bfseries\sffamily\thepage}
\fancyhead[L]{\nouppercase{\bfseries\sffamily\leftmark}}

% used when adding fill-in-the-blanks for students
\newcommand{\blank}[1]{\underline{\hspace{#1}}}

% indents annoy me, and so does repeatedly typing \noindent
\newcommand{\p}{\noindent}

\newcommand{\ENDPRF}{\hfill $\blacksquare$}

\begin{document}

\fancyhead[L]{Math 2210}
\fancyhead[C]{\includegraphics[width=5cm, trim= 0 0.4cm 0 0]{TRU_logo}}
\fancyhead[R]{\term}
\renewcommand{\headrulewidth}{0.4pt}

\setulcolor{red}

\setcounter{section}{3}
\section{Arithmetic in $R[x]$}
\setcounter{subsection}{0}


\subsection{Polynomial Rings}

\defn Let $R$ be a ring.  A \DEF{polynomial (in the indeterminate $x$) with coefficients in $R$} is an infinite formal sum of the form
\begin{center}
$f(x) = \SUM_{j=0}^{\infty}a_{j}x^{j} = a_{0} + a_{1}x + a_{2}x^{2} + \ldots$ with $a_{j}\in R$
\end{center}
for each $j\geq 0$, satisfying the property that the set $\SET{j\geq 0:a_{j}\neq 0}$ is finite.
\vsp

\remark Note that $x$ is a formal object with (so far) no inherent mathematical properties, so the only assumption we ought to place upon its ``powers'' $x^{2}, x^{3}, x^{4}$, etc. is that they are all distinct.
\vsp

\terminology In this case,
\begin{itemize}
\item $a_{0}$, $a_{1}x$, $a_{2}x^{2}, \ldots$ are the \DEF{terms} of $f(x)$.
\item $a_{0}, a_{1}, a_{2}, \ldots \in R$ are the \DEF{coefficients} of $f(x)$.
\end{itemize}
\vsp

\remarks
\begin{itemize}
\item Terms with a coefficient of $0$ are omitted.  For example, the polynomial
\begin{center}
$\SUM_{j=0}^{\infty}a_{j}x^{j}$ with $a_{j} = \begin{cases}j & \text{if } j = 2, 3,\text{ or }5\\0 & \text{otherwise}\end{cases}$
\end{center}
\p is written as
\begin{center}
$2x^2 + 3x^3 + 5x^{5}$
\end{center}
\p instead of
\begin{center}
$0 + 0x + 2x^2 + 3x^3 + 0x^4 + 5x^5 + 0x^6 + 0x^7 + ...$
\end{center}
\vsp

\item The order in which the terms of a polynomial are written is unimportant.  That is, $1 - x + x^2$ may also be written as $-x + x^2 + 1$.
\end{itemize}
\vsp

\terminology If $a_{j} = 0$ for all $j > 0$, then $f(x) = a_{0}$ is a \DEF{constant polynomial}.
\vsp

\p The coefficient of a term preceded by the subtraction symbol is interpreted as its additive inverse.
\vsp

\examples
\begin{enumerate}[(a)]
\item $2x^3 + 4x^4 = 2x^3 - 3x^4$ in $\INTEGER_{7}[x]$ since $-3 = 4$ in $\INTEGER_{7}$.
\item $x + 2x^2 = -2 + 2x^2$ in $\INTEGER_{3}[x]$, since $-2 = 1$ in $\INTEGER_{3}$.
\item $1 + x^2 + x^5 = -1 - x^2 - x^5$ in $\INTEGER_{2}[x]$, since $-1 = 1$ in $\INTEGER_{2}$.
\end{enumerate}
\vsp

% for what value of n\geq 1 is f(x) = g(x)?  (Additive inverses)
\newpage

\thm $R[x]$, the set of all polynomials in the indeterminate $x$ with coefficients in $R$, is a ring under addition $\oplus$ and multiplication $\odot$ defined by
\begin{itemize}
\item $\left(\SUM_{j=0}^{\infty}a_{j}x^{j}\right)\oplus\left(\SUM_{j=0}^{\infty}b_{j}x^{j}\right) =\left(\SUM_{j=0}^{\infty}c_{j}x^{j}\right)$ where $c_{j} = a_{j} + b_{j}$ for each $j\geq 0$, and
\item $\left(\SUM_{j=0}^{\infty}a_{j}x^{j}\right)\odot\left(\SUM_{j=0}^{\infty}b_{j}x^{j}\right) =\left(\SUM_{j=0}^{\infty}c_{j}x^{j}\right)$ where
$c_{j} = \SUM_{k=0}^{j}a_{k}b_{j-k}$ for each $j\geq 0$.
\end{itemize}
\vsp

\example Compute the sum and product of $f(x) = 1 - 2x + x^2$ and $g(x) = 1 + x$ in $\INTEGER[x]$.
\vsp

\solution Write $f(x) = \SUM_{j=0}^{\infty}a_{j}x^{j}$ and $g(x) = \SUM_{j=0}^{\infty}b_{j}x^{j}$, where
\begin{center}
$a_{j} = \begin{cases}\phantom{-}1 & \text{if } j=0\text{ or }2\\-2 & \text{if } j = 1\\\phantom{-}0 & \text{otherwise}\end{cases}$ and 
$b_{j} = \begin{cases}1 & \text{if } j=0\text{ or }j=1\\0 & \text{otherwise}\end{cases}$
\end{center}
By definition, $f(x)\oplus g(x) = \SUM_{j=0}^{\infty}c_{j}x^{j}$, where
\begin{center}
$c_{0} = a_{0} + b_{0} = 1 + 1 = 2$,\hspace{1cm} $c_{1} = a_{1} + b_{1} = -2 + 1 = -1$,\hspace{1cm} $c_{2} = a_{2} + b_{2} = 1 + 0 = 1$,
\end{center}
and $c_{j} = 0$ for all $j\geq 3$.  Hence $f(x) \oplus g(x) = 2 - x + x^2$.  In a similar fashion, $f(x)\odot g(x) = \SUM_{j=0}^{\infty}c_{j}x^{j}$, where
\begin{align*}
c_{0} &= \SUM_{j=0}^{0}a_{j}b_{0-j} = a_{0}b_{0} = 1\cdot 1 = 1,\\
c_{1} &= \SUM_{j=0}^{1}a_{j}b_{1-j} = a_{0}b_{1} + a_{1}b_{0} = 1\cdot 1 + (-2)\cdot 1 = -1\\
c_{2} &= \SUM_{j=0}^{2}a_{j}b_{2-j} = a_{0}b_{2} + a_{1}b_{1} + a_{2}b_{0} = 1\cdot 0 + (-2)\cdot 1 + 1\cdot 1 = -1\\
c_{3} &= \SUM_{j=0}^{3}a_{j}b_{3-j} = a_{0}b_{3} + a_{1}b_{2} + a_{2}b_{1} + a_{3}b_{0} = 1\cdot 0 + (-2)\cdot 0 + 1\cdot 1 + 0\cdot 1 = 1\text{,}
\end{align*}
and $c_{j} = 0$ for all $j\geq 4$.  Hence $f(x) \odot g(x) = 1 - x - x^2 + x^3$.
\vsp

\p Note that addition in $R[x]$ is equivalent to collecting ``like terms'', and the following properties can be used to illustrate that multiplication is no different than what we often casually refer to as ``FOIL''.
\newpage

\exercise Prove the following properties for any $a,b\in R$.
\begin{enumerate}[(a)]
\item $a\odot x = x\odot a = ax$ for all $a\in R$.
\item $a\odot bx = ax\odot b = (ab)x$ for all $a,b\in R$.
\item $a\odot x^{n} = x^{n}\odot a = ax^{n}$ for all $a\in R$ and $n\geq 2$.
\item $a\odot bx^{n} = ax^{n}\odot b = (ab)x^{n}$ for all $a,b\in R$ and $n\geq 2$.
\item $x^{m}\odot x^{n} = x^{n}\odot x^{m} = x^{m+n}$ for all $m,n\geq 1$.
\item $ax^{m}\odot bx^{n} = (ab)x^{m+n}$ for all $a,b\in R$ and $m,n\geq 0$.
\end{enumerate}
\vsp

\example
\begin{align*}
(2x^3 + 3x^5)\odot (4x^2 + 3x^7) &= (2x^3 \oplus 3x^5)\odot (4x^2 \oplus 3x^7)\\
&= \left[(2x^3 \oplus 3x^5)\odot 4x^2\right] \oplus \left[(2x^3 \oplus 3x^5) \odot 3x^7\right]\\
&= \left[(2x^3 \odot 4x^2) \oplus (3x^5 \odot 4x^2)\right] \oplus \left[(2x^3 \odot 3x^7) \oplus (3x^5\odot 3x^7)\right]\\
&= \left[8x^5 \oplus 12x^7\right] \oplus \left[6x^{10} \oplus 9x^{12}\right]\\
&= 8x^5 + 12x^7 + 6x^{10} + 9x^{12}
\end{align*}

\remark Up until this point, care has been taken to represent addition in $R[x]$ using the symbol $\oplus$, reserving $+$ to separate terms within a polynomials itself.  But this distinction is unnecessary, since the sum

\begin{center}
$2x^3 \oplus 3x^5$
\end{center}

\p is equal to the polynomial

\begin{center}
$2x^3 + 3x^5$.
\end{center}

\p Addition and multiplication of polynomials $f(x)$ and $g(x)$ will be henceforth written as $f(x) + g(x)$ and $f(x)g(x)$, respectively.
\vsp

\defn The \DEF{degree} of a polynomial $f(x) = \SUM_{j=0}^{\infty}a_{j}x^{j}$, denoted $\DEG{f(x)}$, is defined by
\begin{center}
 $\DEG{f(x)} = \max \SET{j\geq 0:a_{j} \neq 0}$,
 \end{center}
whenever this set is non-empty.  Otherwise $a_{j} = 0$ for every $j\geq 0$, then $\DEG{f(x)}$ is undefined.
\vsp

\examples
\begin{multicols}{2}
\begin{enumerate}[(a)]
\item $\DEG{x^3 + 4x^7} = 7$
\item $\DEG{3} = 0$
\item $\DEG{x} = 1$
\item $\DEG{0}$ is undefined.
\end{enumerate}
\end{multicols}
\vsp

\exercise Show that $\DEG{f(x)g(x)}\leq \DEG{f(x)}\DEG{g(x)}$ for all $f(x),g(x)\in R[x]$, with equality if $R$ is an integral domain.
\vsp

\exercise If $R$ is unital, then so is $R[x]$ with identity element equal to the constant polynomial $1_{R}$.

\end{document}

