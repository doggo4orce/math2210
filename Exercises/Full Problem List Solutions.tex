\documentclass[11pt,fleqn,dvipsnames,usenames]{article}

% to keep this file less overwhelming
% packages to include

\usepackage[dvipsnames, table]{xcolor}

\usepackage{
  amsthm,
  amsmath,
  amssymb, 
  arydshln, % for hyphenated lines in block matrices
  fancyhdr, % needed for header at top of each page
  graphicx, % to include pictures
  mathtools, % for a longer arrow
  multicol, % displaying enumerates and itemizes into multiple columns
  multirow, % for tables
  multido, % for TOC
  pgfplots, % for axis environment within tikz pictures
  systeme,
  tikz,
}

\usepackage[utf8]{inputenc}
\usepackage{color,soul}

\usepackage[inline, shortlabels]{enumitem}
\usepackage[hidelinks]{hyperref}


% global constants
\newcommand{\term}{Winter 2024}

% mathbb aliases
\newcommand{\COMPLEX}{\mathbb{C}}
\newcommand{\REAL}{\mathbb{R}}
\newcommand{\NATURAL}{\mathbb{N}}
\newcommand{\INTEGER}{\mathbb{Z}}

% for financial stuff
\newcommand{\dollar}{\mathrm{\$}}

% nicer looking trig functions
\newcommand{\SIN}[1]{\sin\left(#1\right)}
\newcommand{\COS}[1]{\cos\left(#1\right)}
\newcommand{\TAN}[1]{\tan\left(#1\right)}
\newcommand{\CSC}[1]{\csc\left(#1\right)}
\newcommand{\SEC}[1]{\sec\left(#1\right)}
\newcommand{\COT}[1]{\cot\left(#1\right)}

% automatically resize set brackets
\newcommand{\SET}[1]{\left\{#1\right\}}

% sums and products
\newcommand{\SUM}{\displaystyle\sum\limits}
\newcommand{\PROD}{\displaystyle\prod\limits}
\newcommand{\of}{\circ}
\newcommand{\restrict}[1]{\raisebox{-.5ex}{$|$}_{#1}}

% set intersection and union
\newcommand{\CAP}{\displaystyle\bigcap\limits}
\newcommand{\CUP}{\displaystyle\bigcup\limits}

% max and min
\newcommand{\MAX}[1]{\ensuremath{\max\left(#1\right)}}
\newcommand{\MIN}[1]{\ensuremath{\min\left(#1\right)}}

% for writing logic within mathematics environment
\newcommand{\FORALL}{\ensuremath{\text{ for all }}}
\newcommand{\FORSOME}{\ensuremath{\text{ for some }}}

% matrix notation
\newcommand{\MATRIX}[2]{\ensuremath{\left[\begin{array}{#1}#2\end{array}\right]}}
\newcommand{\COLUMN}[1]{\ensuremath{\left[\begin{array}{r}#1\end{array}\right]}}

% vector notation
%\newcommand{\vv}{\overset{\rightharpoonup}}
\newcommand{\vv}[1]{{\bf #1}}
\newcommand{\arr}{\overrightarrow}

% dot product
\newcommand{\dotp}{{\scriptstyle\bullet}}

% Text macros
\newcommand{\KER}[1]{\ensuremath{\text{ker}\left(#1\right)}}
\newcommand{\IMG}[1]{\ensuremath{\text{im}\left(#1\right)}}
\newcommand{\CHAR}[1]{\ensuremath{\text{char}\left(#1\right)}}
\newcommand{\BIGO}[1]{\ensuremath{\mathcal{O}\left(#1\right)}}
\newcommand{\TR}[1]{\ensuremath{\text{tr}\left(#1\right)}}

% abbreviations
\newcommand{\ds}{\displaystyle}
\newcommand{\md}{\mdseries}
\newcommand{\vsp}{\vspace{0.5cm}}
\newcommand{\smsp}{\vspace{0.25cm}}
\newcommand{\hsp}{\hspace{0.25cm}}

% new operators
\DeclareMathOperator\SPAN{Span}
\newcommand{\SPANOF}[1]{\ensuremath{\SPAN\left\{#1\right\}}}
\DeclareMathOperator\PROJ{proj}
\DeclareMathOperator\PERP{perp}

% quick abbreviations to avoid using latex environments
\newcommand{\answer}{\noindent \textbf{Answer:} }
\newcommand{\answers}{\noindent \textbf{Answers:} }
\newcommand{\application}{\noindent \textbf{Application:} }
\newcommand{\caution}{\noindent \textbf{Caution:} }
\newcommand{\conclusion}{\noindent \textbf{Conclusion:} }
\newcommand{\consequence}{\noindent \textbf{Consequence:} }
\newcommand{\defn}{\noindent \textbf{Definition:} }
\newcommand{\details}{\noindent \textbf{Details:} }
\newcommand{\example}{\noindent \textbf{Example:} }
\newcommand{\examples}{\noindent \textbf{Examples:} }
\newcommand{\exception}{\noindent \textbf{Exception:} }
\newcommand{\exercise}{\noindent \textbf{Exercise:} }
\newcommand{\exercises}{\noindent \textbf{Exercises:} }
\newcommand{\fact}{\noindent \textbf{Fact:} }
\newcommand{\facts}{\noindent \textbf{Facts:} }
\newcommand{\formula}{\noindent \textbf{Formula:} }
\newcommand{\goal}{\noindent \textbf{Goal:} }
\newcommand{\goals}{\noindent \textbf{Goals:} }
\newcommand{\hint}{\noindent \textbf{Hint:} }
\newcommand{\idea}{\noindent \textbf{Idea:} }
\newcommand{\illustration}{\noindent \textbf{Illustration:} }
\newcommand{\important}{\noindent \textbf{Important:} }
\newcommand{\midea}{\noindent \textbf{Main Idea:} }
\newcommand{\motivation}{\noindent \textbf{Motivation:} }
\newcommand{\nthm}[1]{\noindent \textbf{Theorem} (\textit{#1}):}
\newcommand{\notation}{\noindent \textbf{Notation:} }
\newcommand{\note}{\noindent \textbf{Note:} }
\newcommand{\notes}{\noindent \textbf{Notes:} }
\newcommand{\observation}{\noindent \textbf{Observation:} }
\newcommand{\observations}{\noindent \textbf{Observations:} }
\newcommand{\pict}{\noindent \textbf{Picture:} }
\newcommand{\plan}{\noindent \textbf{Plan:} }
\newcommand{\prf}{\noindent \textbf{Proof:} }
\newcommand{\problem}{\noindent \textbf{Problem:} }
\newcommand{\properties}{\noindent \textbf{Properties:} }
\newcommand{\question}{\noindent \textbf{Question:} }
\newcommand{\questions}{\noindent \textbf{Questions:} }
\newcommand{\recall}{\noindent \textbf{Recall:} }
\newcommand{\reason}{\noindent \textbf{Reason:} }
\newcommand{\remark}{\noindent \textbf{Remark:} }
\newcommand{\remarks}{\noindent \textbf{Remarks:} }
\newcommand{\reminder}{\noindent \textbf{Reminder:} }
\newcommand{\solution}{\noindent \textbf{Solution:} }
\newcommand{\nsolution}[1]{\noindent \textbf{Solution #1:} }
\newcommand{\strategy}{\noindent \textbf{Strategy:} }
\newcommand{\summary}{\noindent \textbf{Summary:} }
\newcommand{\terminology}{\noindent \textbf{Terminology:} }
\newcommand{\thm}{\noindent \textbf{Theorem:} }
\newcommand{\work}{\noindent \textbf{Work:} }


\usepackage[version=4]{mhchem}

% Where to look for pngs and jpegs
\graphicspath{{Images//}}

\usepackage[includehead, includefoot, left= 2cm, top =1.5cm, bottom = 1.5cm, textwidth=17.5cm]{geometry}

\usepackage{pifont, amsmath}

\pagestyle{fancy}
\fancyhf{}
\renewcommand{\headrulewidth}{1pt}
%\fancyhead[R]{\bfseries\sffamily\thepage}
%\fancyfoot[C]{\bfseries\sffamily\thepage}
\fancyhead[L]{\nouppercase{\bfseries\sffamily\leftmark}}

% used when adding fill-in-the-blanks for students
\newcommand{\blank}[1]{\underline{\hspace{#1}}}

% indents annoy me, and so does repeatedly typing \noindent
\newcommand{\p}{\noindent}

\begin{document}

\fancyhead[L]{\course}
\fancyhead[C]{\includegraphics[width=5cm, trim= 0 0.4cm 0 0]{TRU_logo}}
\fancyhead[R]{\term}
\renewcommand{\headrulewidth}{0.4pt}

{\huge Chapter 1}
\vsp

\begin{enumerate}
\item Suppose $a,b,c\in\Z$ such that $a|b$ and $b|c$.  Prove that $a|c$.
\vsp

\solution If $a|b$ and $b|c$, there exists $s,t\in\Z$ such that $b = sa$ and $c = tb$.  It follows that
\begin{center}
$c = tb = t(sa) = (ts)a$,
\end{center}
and hence $a|c$.

\item Suppose that $a,b,c\in\Z$ are chosen so that $b|ac$.  Is it necessarily the case that either $b|a$ or $b|c$?
\vsp

\solution No.  For example, $6|2\cdot 3$ but $6\ndiv 2$ and $6\ndiv 3$.

\item If $b|a+c$ for some $a,b,c\in \Z$, does it automatically follow that either $b|a$ or $b|c$?
\vsp

\solution No.  For example, $2$ does not divide either of $3$ or $-3$, but it does divide $3 + (-3) = 0$.
\vsp

\item Prove each of the following statements.
\begin{enumerate}[(a)]
\item If $b|a$, then $b|sa$ for any $s\in\INTEGER$.
\item If $b|a$ and $b|c$, then $b|sa+tc$ for any $s,t\in\INTEGER$.
\end{enumerate}
\vsp

% \solution
% \begin{enumerate}[(a)]
% \item If $b|a$, then $a = tb$ for some $t\in\Z$.  For any $s\in\Z$, $sa = s(tb) = (st)b$ and hence $b|sa$.
% \item If $b|a$ and $b|c$, then $a = kb$ and $c = mb$ for some $k,m\in\Z$.  Then
% \begin{center}
% $sa + tc = s(kb) + t(mb) = (sk + tm)b$
% \end{center}
% and hence $b|sa + tc$.
% \end{enumerate}

\item Suppose $a,b,c,d\in\Z$ such that $a|c$ and $b|d$.  Prove that $ab|cd$.
\vsmsp

\solution Suppose $a|c$ and $b|d$, and write $c = sa$ and $d = tb$ for some $s,t\in\Z$.  Then $ab|cd$, since $cd = (sa)(tb) = (st)(ab)$.

\item Prove that the square of any integer is either of the form $4k$ or $4k+1$ for some integer $k$.
\vsmsp

\solution For any $n\in\Z$ we may choose, by the division algorithm, unique $q,r\in\Z$ such that
\begin{center}
$n = 2q + r$ and $0\leq r < 2$.
\end{center}

\p It follows that
\begin{center}
$n^2 = (2q+r)^2 = 4q^2 + 4qr + r^2 = 4(q^2 + qr) + r^2$.
\end{center}
By the way $r$ was chosen, we must have $r=0$ or $r=1$.
\vsp

If $r = 0$,
\begin{center}
$n^2 = 4(q^2 + qr) + 0^2 = 4(q^2 + qr)$.
\end{center}
If $r = 1$, then
\begin{center}
$n^2 = 4(q^2 + qr) + 1^2 = 4(q^2 + qr) + 1$.
\end{center}
In either case the result follows.

\item Prove that the cube of any integer $a$ is of the form $9k, 9k+1$, or $9k+8$ for some $k\in\INTEGER$.
\vsmsp

\solution Let $a\in\Z$.  By the division algorithm, there exists $q,r\in\Z$ such that
\begin{center}
$a = 3q + r$ and $0\leq r < 3$.
\end{center}
\p If $r = 0$, then $a^3 = (3q)^3 = 27q^3 = 9(3q^3)$.
\vsmsp

\p If $r = 1$, then $a^3 = (3q + 1)^3 = 27q^3 + 27q^2 + 9q = 9(3q^3 + 3q^2 + q) + 1$.
\vsmsp

\p If $r = 2$, then $a^3 = (3q + 2)^3 = 27q^3 + 54q^2 + 36q + 8 = 9(3q^3 + 6q^2 + 4q) + 8$.
\vsmsp

\p Since there are no other possible values of $r$, the result follows.

\item If $a|c$ and $b|c$ with $\GCD{a,b} = 1$, prove that $ab|c$.\label{abdividesc}
\vsmsp

\solution Suppose $a|c$, $b|c$, and that $\GCD{a,b} = 1$.  Since $a|c$ and $b|c$, we must have $ab|cb$ and $ab|ca$.  By Bezout's identity we may choose $s,t\in\Z$ such that $1 = sa + tb$.  It follows that
\begin{center}
$c = c\cdot 1 = c(sa + tb) = s(ca) + t(cb)$,
\end{center}
which is divisible by $ab$ since $ca$ and $tb$ are.

\item Show that the statement in Problem \ref{abdividesc} fails to hold in general if $\GCD{a,b}\neq 1$.
\vsmsp

\solution Take, for a counterexample, the fact that both $4$ and $6$ divide $12$, but $4\cdot 6 = 24$ does not.

\item Calculate $\GCD{229,66}$ and find $s,t\in\Z$ such that $\GCD{229,66} = 229s + 66t$.
\item If $a,b\in\Z$ are not both zero and $c\in\Z^{+}$, prove that $\GCD{ca,cb} = c\cdot \GCD{a,b}$.
\vsmsp

\solution Since $\GCD{a,b}$ divides both $a$ and $b$, $c\cdot \GCD{a,b}$ divides both $ca$ and $cb$.
Next, note that by Bezout's identity, there exists $s,t\in\Z$ such that $\GCD{a,b} = sa + tb$.  It follows that
\begin{center}
$c\cdot \GCD{a,b} = c(sa + tb) = s(ca) + t(cb)$.
\end{center}
Hence any common divisor of $ca$ and $cb$ must also be a divisor of $c\cdot\GCD{a,b}$.

\item If $n\in \Z$, what are the possible values of $\GCD{n, n+1}$?
\vsmsp

\solution Any common divisor of $n$ and $n+1$ must also be a divisor of $n+1 - n = 1$.  Hence $1$ and $-1$ are the only common divisors of $n$ and $n+1$.  Since $\GCD{n,n+1}$ is positive by definition, it must be equal to $1$.

\item Let $a,b\in\Z$ be non-zero.  Show that if $k\in\INTEGER$ such that $a|k$ and $b|k$, then $\LCM{a, b}|k$.
%\vsmsp

%\solution Let $k\in\Z$ be a common multiple of both $a$ and $b$.  By assumption, $\LCM{a,b}\leq k$.  If $\LCM{a,b} = k$, the result follows immediately, so assume that $k > \LCM{a,b}$.  By the division algorithm, choose $q,r\in\Z$ such that
%\begin{center}
%$k = q\cdot \LCM{a,b} + r$ with $0\leq r < \LCM{a,b}$.
%\end{center}
%\p If $r > 0$, then $r = k - q\cdot \LCM{a,b}$ is a positive common multiple of $a$ and $b$ (since both $k$ and $\LCM{a,b}$ are) which is smaller than $\LCM{a,b}$.  This is a contradiction and hence $r$ must be zero, forcing $\LCM{a,b}|k$ as required.

\item Suppose $a,b\in\Z$ such that $a^2|b^2$.  Prove that $a|b$.
\vsmsp

\solution Using the Fundamental Theorem of Arithmetic, we may choose distinct primes $p_{1},p_{2},\ldots, p_{k}$ and non-negative integers $s_{1},t_{1},s_{2},t_{2},\ldots,s_{k},t_{k}$ so that
\begin{center}
$a = p_{1}^{s_{1}}p_{2}^{s_{2}}\cdots p_{k}^{s_{k}}$ and $b = p_{1}^{t_{1}}t_{2}^{t_{2}}\cdots p_{k}^{t_{k}}$.
\end{center}
\p It follows that
\begin{center}
$a^2 = \left(p_{1}^{s_{1}}p_{2}^{s_{2}}\cdots p_{k}^{s_{k}}\right)^2 = p_{1}^{2s_{1}}p_{2}^{2s_{2}}\cdots p_{k}^{2s_{k}}$ and $b^2 = \left(p_{1}^{t_{1}}p_{2}^{t_{2}}\cdots p_{k}^{t_{k}}\right)^2 = p_{1}^{2t_{1}}p_{2}^{2t_{2}}\cdots p_{k}^{2t_{k}}$.
\end{center}
\p If $a^2|b^2$, we must have $2t_{j}\leq 2s_{j}$ for each $j=1,2,\ldots, k$.  This forces $t_{j}\leq s_{j}$ for each $j=1,2,\ldots, k$ and hence $a|b$.

\item Prove that $\sqrt[3]{2}\notin\Q$.
\vsmsp

\solution Suppose for a contradiction that $\sqrt[3]{2} = \frac{m}{n}$ for some $m,n\in\Z$ with $\GCD{m,n}=1$.  Then $2 = \frac{m^3}{n^3}$, or equivalently $2n^3 = m^3$.  This implies $2|m^3$, and forces $2|m$ since $2$ is prime.  Hence $8|m^3$ and it follows that $4|n^3$.  In particular, $2|n^3$ and thus $2|n$.  It follows that $2$ is a common factor of $m$ and $n$, which is a contradiction.

\end{enumerate}
\vsp

{\huge Chapter 2}
\vsp

\begin{enumerate}
\item Write $\ds{\frac{2 + 3i}{1 + 2i} + \frac{8+i}{2-i}}$ in the form $a+bi$ for some $a,b\in\R$.
\item Prove the following properties hold in $\C$.  Assume that the corresponding properties hold in $\R$.
\begin{enumerate}[(a)]
\item $w + z = z + w$ for all $w,z\in\C$.\hfill (commutativity of addition)
\item $w + (z + \lambda) = (w + z) + \lambda$ for all $w,z,\lambda\in\C$.\hfill (associativity of addition)
\item $wz = zw$ for all $w,z\in\C$.\hfill (commutativity of multiplication)
\item $w(z\lambda) = (wz)\lambda$ for all $w,z,\lambda\in\C$.\hfill (associativity of multiplication)
\item $w(z + \lambda) = wz + w\lambda$ for all $w,z,\lambda\in\C$.\hfill (distributivity)
\end{enumerate}
\vsmsp

\solution 
\begin{enumerate}[(a)]
\item Write $w = (a_{1},b_{1})$ and $z = (a_{2},b_{2})$ for some $a_{1},a_{2},b_{1},b_{2}\in\R$.  Then
\begin{center}
$w + z = (a_{1},b_{1}) + (a_{2},b_{2}) = (a_{1} + a_{2}, b_{1} + b_{2}) = (a_{2} + a_{1}, b_{2} + b_{1}) = (a_{2},b_{2}) + (a_{1},b_{1}) = z + w$.
\end{center}

\item Write $w = (a_{1},b_{1}), z = (a_{2},b_{2})$, and $\lambda = (a_{3},b_{3})$ for some $a_{1},a_{2},a_{3},b_{1},b_{2},b_{3}\in\R$.  Then
\begin{align*}
w + (z + \lambda) &= (a_{1},b_{1}) + \big((a_{2},b_{2}) + (a_{3},b_{3})\big)\\
&= (a_{1},b_{1}) + (a_{2} + a_{3},b_{2}+b_{3})\\
&= \big(a_{1} + (a_{2} + a_{3}), b_{1} + (b_{2}+b_{3})\big)\\
&= \big((a_{1} + a_{2}) + a_{3}, (b_{1} + b_{2})+b_{3}\big)\\
&= (a_{1} + a_{2}, b_{1} + b_{2}) + (a_{3},b_{3})\\
&= (w + z) + \lambda.
\end{align*}

\item Write $w = (a_{1},b_{1})$ and $z = (a_{2},b_{2})$ for some $a_{1},a_{2},b_{1},b_{2}\in\R$.  Then
\begin{align*}
wz &= (a_{1},b_{1})\cdot (a_{2},b_{2})\\
&= (a_{1}a_{2} - b_{1}b_{2},a_{1}b_{2} + b_{1}a_{2})\\
&= (a_{2}a_{1} - b_{2}b_{1},a_{2}b_{1} + b_{2}a_{1})\\
&= (a_{2},b_{2})\cdot (a_{1},b_{1})\\
&= zw
\end{align*}

%\item \label{associativityofmult} Write $w = (a_{1},b_{1}), z = (a_{2},b_{2})$, and $\lambda = (a_{3},b_{3})$ for some $a_{1},a_{2},a_{3},b_{1},b_{2},b_{3}\in\R$.  In this case it is easier to work out the left and right hand sides separately, and check that they yield the same result:
%\vsmsp

%First we work out the left hand side:
%\begin{align*}
%w(z\lambda) &= (a_{1},b_{1})\cdot\big((a_{2},b_{2})\cdot (a_{3},b_{3})\big)\\
%&= (a_{1},b_{1})\cdot (a_{2}a_{3} - b_{2}b_{3},a_{2}b_{3} + b_{2}a_{3})\\
%&= \big(a_{1}(a_{2}a_{3}-b_{2}b_{3}) - b_{1}(a_{2}b_{3} + b_{2}a_{3}), a_{1}(a_{2}b_{3} + b_{2}a_{3}) + b_{1}(a_{2}a_{3} - b_{2}b_{3})\big)\\
%&= \big(a_{1}a_{2}a_{3} - a_{1}b_{2}b_{3} - b_{1}a_{2}b_{3} - b_{1}b_{2}a_{3}, a_{1}a_{2}b_{3} + a_{1}b_{2}a_{3} + b_{1}a_{2}a_{3} - b_{1}b_{2}b_{3}\big)
%\end{align*}

%Next we work out the right hand side:
%\begin{align*}
%(wz)\lambda &= \big((a_{1},b_{1})\cdot(a_{2},b_{2})\big)\cdot (a_{3},b_{3})\\
%&= (a_{1}a_{2} - b_{1}b_{2}, a_{1}b_{2} + b_{1}a_{2})\cdot (a_{3},b_{3})\\
%&= \big(a_{3}(a_{1}a_{2} - b_{1}b_{2}) - b_{3}(a_{1}b_{2} + b_{1}a_{2}), b_{3}(a_{1}a_{2} - b_{1}%b_{2}) + a_{3}(a_{1}b_{2} + b_{1}a_{2})\big)\\
%&= \big(a_{1}a_{2}a_{3} - b_{1}b_{2}a_{3} - a_{1}b_{2}a_{3} - b_{1}a_{2}b_{3}, a_{1}a_{2}b_{3} - b_{1}b_{2}b_{3}+a_{1}b_{2}a_{3} + b_{1}a_{2}a_{3}\big)\\
%&= \big(a_{1}a_{2}a_{3} - a_{1}b_{2}a_{3} - b_{1}a_{2}b_{3} - b_{1}b_{2}a_{3}, a_{1}a_{2}b_{3} + a_{1}b_{2}a_{3} + b_{1}a_{2}a_{3} - b_{1}b_{2}b_{3} \big)
%\end{align*}
%It follows that $w(z\lambda) = (wz)\lambda$, as required.

\item[(d)] Write $w = (a_{1},b_{1}), z = (a_{2},b_{2})$, and $\lambda = (a_{3},b_{3})$ for some $a_{1},a_{2},a_{3},b_{1},b_{2},b_{3}\in\R$.  Then
\begin{align*}
w(z+\lambda) &= (a_{1},b_{1})\cdot \big((a_{2},b_{2}) + (a_{3},b_{3})\big)\\
&= (a_{1},b_{1})\cdot (a_{2}+a_{3},b_{2} + b_{3})\\
&= \big(a_{1}(a_{2} + a_{3}) - b_{1}(b_{2} + b_{3}), a_{1}(b_{2}+b_{3}) + b_{1}(a_{2}+a_{3})\big)\\
&= \big(a_{1}a_{2} + a_{1}a_{3} - b_{1}b_{2} - b_{1}b_{3}, a_{1}b_{2} + a_{1}b_{3} + b_{1}a_{2} + b_{1}a_{3}\big)\\
&= (a_{1}a_{2} - b_{1}b_{2},a_{1}b_{2} + b_{1}a_{2}) + (a_{1}a_{3} - b_{1}b_{3}, a_{1}b_{3} + b_{1}a_{3})\\
&= (a_{1},b_{1})\cdot(a_{2},b_{2}) + (a_{1},b_{1})\cdot (a_{3},b_{3})\\
&= wz + w\lambda
\end{align*}
\end{enumerate}

\item Prove that $\RE{iz} = -\IM{z}$ for every $z\in\C$.
\item Show that for any $z_{1},z_{2}\in\COMPLEX$, we have $\CC{z_{1} + z_{2}} = \CC{z_{1}} + \CC{z_{2}}$ and $\CC{z_{1}z_{2}} = \CC{z_{1}}\cdot\CC{z_{2}}$.
\item Compute $3i^{11} + 6i^{3} + \frac{8}{i^{20}} + i^{-1}$.
\item If $z\in\C$ with $|z| = 1$ and $z\neq 1$, then check that $\RE{\dfrac{1}{1-z}} = \dfrac{1}{2}$.
\item Show that if $\big(\CC{z}\big)^2 = z^2$, then $z$ is either real or purely imaginary.
%\item Be sure to provide a complete justification when answering each of the following questions.
%\begin{enumerate}[(a)]
%\item Does there exist $a_{0},a_{1},a_{2},a_{3}\in \R$ such that
%\begin{center}
%$3(z-2i)(z+i)(z-2) = a_{0}(z-a_{1})(z - a_{2})(z - z_{3})$?
%\end{center}
%\item Does there exist $z_{0},z_{1},z_{2},z_{3},z_{4},z_{5}\in \C$ such that
%\begin{center}
%$3z^5 - 2iz^4 + \pi z^2 - i = z_{0}(z - z_{1})(z - z_{2})(z - z_{3})(z - z_{4})(z - z_{5})$?
%\end{center}
%\end{enumerate}
\item Find a polar coordinate representation for each of the following complex numbers.
\begin{multicols}{3}
\begin{enumerate}[(a)]
\item $-6$
\item $\sqrt{3} + i$
\item $-2-2i$
\end{enumerate}
\end{multicols}
\item The following ordered pairs are each polar coordinate representations of a complex number.  Write the corresponding complex number in the form $a + bi$.
\begin{multicols}{3}
\begin{enumerate}[(a)]
\item $(\pi/6,3)$
\item $(3\pi/2,2)$
\item $(7\pi/4,1)$
\end{enumerate}
\end{multicols}
\item Calculate $\dfrac{\big(\sqrt{2} + \sqrt{6}i\big)^5}{\big(1+i\big)^9}$.
\item Prove that for any $z\in\C$, $|z^{n}| = |z|^{n}$.
\item Is it true that $\PARG{z_1z_2} = \PARG{z_1} + \PARG{z_2}$ for every $z_{1},z_{2}\in\C$?
\item Find all the $8$th roots of unity, and sketch them in the complex plane.
\item Let $k\in\Z^{+}$ and suppose that $\zeta$ is a primitive $n$th root of unity.  Show that $\zeta^{k}$ is a primitive $n$th root of unity if and only if $\GCD{n,k} = 1$.
\end{enumerate}
\vsp

{\huge Chapter 3}
\vsp

\begin{enumerate}
\item Prove that if $a\overset{n}{\equiv}b$, then $a^2 + b^2\overset{n^2}{\equiv}2ab$.
\item Suppose $a,b\in\Z$ such that $a\overset{n}{\equiv}b$ for every prime $p$.  Prove that $a=b$.
\item Decide whether each of the following statements are true.  If they are true, then prove them.  Otherwise, provide a counter-example.
\begin{enumerate}[(a)]
\item If $a^2\overset{n}{\equiv}b^2$, then either $a\overset{n}{\equiv}b$ or $a\overset{n}{\equiv}-b$.
\item If $p$ is prime and $a^2\overset{p}{\equiv}b^2$, then either $a\overset{p}{\equiv}b$ or $a\overset{p}{\equiv}-b$.
\item If $[a] = [b]$ in $\ZN$, then $\GCD{a,n} = \GCD{b,n}$.
\end{enumerate}
\item If $[a] = [1]$ in $\ZN$, prove that $\GCD{a,n} = 1$.  Demonstrate also that the converse fails to be true in general.
\end{enumerate}
\end{document}





