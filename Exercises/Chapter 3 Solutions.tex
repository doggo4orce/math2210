\documentclass[11pt,fleqn,dvipsnames,usenames]{article}

% to keep this file less overwhelming
% packages to include

\usepackage[dvipsnames, table]{xcolor}

\usepackage{
  amsthm,
  amsmath,
  amssymb, 
  arydshln, % for hyphenated lines in block matrices
  fancyhdr, % needed for header at top of each page
  graphicx, % to include pictures
  mathtools, % for a longer arrow
  multicol, % displaying enumerates and itemizes into multiple columns
  multirow, % for tables
  multido, % for TOC
  pgfplots, % for axis environment within tikz pictures
  systeme,
  tikz,
}

\usepackage[utf8]{inputenc}
\usepackage{color,soul}

\usepackage[inline, shortlabels]{enumitem}
\usepackage[hidelinks]{hyperref}


% global constants
\newcommand{\term}{Winter 2024}

% mathbb aliases
\newcommand{\COMPLEX}{\mathbb{C}}
\newcommand{\REAL}{\mathbb{R}}
\newcommand{\NATURAL}{\mathbb{N}}
\newcommand{\INTEGER}{\mathbb{Z}}

% for financial stuff
\newcommand{\dollar}{\mathrm{\$}}

% nicer looking trig functions
\newcommand{\SIN}[1]{\sin\left(#1\right)}
\newcommand{\COS}[1]{\cos\left(#1\right)}
\newcommand{\TAN}[1]{\tan\left(#1\right)}
\newcommand{\CSC}[1]{\csc\left(#1\right)}
\newcommand{\SEC}[1]{\sec\left(#1\right)}
\newcommand{\COT}[1]{\cot\left(#1\right)}

% automatically resize set brackets
\newcommand{\SET}[1]{\left\{#1\right\}}

% sums and products
\newcommand{\SUM}{\displaystyle\sum\limits}
\newcommand{\PROD}{\displaystyle\prod\limits}
\newcommand{\of}{\circ}
\newcommand{\restrict}[1]{\raisebox{-.5ex}{$|$}_{#1}}

% set intersection and union
\newcommand{\CAP}{\displaystyle\bigcap\limits}
\newcommand{\CUP}{\displaystyle\bigcup\limits}

% max and min
\newcommand{\MAX}[1]{\ensuremath{\max\left(#1\right)}}
\newcommand{\MIN}[1]{\ensuremath{\min\left(#1\right)}}

% for writing logic within mathematics environment
\newcommand{\FORALL}{\ensuremath{\text{ for all }}}
\newcommand{\FORSOME}{\ensuremath{\text{ for some }}}

% matrix notation
\newcommand{\MATRIX}[2]{\ensuremath{\left[\begin{array}{#1}#2\end{array}\right]}}
\newcommand{\COLUMN}[1]{\ensuremath{\left[\begin{array}{r}#1\end{array}\right]}}

% vector notation
%\newcommand{\vv}{\overset{\rightharpoonup}}
\newcommand{\vv}[1]{{\bf #1}}
\newcommand{\arr}{\overrightarrow}

% dot product
\newcommand{\dotp}{{\scriptstyle\bullet}}

% Text macros
\newcommand{\KER}[1]{\ensuremath{\text{ker}\left(#1\right)}}
\newcommand{\IMG}[1]{\ensuremath{\text{im}\left(#1\right)}}
\newcommand{\CHAR}[1]{\ensuremath{\text{char}\left(#1\right)}}
\newcommand{\BIGO}[1]{\ensuremath{\mathcal{O}\left(#1\right)}}
\newcommand{\TR}[1]{\ensuremath{\text{tr}\left(#1\right)}}

% abbreviations
\newcommand{\ds}{\displaystyle}
\newcommand{\md}{\mdseries}
\newcommand{\vsp}{\vspace{0.5cm}}
\newcommand{\smsp}{\vspace{0.25cm}}
\newcommand{\hsp}{\hspace{0.25cm}}

% new operators
\DeclareMathOperator\SPAN{Span}
\newcommand{\SPANOF}[1]{\ensuremath{\SPAN\left\{#1\right\}}}
\DeclareMathOperator\PROJ{proj}
\DeclareMathOperator\PERP{perp}

% quick abbreviations to avoid using latex environments
\newcommand{\answer}{\noindent \textbf{Answer:} }
\newcommand{\answers}{\noindent \textbf{Answers:} }
\newcommand{\application}{\noindent \textbf{Application:} }
\newcommand{\caution}{\noindent \textbf{Caution:} }
\newcommand{\conclusion}{\noindent \textbf{Conclusion:} }
\newcommand{\consequence}{\noindent \textbf{Consequence:} }
\newcommand{\defn}{\noindent \textbf{Definition:} }
\newcommand{\details}{\noindent \textbf{Details:} }
\newcommand{\example}{\noindent \textbf{Example:} }
\newcommand{\examples}{\noindent \textbf{Examples:} }
\newcommand{\exception}{\noindent \textbf{Exception:} }
\newcommand{\exercise}{\noindent \textbf{Exercise:} }
\newcommand{\exercises}{\noindent \textbf{Exercises:} }
\newcommand{\fact}{\noindent \textbf{Fact:} }
\newcommand{\facts}{\noindent \textbf{Facts:} }
\newcommand{\formula}{\noindent \textbf{Formula:} }
\newcommand{\goal}{\noindent \textbf{Goal:} }
\newcommand{\goals}{\noindent \textbf{Goals:} }
\newcommand{\hint}{\noindent \textbf{Hint:} }
\newcommand{\idea}{\noindent \textbf{Idea:} }
\newcommand{\illustration}{\noindent \textbf{Illustration:} }
\newcommand{\important}{\noindent \textbf{Important:} }
\newcommand{\midea}{\noindent \textbf{Main Idea:} }
\newcommand{\motivation}{\noindent \textbf{Motivation:} }
\newcommand{\nthm}[1]{\noindent \textbf{Theorem} (\textit{#1}):}
\newcommand{\notation}{\noindent \textbf{Notation:} }
\newcommand{\note}{\noindent \textbf{Note:} }
\newcommand{\notes}{\noindent \textbf{Notes:} }
\newcommand{\observation}{\noindent \textbf{Observation:} }
\newcommand{\observations}{\noindent \textbf{Observations:} }
\newcommand{\pict}{\noindent \textbf{Picture:} }
\newcommand{\plan}{\noindent \textbf{Plan:} }
\newcommand{\prf}{\noindent \textbf{Proof:} }
\newcommand{\problem}{\noindent \textbf{Problem:} }
\newcommand{\properties}{\noindent \textbf{Properties:} }
\newcommand{\question}{\noindent \textbf{Question:} }
\newcommand{\questions}{\noindent \textbf{Questions:} }
\newcommand{\recall}{\noindent \textbf{Recall:} }
\newcommand{\reason}{\noindent \textbf{Reason:} }
\newcommand{\remark}{\noindent \textbf{Remark:} }
\newcommand{\remarks}{\noindent \textbf{Remarks:} }
\newcommand{\reminder}{\noindent \textbf{Reminder:} }
\newcommand{\solution}{\noindent \textbf{Solution:} }
\newcommand{\nsolution}[1]{\noindent \textbf{Solution #1:} }
\newcommand{\strategy}{\noindent \textbf{Strategy:} }
\newcommand{\summary}{\noindent \textbf{Summary:} }
\newcommand{\terminology}{\noindent \textbf{Terminology:} }
\newcommand{\thm}{\noindent \textbf{Theorem:} }
\newcommand{\work}{\noindent \textbf{Work:} }


\usepackage[version=4]{mhchem}

% Where to look for pngs and jpegs
\graphicspath{{Images//}}

\usepackage[includehead, includefoot, left= 2cm, top =1.5cm, bottom = 1.5cm, textwidth=17.5cm]{geometry}

\usepackage{pifont, amsmath}

\pagestyle{fancy}
\fancyhf{}
\renewcommand{\headrulewidth}{1pt}
%\fancyhead[R]{\bfseries\sffamily\thepage}
%\fancyfoot[C]{\bfseries\sffamily\thepage}
\fancyhead[L]{\nouppercase{\bfseries\sffamily\leftmark}}

% used when adding fill-in-the-blanks for students
\newcommand{\blank}[1]{\underline{\hspace{#1}}}

% indents annoy me, and so does repeatedly typing \noindent
\newcommand{\p}{\noindent}

\begin{document}

\fancyhead[L]{\course}
\fancyhead[C]{\includegraphics[width=5cm, trim= 0 0.4cm 0 0]{TRU_logo}}
\fancyhead[R]{\term}
\renewcommand{\headrulewidth}{0.4pt}

{\huge Chapter 3 Solutions}
\vsp

\begin{enumerate}
\item Prove that if $a\overset{n}{\equiv}b$ and $c\overset{n}{\equiv}d$, then $a-c\overset{n}{\equiv}b-d$.
\item Prove that if $a\overset{n}{\equiv}b$, then $a^2 + b^2\overset{n^2}{\equiv}2ab$.
\vsmsp

\solution If $a\overset{n}{\equiv}b$, then $n|b-a$.  So $n^2$ must be a divisor of
\begin{center}
$(b-a)^2 = b^2 - 2ab + a^2 = (a^2 + b^2) - 2ab$.
\end{center}
Hence $a^2 + b^2\overset{n^2}{\equiv}2ab$.
\item Solve each congruence.
\begin{multicols}{3}
\begin{enumerate}[(a)]
\item $x^2\overset{3}{\equiv}1$
\item $x^2\overset{7}{\equiv}2$
\item $x^2\overset{11}{\equiv}3$
\end{enumerate}
\end{multicols}
\vsmsp

\solution
\begin{enumerate}[(a)]
\item Every integer $x$ is congruent modulo $3$ to exactly one element of the set $\SET{0,1,2}$.  So check each possibility:
\begin{itemize}[\ ]
\item If $x\overset{3}{\equiv}0$, then $x^2\overset{3}{\equiv}0^2 = 0\not\overset{3}{\equiv}1$.
\item If $x\overset{3}{\equiv}1$, then $x^2\overset{3}{\equiv}1^2 = 1\overset{3}{\equiv}1$.
\item If $x\overset{3}{\equiv}2$, then $x^2\overset{3}{\equiv}2^2 = 4\overset{3}{\equiv}1$.
\end{itemize}
So the congruence is satisfied whenever $x\overset{3}{\equiv}1$ or $x\overset{3}{\equiv}1$.
\item Every integer $x$ is congruent modulo $7$ to exactly one element of the set $\SET{0,1,2,3,4,5,6}$.  So check each possibility:
\begin{itemize}[\ ]
\item If $x\overset{7}{\equiv}0$, then $x^2\overset{7}{\equiv}0^2 = 0\not\overset{7}{\equiv}2$.
\item If $x\overset{7}{\equiv}1$, then $x^2\overset{7}{\equiv}1^2 = 1\not\overset{7}{\equiv}2$.
\item If $x\overset{7}{\equiv}2$, then $x^2\overset{7}{\equiv}2^2 = 4\not\overset{7}{\equiv}2$.
\item If $x\overset{7}{\equiv}3$, then $x^2\overset{7}{\equiv}3^2 = 9\overset{7}{\equiv}2$.
\item If $x\overset{7}{\equiv}4$, then $x^2\overset{7}{\equiv}4^2 = 16\overset{7}{\equiv}2$.
\item If $x\overset{7}{\equiv}5$, then $x^2\overset{7}{\equiv}5^2 = 25\overset{7}{\equiv}4\overset{7}{\equiv}2$.
\item If $x\overset{7}{\equiv}6$, then $x^2\overset{7}{\equiv}6^2 = 36\overset{7}{\equiv}1\not\overset{7}{\equiv}2$.
\end{itemize}
So the congruence is satisfied whenever $x\overset{7}{\equiv}3$ or $x\overset{7}{\equiv}4$.
\end{enumerate}
%\item Suppose $a,b\in\Z$ such that $a\overset{p}{\equiv}b$ for every prime $p$.  Prove that $a=b$.
<<<<<<< HEAD
\item Prove that if $p$ is prime and $a^2\overset{p}{\equiv}b^2$, then either $a\overset{p}{\equiv}b$ or $a\overset{p}{\equiv}-b$.
\vsmsp

\solution Suppose that $p$ is prime.  If $a^2\overset{p}{\equiv}b^2$, then $p$ is a divisor of
\begin{center}
$a^2 - b^2 = (a-b)(a+b)$.
\end{center}
By Euclid's Lemma, $p|a-b$ or $p|a+b$.  Therefore $a\overset{p}{\equiv}b$ or $a\overset{p}{\equiv}-b$
\item Find $a,b\in\Z$ such that $a^2\overset{n}{\equiv}b^2$, but $a\not\overset{n}{\equiv}b$ and $a\not\overset{n}{\equiv}-b$.
\vsmsp

\solution Take $a = 7$, $b=3$, and $n=40$.  Then
\begin{center}
$7^2 = 49 \overset{40}{\equiv}9 = 3^2$
\end{center}
but $7\not\overset{40}{\equiv}3$ and $7\not\overset{40}{\equiv}-3$.

=======
\item \label{asquaredbsquaredpprime} Prove that if $p$ is prime and $a^2\overset{p}{\equiv}b^2$, then either $a\overset{p}{\equiv}b$ or $a\overset{p}{\equiv}-b$.
\item Is the assumption that $p$ is prime required in Problem \ref{asquaredbsquaredpprime}?  That is, does there exist $n\in\Z^{+}$ and $a,b\in\Z$ such that
\begin{center}
$a^2\overset{n}{\equiv}b^2$ but $a\not\overset{n}{\equiv}b$ and $a\not\overset{n}{\equiv}-b$?
\end{center}
>>>>>>> ba054c934216e2a9c1165f9e7f476d225b70ecf5
\item If $[a] = [b]$ in $\ZN$, then show that $\GCD{a,n} = \GCD{b,n}$.
\vsmsp

\solution Suppose $[a] = [b]$.  Then $a\overset{n}{\equiv}b$ and hence $a = b + kn$ for some $k\in\Z$.  Then $\GCD{a,n} = \GCD{b,n}$ follows from the fact that the common divisors of $a$ and $n$ are exactly the common divisors of $b$ and $n$.  Indeed, any integer that divides both $a$ and $n$, must also divide $b = a - kn$.  And conversely, an integer that divides both $b$ and $n$ must also divide $a = b + kn$.

\item Let $n\in\Z^{+}$.  If $a\in \Z$ such that $[a] = [1]$ in $\ZN$, prove that $\GCD{a,n} = 1$.  Demonstrate also that the converse fails to be true in general.
\vsmsp

\solution Suppose $a\in\Z$ such that $[a] = [1]$ in $\ZN$.  Suppose for a contradiction that $d>1$ is a common divisor of $a$ and $n$.  Then $1 \leq n/d < n$ and hence $n$ does not divide $n/d$.  But this contradicts the fact that
\begin{center}
$n/d = 1\cdot n/d \overset{n}{\equiv} a\cdot n/d = (a/d)\cdot n \overset{n}{\equiv}(a/d)\cdot 0 = 0$.
\end{center}
\vsmsp

\p To see that the converse does not hold in general, note that 

\item Let $A = \SET{1,2,3,4}$.  Determine which of the following relations are reflexive.  Which are symmetric?  Which are transitive?
\begin{enumerate}[(a)]
\item $\SET{(1,1), (1,2), (2,2), (3,3), (4,4)}$
\item $\SET{(1,2), (2,3), (3,1), (4,4)}$
\item $\SET{(1,1), (1,2), (1,4), (2,1), (2,2), (2,4), (3,3), (4,4), (4,1)}$
\end{enumerate}
\item Let $A$ be a non-empty set and let $\mathcal{P}$ be a collection of subsets of $A$ satisfying the following two properties:
\begin{enumerate}[(1)]
\item For any $a\in A$, there exists $U\in\mathcal{P}$ such that $a\in U$.  That is, $\mathcal{P}$ \textbf{covers} $A$.
\item For any $A,B\in\mathcal{P}$, either $A = B$ or $A\cap B = \emptyset$.  That is, $\mathcal{P}$ is \textbf{pairwise disjoint}.
\end{enumerate}
Define the relation $T$ on $A$ by
\begin{center}
$T = \big\{(a,b)\in A\times A:\text{ there exists }A\in\mathcal{P}\text{ such that }a\in A\text{ and }b\in A\big\}$.
\end{center}
\begin{enumerate}[(a)]
\item Prove that $T$ is an equivalence relation.
\item Describe the associated equivalence classes.
\end{enumerate}
\item Recall that addition and multiplication in $\Z_n$ were defined by
\begin{center}
$[a] \oplus [b] = [a + b]$ and $[a]\odot [b] = [ab]$.
\end{center}
\p Both of these objects refer to representatives of their operands.  Verify that the results of these operations are independent of the choice of representatives.  That is, prove that if $a,a',b,b'\in\Z$ are such that $[a] = [a']$ and $[b] = [b']$, then
\begin{center}
$[a]\oplus [b] = [a']\oplus [b']$ and $[a]\odot [b] = [a']\odot [b']$.
\end{center}

\note It is because of this fact that these operations are well defined!

\item Define the relation $T$ on $\R$ by
\begin{center}
$T = \SET{(x,x): x\in\R}\cup \SET{(x,-x): x\in\R}$.
\end{center}
Prove that $T$ is an equivalence relation.

\item Let $\mathcal{P}(\Z)$ denote the set of all subsets of $\Z$.  Prove that
\begin{center}
$T = \SET{(U,V)\in \mathcal{P}(\Z)\times\mathcal{P}(\Z):\text{ there exists a bijection }f:U\to V}$
\end{center}
is an equivalence relation on $\mathcal{P}(\Z)$.
\vsmsp

\recall A \textbf{bijection} is a function which is one-to-one and onto.

\item Let $n\in\Z^+$ with $n > 2$.  Prove that a zero-divisor in $\Z_n$ cannot be a unit.
\item Solve $12x = 3$ in $\Z_{31}$
\item Find all solutions to $16x = 14$ in $\Z_{26}$.
\end{enumerate}
\end{document}





