\documentclass[11pt,fleqn,dvipsnames,usenames]{article}

% to keep this file less overwhelming
% packages to include

\usepackage[dvipsnames, table]{xcolor}

\usepackage{
  amsthm,
  amsmath,
  amssymb, 
  arydshln, % for hyphenated lines in block matrices
  fancyhdr, % needed for header at top of each page
  graphicx, % to include pictures
  mathtools, % for a longer arrow
  multicol, % displaying enumerates and itemizes into multiple columns
  multirow, % for tables
  multido, % for TOC
  pgfplots, % for axis environment within tikz pictures
  systeme,
  tikz,
}

\usepackage[utf8]{inputenc}
\usepackage{color,soul}

\usepackage[inline, shortlabels]{enumitem}
\usepackage[hidelinks]{hyperref}


% global constants
\newcommand{\term}{Winter 2024}

% mathbb aliases
\newcommand{\COMPLEX}{\mathbb{C}}
\newcommand{\REAL}{\mathbb{R}}
\newcommand{\NATURAL}{\mathbb{N}}
\newcommand{\INTEGER}{\mathbb{Z}}

% for financial stuff
\newcommand{\dollar}{\mathrm{\$}}

% nicer looking trig functions
\newcommand{\SIN}[1]{\sin\left(#1\right)}
\newcommand{\COS}[1]{\cos\left(#1\right)}
\newcommand{\TAN}[1]{\tan\left(#1\right)}
\newcommand{\CSC}[1]{\csc\left(#1\right)}
\newcommand{\SEC}[1]{\sec\left(#1\right)}
\newcommand{\COT}[1]{\cot\left(#1\right)}

% automatically resize set brackets
\newcommand{\SET}[1]{\left\{#1\right\}}

% sums and products
\newcommand{\SUM}{\displaystyle\sum\limits}
\newcommand{\PROD}{\displaystyle\prod\limits}
\newcommand{\of}{\circ}
\newcommand{\restrict}[1]{\raisebox{-.5ex}{$|$}_{#1}}

% set intersection and union
\newcommand{\CAP}{\displaystyle\bigcap\limits}
\newcommand{\CUP}{\displaystyle\bigcup\limits}

% max and min
\newcommand{\MAX}[1]{\ensuremath{\max\left(#1\right)}}
\newcommand{\MIN}[1]{\ensuremath{\min\left(#1\right)}}

% for writing logic within mathematics environment
\newcommand{\FORALL}{\ensuremath{\text{ for all }}}
\newcommand{\FORSOME}{\ensuremath{\text{ for some }}}

% matrix notation
\newcommand{\MATRIX}[2]{\ensuremath{\left[\begin{array}{#1}#2\end{array}\right]}}
\newcommand{\COLUMN}[1]{\ensuremath{\left[\begin{array}{r}#1\end{array}\right]}}

% vector notation
%\newcommand{\vv}{\overset{\rightharpoonup}}
\newcommand{\vv}[1]{{\bf #1}}
\newcommand{\arr}{\overrightarrow}

% dot product
\newcommand{\dotp}{{\scriptstyle\bullet}}

% Text macros
\newcommand{\KER}[1]{\ensuremath{\text{ker}\left(#1\right)}}
\newcommand{\IMG}[1]{\ensuremath{\text{im}\left(#1\right)}}
\newcommand{\CHAR}[1]{\ensuremath{\text{char}\left(#1\right)}}
\newcommand{\BIGO}[1]{\ensuremath{\mathcal{O}\left(#1\right)}}
\newcommand{\TR}[1]{\ensuremath{\text{tr}\left(#1\right)}}

% abbreviations
\newcommand{\ds}{\displaystyle}
\newcommand{\md}{\mdseries}
\newcommand{\vsp}{\vspace{0.5cm}}
\newcommand{\smsp}{\vspace{0.25cm}}
\newcommand{\hsp}{\hspace{0.25cm}}

% new operators
\DeclareMathOperator\SPAN{Span}
\newcommand{\SPANOF}[1]{\ensuremath{\SPAN\left\{#1\right\}}}
\DeclareMathOperator\PROJ{proj}
\DeclareMathOperator\PERP{perp}

% quick abbreviations to avoid using latex environments
\newcommand{\answer}{\noindent \textbf{Answer:} }
\newcommand{\answers}{\noindent \textbf{Answers:} }
\newcommand{\application}{\noindent \textbf{Application:} }
\newcommand{\caution}{\noindent \textbf{Caution:} }
\newcommand{\conclusion}{\noindent \textbf{Conclusion:} }
\newcommand{\consequence}{\noindent \textbf{Consequence:} }
\newcommand{\defn}{\noindent \textbf{Definition:} }
\newcommand{\details}{\noindent \textbf{Details:} }
\newcommand{\example}{\noindent \textbf{Example:} }
\newcommand{\examples}{\noindent \textbf{Examples:} }
\newcommand{\exception}{\noindent \textbf{Exception:} }
\newcommand{\exercise}{\noindent \textbf{Exercise:} }
\newcommand{\exercises}{\noindent \textbf{Exercises:} }
\newcommand{\fact}{\noindent \textbf{Fact:} }
\newcommand{\facts}{\noindent \textbf{Facts:} }
\newcommand{\formula}{\noindent \textbf{Formula:} }
\newcommand{\goal}{\noindent \textbf{Goal:} }
\newcommand{\goals}{\noindent \textbf{Goals:} }
\newcommand{\hint}{\noindent \textbf{Hint:} }
\newcommand{\idea}{\noindent \textbf{Idea:} }
\newcommand{\illustration}{\noindent \textbf{Illustration:} }
\newcommand{\important}{\noindent \textbf{Important:} }
\newcommand{\midea}{\noindent \textbf{Main Idea:} }
\newcommand{\motivation}{\noindent \textbf{Motivation:} }
\newcommand{\nthm}[1]{\noindent \textbf{Theorem} (\textit{#1}):}
\newcommand{\notation}{\noindent \textbf{Notation:} }
\newcommand{\note}{\noindent \textbf{Note:} }
\newcommand{\notes}{\noindent \textbf{Notes:} }
\newcommand{\observation}{\noindent \textbf{Observation:} }
\newcommand{\observations}{\noindent \textbf{Observations:} }
\newcommand{\pict}{\noindent \textbf{Picture:} }
\newcommand{\plan}{\noindent \textbf{Plan:} }
\newcommand{\prf}{\noindent \textbf{Proof:} }
\newcommand{\problem}{\noindent \textbf{Problem:} }
\newcommand{\properties}{\noindent \textbf{Properties:} }
\newcommand{\question}{\noindent \textbf{Question:} }
\newcommand{\questions}{\noindent \textbf{Questions:} }
\newcommand{\recall}{\noindent \textbf{Recall:} }
\newcommand{\reason}{\noindent \textbf{Reason:} }
\newcommand{\remark}{\noindent \textbf{Remark:} }
\newcommand{\remarks}{\noindent \textbf{Remarks:} }
\newcommand{\reminder}{\noindent \textbf{Reminder:} }
\newcommand{\solution}{\noindent \textbf{Solution:} }
\newcommand{\nsolution}[1]{\noindent \textbf{Solution #1:} }
\newcommand{\strategy}{\noindent \textbf{Strategy:} }
\newcommand{\summary}{\noindent \textbf{Summary:} }
\newcommand{\terminology}{\noindent \textbf{Terminology:} }
\newcommand{\thm}{\noindent \textbf{Theorem:} }
\newcommand{\work}{\noindent \textbf{Work:} }


\usepackage[version=4]{mhchem}

% Where to look for pngs and jpegs
\graphicspath{{Images//}}

\usepackage[includehead, includefoot, left= 2cm, top =1.5cm, bottom = 1.5cm, textwidth=17.5cm]{geometry}

\usepackage{pifont, amsmath}

\pagestyle{fancy}
\fancyhf{}
\renewcommand{\headrulewidth}{1pt}
%\fancyhead[R]{\bfseries\sffamily\thepage}
%\fancyfoot[C]{\bfseries\sffamily\thepage}
\fancyhead[L]{\nouppercase{\bfseries\sffamily\leftmark}}

% used when adding fill-in-the-blanks for students
\newcommand{\blank}[1]{\underline{\hspace{#1}}}

% indents annoy me, and so does repeatedly typing \noindent
\newcommand{\p}{\noindent}

\begin{document}

\fancyhead[L]{\course}
\fancyhead[C]{\includegraphics[width=5cm, trim= 0 0.4cm 0 0]{TRU_logo}}
\fancyhead[R]{\term}
\renewcommand{\headrulewidth}{0.4pt}

\p {\huge \S4.1 Problems}
\vsp

\begin{enumerate}[1.]
\item Is $\Q$ a binary algebraic structure under the operation $*$ defined by
\begin{center}
$\dfrac{a}{b}*\dfrac{c}{d} = \dfrac{a}{d} + \dfrac{c}{b}$
\end{center}
for all $\dfrac{a}{b},\dfrac{c}{d}\in \Q$?    If not, describe the problem completely.
\vsmsp

\solution No.  The operation $*$ is not well-defined.  In particular, $1/2 = 2/4$, but
\begin{center}
$\dfrac{1}{2}*\dfrac{2}{3} \neq \dfrac{2}{4}*\dfrac{2}{3}$,
\end{center}
since
\begin{center}
$\dfrac{1}{2}*\dfrac{2}{3} = \dfrac{1}{3} + \dfrac{2}{2} = \dfrac{4}{3}$
\end{center}
and
\begin{center}
$\dfrac{2}{4}*\dfrac{2}{3} = \dfrac{1}{3} + \dfrac{2}{4} = \dfrac{5}{6}$.
\end{center}
\item Let $\Q^{+} = \SET{x\in\Q:x > 0}$.  Define the binary operation on $\Q^{*}$ by
\begin{center}
$x*y = \sqrt{xy}$, for all $x,y\in\Q^{+}$.
\end{center}
Is $(\Q^{+},*)$ a binary algebraic structure?
\vsmsp

\solution No.  $*$ is not a binary operation on $\Q$, since $1*2 = \sqrt{1\cdot 2} = \sqrt{2}\notin \Q$.

\item \label{m2rbas}Recall that $M_{2}(\R)$ is the set of all $2\times 2$ matrices of the form
\begin{center}
$\MATRIX{rr}{a & b\\c& d}$ for some $a,b,c,d\in\R$.
\end{center}
$M_{2}(\R)$ is a binary algebraic structure under both the addition and multiplication operations defined by
\begin{center}
$\MATRIX{rr}{a_{1} & b_{1}\\c_{1} & d_{1}} + \MATRIX{rr}{a_{2} & b_{2}\\c_{2} & d_{2}} = \MATRIX{rr}{a_{1} + a_{2} & b_{1} + b_{2}\\c_{1} + c_{2} & d_{1} + d_{2}}$ 
\end{center}
and 
\begin{center}
$\MATRIX{rr}{a_{1} & b_{1}\\c_{1} & d_{1}} * \MATRIX{rr}{a_{2} & b_{2}\\c_{2} & d_{2}} = \MATRIX{rr}{a_{1}a_{2} + b_{1}c_{2} & a_{1}b_{2} + b_{1}d_{2}\\c_{1}a_{2} + d_{1}c_{2} & c_{1}b_{2} + d_{1}d_{2}}$.
\end{center}
\begin{enumerate}[(a)]
\item Prove that $(M_{2}(\R), +)$ is both commutative and associative.
\item Prove that $(M_{2}(\R), *)$ is associative but not commutative.
\end{enumerate}
\end{enumerate}
\vsp

\solution
\begin{enumerate}[(a)]
\item For any $\MATRIX{rr}{a_{1} & b_{1}\\c_{1} & d_{1}}, \MATRIX{rr}{a_{2} & b_{2}\\c_{2} & d_{2}}\in M_{2}(\R)$, we have
\begin{adjustwidth}{-50pt}{-50pt}
\begin{center}
$\MATRIX{rr}{a_{1} & b_{1}\\c_{1} & d_{1}} + \MATRIX{rr}{a_{2} & b_{2}\\c_{2} & d_{2}} = \MATRIX{rr}{a_{1} + a_{2} & b_{1} + b_{2}\\c_{1} + c_{2} & d_{1} + d_{2}} = \MATRIX{rr}{a_{2} + a_{1} & b_{2} + b_{1}\\c_{2} + c_{1} & d_{2} + d_{1}} = \MATRIX{rr}{a_{2} & b_{2}\\c_{2} & d_{2}} + \MATRIX{rr}{a_{1} & b_{1}\\c_{1} & d_{1}}$.
\end{center}
\end{adjustwidth}
and for any $\MATRIX{rr}{a_{1} & b_{1}\\c_{1} & d_{1}}, \MATRIX{rr}{a_{2} & b_{2}\\c_{2} & d_{2}}, \MATRIX{rr}{a_{3} & b_{3}\\c_{3} & d_{3}}\in M_{2}(\R)$, we have
\begin{align*}
\MATRIX{cc}{a_{1} & b_{1}\\c_{1} & d_{1}} + \left(\MATRIX{cc}{a_{2} & b_{2}\\c_{2} & d_{2}} + \MATRIX{cc}{a_{3} & b_{3}\\c_{3} & d_{3}}\right) &= \MATRIX{rr}{a_{1} & b_{1}\\c_{1} & d_{1}} + \MATRIX{cc}{a_{2} + a_{3} & b_{2} + b_{3}\\c_{2} + c_{3} & d_{2} + d_{3}}\\
&= \MATRIX{cc}{a_{1} + (a_{2} + a_{3}) & b_{1} + (b_{2} + b_{3})\\c_{1} + (c_{2} + c_{3}) & d_{1} + (d_{2} + d_{3})}\\
&= \MATRIX{cc}{(a_{1} + a_{2}) + a_{3} & (b_{1} + b_{2}) + b_{3}\\(c_{1} + c_{2}) + c_{3} & (d_{1} + d_{2}) + d_{3}}\\
&= \MATRIX{cc}{a_{1} + a_{2} & b_{1} + b_{2}\\c_{1} + c_{2} & d_{1} + d_{2}}  + \MATRIX{cc}{a_{3} & b_{3}\\c_{3} & d_{3}}\\
&= \left(\MATRIX{cc}{a_{1} & b_{1}\\c_{1} & d_{1}} + \MATRIX{cc}{a_{2} & b_{2}\\c_{2} & d_{2}}\right) + \MATRIX{cc}{a_{3} & b_{3}\\c_{3} & d_{3}}.
\end{align*}
Therefore $(M_{2}(\R), +)$ is both commutative and associative.
\item For any $\MATRIX{rr}{a_{1} & b_{1}\\c_{1} & d_{1}}, \MATRIX{rr}{a_{2} & b_{2}\\c_{2} & d_{2}}, \MATRIX{rr}{a_{3} & b_{3}\\c_{3} & d_{3}}\in M_{2}(\R)$, we have
\begin{adjustwidth}{-75pt}{0pt}
\begin{align*}
\MATRIX{rr}{a_{1} & b_{1}\\c_{1} & d_{1}}* \left(\MATRIX{rr}{a_{2} & b_{2}\\c_{2} & d_{2}}* \MATRIX{rr}{a_{3} & b_{3}\\c_{3} & d_{3}}\right) 
&= \MATRIX{rr}{a_{1} & b_{1}\\c_{1} & d_{1}}* \MATRIX{rr}{a_{2}a_{3} + b_{2}c_{3} & a_{2}b_{3} + b_{2}d_{3}\\c_{2}a_{3} + d_{2}c_{3} & c_{2}b_{3} + d_{2}d_{3}}\\
&= \MATRIX{rr}{a_{1}(a_{2}a_{3} + b_{2}c_{3}) + b_{1}(c_{2}a_{3} + d_{2}c_{3}) & a_{1}(a_{2}b_{3} + b_{2}d_{3}) + b_{2}(c_{2}b_{3} + d_{2}d_{3})\\
c_{1}(a_{2}a_{3} + b_{2}c_{3}) + d_{1}(c_{2}a_{3} + d_{2}c_{3}) & c_{1}(a_{2}b_{3} + b_{2}d_{3}) + d_{1}(c_{2}b_{3} + d_{2}d_{3})}\\
&= \MATRIX{rr}{a_{1}a_{2}a_{3} + a_{1}b_{2}c_{3} + b_{1}c_{2}a_{3} + b_{1}d_{2}c_{3} & a_{1}a_{2}b_{3} + a_{1}b_{2}d_{3} + b_{1}c_{2}b_{3} + b_{1}d_{2}d_{3}\\
c_{1}a_{2}a_{3} + c_{1}b_{2}c_{3} + d_{1}c_{2}a_{3} + d_{1}d_{2}c_{3} & c_{1}a_{2}b_{3} + c_{1}b_{2}d_{3} + d_{1}c_{2}b_{3} + d_{1}d_{2}d_{3}}\\
&= \MATRIX{rr}{(a_{1}a_{2} + b_{1}c_{2})a_{3} + (a_{1}b_{2} + b_{1}d_{2})c_{3} & (a_{1}a_{2} + b_{1}c_{2})b_{3} + (a_{1}b_{2} + b_{1}d_{2})d_{3}\\
(c_{1}a_{2} + d_{1}c_{2})a_{3} + (c_{1}b_{2} + d_{1}d_{2})c_{3} & (c_{1}a_{2} + d_{1}c_{2})b_{3} + (c_{1}b_{2} + d_{1}d_{2})d_{3}}\\
&= \MATRIX{rr}{ a_{1}a_{2} + b_{1}c_{2} & a_{1}b_{2} + b_{1}d_{2}\\ c_{1}a_{2} + d_{1}c_{2} & c_{1}b_{2} + d_{1}d_{2}}* \MATRIX{rr}{a_{3} & b_{3}\\c_{3} & d_{3}}\\
&=  \left(\MATRIX{rr}{a_{1} & b_{1}\\c_{1} & d_{1}}* \MATRIX{rr}{a_{2} & b_{2}\\c_{2} & d_{2}}\right)* \MATRIX{rr}{a_{3} & b_{3}\\c_{3} & d_{3}}
\end{align*}
\end{adjustwidth}
Therefore $(M_{2}(\R), *)$ is associative.  But note that $(M_{2}(\R), *)$ is not commutative, since
\begin{center}
$\MATRIX{rr}{1 & 2\\1 & 0} * \MATRIX{rr}{2 & 0\\1 & -1} = \MATRIX{rr}{1\cdot 2 + 2\cdot 1 & 1\cdot 0 + 2\cdot(-1)\\1\cdot 2 + 0\cdot 1 & 1\cdot 0 + 0\cdot (-1)} = \MATRIX{rr}{4 & -2\\ 2 & 0}$
\end{center}
but
\begin{center}
$\MATRIX{rr}{2 & 0\\1 & -1} * \MATRIX{rr}{1 & 2\\1 & 0} = \MATRIX{cc}{2 \cdot 1 + 0\cdot 1 & 2\cdot 2 + 0\cdot 0\\ 1\cdot 1 + (-1)\cdot 1 & 1\cdot 2 + (-1)\cdot 0} = \MATRIX{rr}{2 & 4\\0 & 2}$.
\end{center}
\end{enumerate}
\vsp

\p {\huge \S4.2 Problems}
\vsp

\begin{enumerate}[1.]
\item The set $2\Z := \SET{2n:n\in\Z} = \SET{0, \pm 2, \pm 4,\ldots}$ of even integers is a binary algebraic structure under addition.  Prove that $(\Z,+)\cong (2\Z,+)$.
\vsmsp

\solution Define $\varphi:\Z\to 2\Z$ by $\varphi(n) = 2n$ for every $n\in \Z$.
\vsmsp

First note that $\varphi$ is injective, since if $n_{1},n_{2}\in\Z$ such that $\varphi(n_{1}) = \varphi(n_{2})$, then $2n_{1} = 2n_{2}$ which forces $n_{1} = n_{2}$.  Next note that $\varphi$ is surjective, since if $m\in 2\Z$, then $m = 2n$ for some $n\in \Z$ and hence $\varphi(n) = 2n = m$.  So $\varphi$ is a bijection, and for any $n_{1},n_{2}\in \Z$, we have $\varphi(n_{1} + n_{2}) = 2(n_{1} + n_{2}) = 2n_{1} + 2n_{2} = \varphi(n_{1}) + \varphi(n_{2})$.  Hence $\varphi$ is an isomorphism and $(\Z,+)\cong (2\Z,+)$.
%
\item Let $(S_{1},*_{1})$, $(S_{2}, *_{2})$, and $(S_{3}, *_{3})$ be binary algebraic structures such that
\begin{center}
$(S_{1},*_{1})\cong(S_{2},*_{2})$ and $(S_{2},*_{2})\cong(S_{3},*_{3})$.
\end{center}
Prove that $(S_{1},*_{1})\cong(S_{3},*_{3})$.
\vsmsp

\solution By assumption, there exists $\varphi_{1}:S_{1}\to S_{2}$ and $\varphi_{2}:S_{2}\to S_{3}$.  Since $\varphi_{1}$ and $\varphi_{2}$ are bijections, so is $\varphi_{2}\circ \varphi_{1}:S_{1}\to S_{3}$.  And for any $a,b\in S_{1}$, we have

\begin{align*}
\varphi_{2}\circ \varphi_{1}(a *_{1} b) &= \varphi_{2}\big(\varphi_{1}(a*_{1}b)\big)\\
&= \varphi_{2}\big(\varphi_{1}(a)*_{2}\varphi_{1}(b)\big)\\
&= \varphi_{2}\big(\varphi_{1}(a)\big)*_{2} \varphi_{2}\big(\varphi_{1}(b)\big)\\
&= (\varphi_{2}\circ\varphi_{1})(a)*_{2}(\varphi_{2}\circ \varphi_{1})(b).
\end{align*}
Hence $\varphi_{2}\circ \varphi_{1}$ is an isomorphism and $(S_{1},*_{1})\cong(S_{3},*_{3})$.
%

\item Let $(S,*)$ and $(S',*')$ be binary algebraic structures and suppose that $\varphi:S\to S'$ is an isomorphism.  Prove that $\varphi^{-1}:S'\to S$ is also an isomorphism.
\note This is why it doesn't matter whether we write $(S,*)\cong (S',*')$ or $(S',*')\cong (S,*)$!
%
\vsmsp

\solution Let $x',y'\in S'$.  Then choose $x,y\in S$ such that $\varphi(x) = x'$ and $\varphi(y) = y'$.  Then
\begin{center}
$\varphi^{-1}(x'*'y') = \varphi^{-1}\big(\varphi(x)*'\varphi(y)\big) = \varphi^{-1}\big(\varphi(x*y)\big) = x*y = \varphi^{-1}(x')*'\varphi^{-1}(y')$.
\end{center}

\item Let $X$ be a non-empty set, and consider the following:
\begin{itemize}
\item The set $2^{X}$ of all functions $f:X\to\Z_{2}$ is a binary algebraic structure under the point-wise product $\cdot$, defined for any $f,g\in 2^{X}$ by
\begin{center}
$(f \cdot g)(x) = f(x)g(x)$ for all $x\in X$.
\end{center}
\item The set $\mathcal{P}(X)$ of all subsets of $X$ is a binary algebraic structure under the intersection operation $\cap$ defined by
\begin{center}
$A\cap B = \SET{x\in X:x\in A\text{ and }x\in B}$.
\end{center}
\end{itemize}
\begin{enumerate}[(a)]
\item What are the identity elements of $(2^{X}, \cdot)$ and $(\mathcal{P}(X), \cap)$?
\item Prove that $(2^{X}, \cdot)\cong (\mathcal{P}(X), \cap)$.
\end{enumerate}
%
\vsmsp

\solution
\begin{enumerate}[(a)]
\item The identity element of $(2^{X},\cdot)$ is the constant function $h:X\to\Z_{2}$ defined by
\begin{center}
$h(x) = 1$ for all $x\in X$.
\end{center}
Indeed for any $f\in 2^{X}$, and for any $x\in x$, the point-wise products
\begin{center}
$(f\cdot h)(x) = f(x)h(x) = f(x)\cdot 1 = f(x)$
\end{center}
and
\begin{center}
$(h\cdot f)(x) = h(x)f(x) = 1\cdot f(x) = f(x)$.
\end{center}
Hence $h\cdot f = f\cdot h = f$ for any $f\in 2^{X}$.
\item Define $\varphi:2^{X}\to \mathcal{P}(X)$ as follows.  For each $f\in 2^{X}$, set $A_{f} = \SET{x\in X:f(x) = 1}$.  Then define $\varphi(f) = A_{f}$.  We will check that $\varphi$ is an isomorphism.  First, check that $\varphi$ is injective.  If $f,g\in 2^{X}$ are chosen so that $\varphi(f) = \varphi(g)$, then $A_{f} = A_{g}$.  Suppose for a contradiction that there exists $x\in X$ such that $f(x)\neq g(x)$.  Then either $f(x) = 0$ and $g(x) = 1$ or $f(x) = 1$ and $g(x) = 0$.  In the former case, this would mean $x\in A_{g}$ but $x\not in A_{f}$.  In the latter case we would have $x \in A_{f}$ but $\notin A_{g}$.  Both cases contradict the fact that $A_{f} = A_{g}$.  Hence $f(x) = g(x)$ for all $x\in X$, that is, $f = g$.
\vsmsp

To see that $\varphi$ is surjective, let $A\in \mathcal{P}(X)$.  Define $f:X\to\Z_{2}$ by
\begin{center}
$f(x) = \begin{cases}1 & \text{ if }x\in A\\0 & \text{ otherwise}\end{cases}$,
\end{center}
and it follows immediately that $\varphi(f) = A$.  To see that $\varphi$ is operation perserving, note that for any $f,g\in 2^{X}$,
\begin{align*}
A_{fg} &= \SET{x\in X:(f\cdot g)(x) = 1}\\ &= \SET{x\in X:f(x)g(x) = 1}\\ &= \SET{x\in X:f(x) = 1\text{ and }g(x) =1}\\
&= A_{f}\cap A_{g}.
\end{align*}
It follows that $\varphi(f\cdot g) = A_{f\cdot g} = A_{f}\cap A_{g} = \varphi(f)\cap \varphi(g)$.
\end{enumerate}

\item Suppose $(S,*)$ and $(S',*')$ are binary algebraic structures, and that $\varphi:S\to S'$ is an isomorphism.
\begin{enumerate}[(a)]
\item Prove that if $(S,*)$ is commutative, then so is $(S',*')$.
\item Prove that if $(S,*)$ is associative, then so is $(S',*')$.
\item Prove that if $e\in S$ is an identity for $*$, then $\varphi(e)$ is an identity for $*'$.
\end{enumerate}
\vsmsp

\solution
\begin{enumerate}[(a)]
\item Suppose $(S,*)$ is commutative and let $x',y'\in S'$.  Choose $x,y\in S$ such that $\varphi(x) = x'$ and $\varphi(y) = y'$.
Then $x' *' y' = \varphi(x)*'\varphi(y) = \varphi(x*y) = \varphi(y*x) = \varphi(y)*'\varphi(x) = y'*'x'$.  Hence $(S',*')$ is commutative.
\item Suppose $(S,*)$ is associative and let $x',y',z'\in S'$.  Choose $x,y,z\in S$ such that $\varphi(x) = x'$, $\varphi(y) = y'$, and $\varphi(z) = z'$.  Then
\begin{align*}
x' *' (y'*'z') &= \varphi(x)*'\big(\varphi(y) *'\varphi(z)\big)\\
&= \varphi(x) *' \varphi(y*z)\\
&= \varphi\big(x*(y*z)\big)\\
&= \varphi\big((x*y)*z\big)\\
&= \varphi(x*y) *' \varphi(z)\\
&= \big(\varphi(x) *'\varphi(y)\big)*'\varphi(z)\\
&= (x' *' y')*'z',
\end{align*}
and it follows that $(S',*')$ is associative.
\item Let $x'\in S'$ and choose $x\in X$ such that $\varphi(x) = x'$.  Then
\begin{center}
$\varphi(e)*' x' = \varphi(e) *'\varphi(x) = \varphi(e*x) = \varphi(x) = x'$
\end{center}
and
\begin{center}
$x'*'\varphi(e) = \varphi(x)*'\varphi(e) = \varphi(x*e) = \varphi(x) = x'$.
\end{center}
Hence $\varphi(e)\in S$ is an identity for $*'$.
\end{enumerate}
\end{enumerate}
\vsp

\p {\huge \S4.3 Problems}
\vsp

\begin{enumerate}[1.]
\item Let $R$ be a ring and suppose $a,b,c\in R$.  Prove each of the following statements.
\begin{multicols}{2}
\begin{enumerate}[(a)]
\item $0_{R}\cdot a = 0_{R} = a\cdot 0_{R}$
\item $a(-b) = -ab = (-a)b$
\item $a(b-c) = ab - ac$
\item $-(-a) = a$
\item $-(a+b) = -a - b$
\item $-(a-b) = b - a$
\item $(-a)(-b) = ab$
\item[\ ] 
\end{enumerate}
\end{multicols}
\vsmsp

\solution
\begin{enumerate}[(a)]
\item $0_{R}a = (0_{R} + 0_{R})a = 0_{R}a + 0_{R}a$.  It follows that
\begin{center}
$0_{R}a = 0_{R}a + (0_{R} - 0_{R}) = (0_{R} + 0_{R}) - 0_{R} = 0_{R} - 0_{R} = 0_{R}$.
\end{center}
Using a similar argument, it may be checked that $a\cdot 0_{R} = 0_{R}$.
\item To see that $a(-b) = -ab$, check that
\begin{center}
$ab + a(-b) = a\big(b + (-b)\big) = a\cdot 0_{R} = 0_{R}$.
\end{center}
Using a similar argument, it may be checked that $(-a)b = -ab$.
\item $a(b-c) = a\big(b + (-c)\big) = ab + a(-c) = ab + \big(-(ac)\big) = ab - ac$
\item To see that $-(-a) = a$, check that $a + (-a) = 0_{R}$.
\item To see that $-(a+b) = -a - b$, check that
\begin{align*}
(a+b) + (-a - b) &= (a+b) + \big(-a + (-b)\big)\\
&= \big((a + b) + (-a)\big) + (-b)\\
&= \big((b + a) + (-a)\big) + (-b)\\
&= \big(b + (a + (-a))\big) + (-b)\\
&= (b + 0_{R}) + (-b)\\
&= b + (-b)\\
&= 0_{R}
\end{align*}
\item $-(a-b) = -\big(a + (-b)\big) = -a + \big(-(-b)\big) = -a + b = b + (-a) = b-a$
\item $(-a)(-b) = -\big(a(-b)\big) = -\big(-(ab)\big) = ab$
\end{enumerate}

\item Prove that if $R$ is unital, then $(-1_{R})a = -a$ for all $a\in R$.
\vsmsp

\solution

To see that $(-1_{R})a = -a$, check that
\begin{center}
$a + (-1_{R}a) = 1_{R}a + (-1_{R}a) = \big(1_{R} + (-1_{R})\big)a = 0_{R}a = 0_{R}$.
\end{center}

\item Recall that the set $M_{2}(\R)$ of all $2\times 2$ matrices of the form
\begin{center}
$\MATRIX{rr}{a & b\\c & d}$, for some $a,b,c,d\in\R$,
\end{center}
is a binary algebraic structure under both addition and multiplication, defined by
\begin{center}
$\MATRIX{rr}{a_1 & b_1\\c_1 & d_1} + \MATRIX{rr}{a_2 & b_2\\c_2 & d_2} = \MATRIX{cc}{a_1 + a_2 & b_1 + b_2\\c_1 + c_2& d_1 + d_2}$
\end{center}
and
\begin{center}
$\MATRIX{rr}{a_1 & b_1\\c_1 & d_1}\cdot \MATRIX{rr}{a_2 & b_2\\c_2 & d_2} = \MATRIX{cc}{a_1a_2 + b_1b_2 & a_1c_2 + b_1d_2\\c_1a_2 + d_1b_2 & c_1b_2 + d_1d_2}$,
\end{center}
for all $a_{1},a_{2},b_{1},b_{2},c_{1},c_{2},d_{1},d_{2}\in\R$.
\vsp
\begin{enumerate}[(a)]
\item Prove that $M_{2}(\R)$ is a ring.
\item Prove that $M_{2}(\R)$ is not commutative.
\item Prove that $M_{2}(\R)$ is unital with multiplicative identity given by $\MATRIX{rr}{1 & 0\\0 & 1}$.
\end{enumerate}
\vsmsp

\solution 
\begin{enumerate}[(a)]
\item To show that $(M_{2}(\R), +, \cdot)$ is a ring, we must check that for all $A,B,C\in M_{2}(\R)$, that
\begin{enumerate}[(1)]
\item $A + B  = B + A$ for all $A,B\in M_{2}(\R)$.
\item $A + (B + C) = (A + B) + C$ for all $A,B,C \in M_{2}(\R)$.
\item There exists $0_{M_{2}(R)}\in M_{2}(\R)$ such that $A + 0 = A$ for all $A\in M_{2}(\R)$.
\item For any $A\in M_{2}(\R)$, there exists $B\in M_{2}(\R)$ such that $A + B = 0_{M_{2}(\R)}$.
\item $A(BC) = (AB)C$ for all $A,B,C \in M_{2}(\R)$.
\item $A(B+C) = AB + AC$ for all $A,B,C \in M_{2}(\R)$.
\item $(A +B)C = AC + BC$ for all $A,B,C \in M_{2}(\R)$.
\end{enumerate}
Note that (1), (2), and (5) were already checked in \S4.1 Problem \ref{m2rbas}
\vsmsp

To see (3), note that for any $\MATRIX{rr}{a & b\\c & d}\in M_{2}(\R)$, we have
\begin{center}
$\MATRIX{rr}{a & b\\c & d} + \MATRIX{rr}{0 & 0\\0 & 0} = \MATRIX{rr}{a + 0 & b + 0\\c + 0 & d + 0} = \MATRIX{rr}{a & b\\c & d}$,
\end{center}
and hence $0_{M_{2}(\R)} = \MATRIX{rr}{0 & 0\\0 & 0}$.

To see (4), note that for any $A = \MATRIX{rr}{a & b\\c & d}\in M_{2}(\R)$, we may take $B = \MATRIX{rr}{-a & -b\\-c & -d}\in M_{2}(\R)$ and get
\begin{center}
$A + B = \MATRIX{rr}{a & b\\c & d} + \MATRIX{rr}{-a & -b\\-c & -d} = \MATRIX{rr}{a + (-a) & b + (-b)\\c + (-c) & d + (-d)} = \MATRIX{rr}{0 & 0\\0 & 0} = 0_{M_{2}(\R)}$.
\end{center}
\item 
 
\item 

\end{enumerate}

To see (6), let $A = \MATRIX{rr}{a_{1} & b_{1}\\c_{1} & d_{1}}, B = \MATRIX{rr}{a_{2} & b_{2}\\c_{2} & d_{2}}$, and $C = \MATRIX{rr}{a_{3} & b_{3}\\c_{3} & d_{3}}$ and compute
\begin{align*}
A(B+C) &= \MATRIX{rr}{a_{1} & b_{1}\\c_{1} & d_{1}} \cdot\left(\MATRIX{rr}{a_{2} & b_{2}\\c_{2} & d_{2}} + \MATRIX{rr}{a_{3} & b_{3}\\c_{3} & d_{3}}\right)\\
&= \MATRIX{rr}{a_{1} & b_{1}\\c_{1} & d_{1}} \cdot \MATRIX{rr}{a_{2} + a_{3} & b_{2} + b_{3}\\c_{2} + c_{3} & d_{2} + d_{3}}\\
&= \MATRIX{rr}{a_{1}(a_{2} + a_{3}) + b_{1}(c_{2} + c_{3}) & a_{1}(b_{2} + b_{3}) + b_{1}(d_{2} + d_{3})\\c_{1}(a_{2} + a_{3}) + d_{1}(c_{2} + c_{3}) & c_{1}(b_{2} + b_{3}) + d_{1}(d_{2} + d_{3})}\\
&= \MATRIX{rr}{a_{1}a_{2} + a_{1}a_{3} + b_{1}c_{2} + b_{1}c_{3} & a_{1}b_{2} + a_{1}b_{3} + b_{1}d_{2} + b_{1}d_{3}\\c_{1}a_{2} + c_{1}a_{3} + d_{1}c_{2} + d_{1}c_{3} & c_{1}b_{2} + c_{1}b_{3} + d_{1}d_{2} + d_{1}d_{3}}\\
&= \MATRIX{rr}{a_{1}a_{2} + b_{1}c_{2} & a_{1}b_{2} + b_{1}d_{2}\\c_{1}a_{2} + d_{1}c_{2} & c_{1}b_{2} + d_{1}d_{2}} + \MATRIX{rr}{a_{1}a_{3} +  b_{1}c_{3} & a_{1}b_{3} + b_{1}d_{3}\\c_{1}a_{3} + d_{1}c_{3} & c_{1}b_{3} + d_{1}d_{3}}\\
&= \MATRIX{rr}{a_{1} & b_{1}\\c_{1} & d_{1}}\cdot \MATRIX{rr}{a_{2} & b_{2}\\c_{2} & d_{2}} + \MATRIX{rr}{a_{1} & b_{1}\\c_{1} & d_{1}}\cdot \MATRIX{rr}{a_{3} & b_{3}\\c_{3} & d_{3}}\\
&= AB + AC
\end{align*}

Property (7) may be checked in a similar fashion.

\item Recall that the set $F(\R)$ of all functions $f:\R\to\R$ is a binary algebraic structure under both of the pointwise addition and multiplication operations, defined for $f,g\in F(\R)$ by
\begin{center}
$(f+g)(x) = f(x) + g(x)$ and $(fg)(x) = f(x)g(x)$ for all $x\in \R$.
\end{center}
\begin{enumerate}[(a)]
\item Prove that $F(\R)$ is a ring.
\item Prove that $F(\R)$ is commutative.
\item Prove that $F(\R)$ is unital.
\end{enumerate}

\item Is the zero ring unital?

\item Show that $(F(\R),+,\circ)$ fails to be a ring if $+$ and $\circ$ denote point-wise addition and function composition, respectively.
\item Prove that $\Z[\sqrt{3}] = \SET{a + b\sqrt{3}:a,b\in\Z}$ is a subring of $\Q$.
\item Prove that the subset $S = \SET{\MATRIX{rr}{a & b\\-b & a}:a,b\in \R}$ of $M_{2}(\R)$ is a subring.
\item Is $\Q[\sqrt[3]{2}] = \SET{a + b\sqrt[3]{2} + c\sqrt[3]{4}:a,b,c\in\Q}$ a subring of $\R$?
\item Let $R$ and $S$ be rings.
\begin{enumerate}[(a)]
\item Prove that $R\times S$ is commutative if and only if both $R$ and $S$ are.
\item Prove that $R\times S$ is unital if and only if both $R$ and $S$ are.
\end{enumerate}
\end{enumerate}
\vsp

\newpage

\p {\huge \S4.4 Problems}
\vsp

\begin{enumerate}[1.]
\item Show that $\MATRIX{rr}{1 & 2\\2 & 4}$ is a zero divisor in $M_{2}(\REAL)$.

\item Draw a Venn diagram to illustrate the inclusion between rings, commutative rings, unital rings, integral domains, and fields.

\item Let $R$ and $S$ be rings.  Prove that $R\times S$ is an integral domain if and only if $R$ and $S$ are.

\item Let $R$ be a unital ring.  Is the set $U = \SET{a\in R:a\text{ is a unit}}$ a subring of $R$?

\item Let $R$ be a ring and suppose $a,b\in R$.  Prove the following statements.
\begin{enumerate}[(a)]
\item There exists a unique $x\in R$ such that $a + x = b$.
\item If $a$ is a unit, there exists a unique $x\in R$ such that $ax=b$.   
\end{enumerate}

\item Suppose $F$ is a field and let $a,b\in F$.  Prove that if $a$ is non-zero, there exists a unique $x\in F$ such that $ax = b$.

\end{enumerate}
\end{document}





