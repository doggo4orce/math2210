\documentclass[11pt,fleqn,dvipsnames,usenames]{article}

% to keep this file less overwhelming
% packages to include

\usepackage[dvipsnames, table]{xcolor}

\usepackage{
  amsthm,
  amsmath,
  amssymb, 
  arydshln, % for hyphenated lines in block matrices
  fancyhdr, % needed for header at top of each page
  graphicx, % to include pictures
  mathtools, % for a longer arrow
  multicol, % displaying enumerates and itemizes into multiple columns
  multirow, % for tables
  multido, % for TOC
  pgfplots, % for axis environment within tikz pictures
  systeme,
  tikz,
}

\usepackage[utf8]{inputenc}
\usepackage{color,soul}

\usepackage[inline, shortlabels]{enumitem}
\usepackage[hidelinks]{hyperref}


% global constants
\newcommand{\term}{Winter 2024}

% mathbb aliases
\newcommand{\COMPLEX}{\mathbb{C}}
\newcommand{\REAL}{\mathbb{R}}
\newcommand{\NATURAL}{\mathbb{N}}
\newcommand{\INTEGER}{\mathbb{Z}}

% for financial stuff
\newcommand{\dollar}{\mathrm{\$}}

% nicer looking trig functions
\newcommand{\SIN}[1]{\sin\left(#1\right)}
\newcommand{\COS}[1]{\cos\left(#1\right)}
\newcommand{\TAN}[1]{\tan\left(#1\right)}
\newcommand{\CSC}[1]{\csc\left(#1\right)}
\newcommand{\SEC}[1]{\sec\left(#1\right)}
\newcommand{\COT}[1]{\cot\left(#1\right)}

% automatically resize set brackets
\newcommand{\SET}[1]{\left\{#1\right\}}

% sums and products
\newcommand{\SUM}{\displaystyle\sum\limits}
\newcommand{\PROD}{\displaystyle\prod\limits}
\newcommand{\of}{\circ}
\newcommand{\restrict}[1]{\raisebox{-.5ex}{$|$}_{#1}}

% set intersection and union
\newcommand{\CAP}{\displaystyle\bigcap\limits}
\newcommand{\CUP}{\displaystyle\bigcup\limits}

% max and min
\newcommand{\MAX}[1]{\ensuremath{\max\left(#1\right)}}
\newcommand{\MIN}[1]{\ensuremath{\min\left(#1\right)}}

% for writing logic within mathematics environment
\newcommand{\FORALL}{\ensuremath{\text{ for all }}}
\newcommand{\FORSOME}{\ensuremath{\text{ for some }}}

% matrix notation
\newcommand{\MATRIX}[2]{\ensuremath{\left[\begin{array}{#1}#2\end{array}\right]}}
\newcommand{\COLUMN}[1]{\ensuremath{\left[\begin{array}{r}#1\end{array}\right]}}

% vector notation
%\newcommand{\vv}{\overset{\rightharpoonup}}
\newcommand{\vv}[1]{{\bf #1}}
\newcommand{\arr}{\overrightarrow}

% dot product
\newcommand{\dotp}{{\scriptstyle\bullet}}

% Text macros
\newcommand{\KER}[1]{\ensuremath{\text{ker}\left(#1\right)}}
\newcommand{\IMG}[1]{\ensuremath{\text{im}\left(#1\right)}}
\newcommand{\CHAR}[1]{\ensuremath{\text{char}\left(#1\right)}}
\newcommand{\BIGO}[1]{\ensuremath{\mathcal{O}\left(#1\right)}}
\newcommand{\TR}[1]{\ensuremath{\text{tr}\left(#1\right)}}

% abbreviations
\newcommand{\ds}{\displaystyle}
\newcommand{\md}{\mdseries}
\newcommand{\vsp}{\vspace{0.5cm}}
\newcommand{\smsp}{\vspace{0.25cm}}
\newcommand{\hsp}{\hspace{0.25cm}}

% new operators
\DeclareMathOperator\SPAN{Span}
\newcommand{\SPANOF}[1]{\ensuremath{\SPAN\left\{#1\right\}}}
\DeclareMathOperator\PROJ{proj}
\DeclareMathOperator\PERP{perp}

% quick abbreviations to avoid using latex environments
\newcommand{\answer}{\noindent \textbf{Answer:} }
\newcommand{\answers}{\noindent \textbf{Answers:} }
\newcommand{\application}{\noindent \textbf{Application:} }
\newcommand{\caution}{\noindent \textbf{Caution:} }
\newcommand{\conclusion}{\noindent \textbf{Conclusion:} }
\newcommand{\consequence}{\noindent \textbf{Consequence:} }
\newcommand{\defn}{\noindent \textbf{Definition:} }
\newcommand{\details}{\noindent \textbf{Details:} }
\newcommand{\example}{\noindent \textbf{Example:} }
\newcommand{\examples}{\noindent \textbf{Examples:} }
\newcommand{\exception}{\noindent \textbf{Exception:} }
\newcommand{\exercise}{\noindent \textbf{Exercise:} }
\newcommand{\exercises}{\noindent \textbf{Exercises:} }
\newcommand{\fact}{\noindent \textbf{Fact:} }
\newcommand{\facts}{\noindent \textbf{Facts:} }
\newcommand{\formula}{\noindent \textbf{Formula:} }
\newcommand{\goal}{\noindent \textbf{Goal:} }
\newcommand{\goals}{\noindent \textbf{Goals:} }
\newcommand{\hint}{\noindent \textbf{Hint:} }
\newcommand{\idea}{\noindent \textbf{Idea:} }
\newcommand{\illustration}{\noindent \textbf{Illustration:} }
\newcommand{\important}{\noindent \textbf{Important:} }
\newcommand{\midea}{\noindent \textbf{Main Idea:} }
\newcommand{\motivation}{\noindent \textbf{Motivation:} }
\newcommand{\nthm}[1]{\noindent \textbf{Theorem} (\textit{#1}):}
\newcommand{\notation}{\noindent \textbf{Notation:} }
\newcommand{\note}{\noindent \textbf{Note:} }
\newcommand{\notes}{\noindent \textbf{Notes:} }
\newcommand{\observation}{\noindent \textbf{Observation:} }
\newcommand{\observations}{\noindent \textbf{Observations:} }
\newcommand{\pict}{\noindent \textbf{Picture:} }
\newcommand{\plan}{\noindent \textbf{Plan:} }
\newcommand{\prf}{\noindent \textbf{Proof:} }
\newcommand{\problem}{\noindent \textbf{Problem:} }
\newcommand{\properties}{\noindent \textbf{Properties:} }
\newcommand{\question}{\noindent \textbf{Question:} }
\newcommand{\questions}{\noindent \textbf{Questions:} }
\newcommand{\recall}{\noindent \textbf{Recall:} }
\newcommand{\reason}{\noindent \textbf{Reason:} }
\newcommand{\remark}{\noindent \textbf{Remark:} }
\newcommand{\remarks}{\noindent \textbf{Remarks:} }
\newcommand{\reminder}{\noindent \textbf{Reminder:} }
\newcommand{\solution}{\noindent \textbf{Solution:} }
\newcommand{\nsolution}[1]{\noindent \textbf{Solution #1:} }
\newcommand{\strategy}{\noindent \textbf{Strategy:} }
\newcommand{\summary}{\noindent \textbf{Summary:} }
\newcommand{\terminology}{\noindent \textbf{Terminology:} }
\newcommand{\thm}{\noindent \textbf{Theorem:} }
\newcommand{\work}{\noindent \textbf{Work:} }


\usepackage[version=4]{mhchem}

% Where to look for pngs and jpegs
\graphicspath{{Images//}}

\usepackage[includehead, includefoot, left= 2cm, top =1.5cm, bottom = 1.5cm, textwidth=17.5cm]{geometry}

\usepackage{pifont, amsmath}

\pagestyle{fancy}
\fancyhf{}
\renewcommand{\headrulewidth}{1pt}
%\fancyhead[R]{\bfseries\sffamily\thepage}
%\fancyfoot[C]{\bfseries\sffamily\thepage}
\fancyhead[L]{\nouppercase{\bfseries\sffamily\leftmark}}

% used when adding fill-in-the-blanks for students
\newcommand{\blank}[1]{\underline{\hspace{#1}}}

% indents annoy me, and so does repeatedly typing \noindent
\newcommand{\p}{\noindent}

\begin{document}

\fancyhead[L]{\course}
\fancyhead[C]{\includegraphics[width=5cm, trim= 0 0.4cm 0 0]{TRU_logo}}
\fancyhead[R]{\term}
\renewcommand{\headrulewidth}{0.4pt}

{\huge Chapter 1}
\vsp

\begin{enumerate}
\item Suppose $a,b,c\in\Z$ such that $a|b$ and $b|c$.  Prove that $a|c$.
\item Suppose that $a,b,c\in\Z$ are chosen so that $b|ac$.  Is it necessarily the case that either $b|a$ or $b|c$?
\item If $b|a+c$ for some $a,b,c\in \Z$, does it automatically follow that either $b|a$ or $b|c$?
\item Prove each of the following statements.
\begin{enumerate}[(a)]
\item If $b|a$, then $b|sa$ for any $s\in\INTEGER$.
\item If $b|a$ and $b|c$, then $b|sa+tc$ for any $s,t\in\INTEGER$.
\end{enumerate}
\item Suppose $a,b,c,d\in\Z$ such that $a|c$ and $b|d$.  Prove that $ab|cd$.
\item Prove that the square of any integer is either of the form $4k$ or $4k+1$ for some integer $k$.
\item Prove that the cube of any integer $a$ is of the form $9k, 9k+1$, or $9k+8$ for some $k\in\INTEGER$.
\item If $a|c$ and $b|c$ with $\GCD{a,b} = 1$, prove that $ab|c$.\label{abdividesc}
\item Show that the statement in Problem \ref{abdividesc} fails to hold in general if $\GCD{a,b}\neq 1$.
\item Calculate $\GCD{229,66}$ and find $s,t\in\Z$ such that $\GCD{229,66} = 229s + 66t$.
\item If $a,b\in\Z$ are not both zero and $c\in\Z^{+}$, prove that $\GCD{ca,cb} = c\cdot \GCD{a,b}$.
\item If $n\in \Z$, what are the possible values of $\GCD{n, n+1}$?
\item Let $a,b\in\Z$ be non-zero.  Show that if $k\in\INTEGER$ such that $a|k$ and $b|k$, then $\LCM{a, b}|k$.
\item Suppose $a,b\in\Z$ such that $a^2|b^2$.  Prove that $a|b$.
%\item If $3|a^2 + b^2$, prove that $3|a$ and $3|b$.
%\item If $x,y\in\Q$, verify that $x+y, x-y, xy\in \Q$.
%\item Show that if $x,y\in\Q$ with $y\neq 0$, then $x/y\in\Q$.
\item Prove that $\sqrt[3]{2}\notin\Q$.
%\item Suppose that $a\in\Z$ with $\sqrt{a}\in \Q$.  Prove that $\sqrt{a}\in\Z$.
\end{enumerate}
\vsp

{\huge Chapter 2}
\vsp

\begin{enumerate}
\item Write $\ds{\frac{2 + 3i}{1 + 2i} + \frac{8+i}{2-i}}$ in the form $a+bi$ for some $a,b\in\R$.
\item Prove the following properties hold in $\C$.  Assume that the corresponding properties hold in $\R$.
\begin{enumerate}[(a)]
\item $w + z = z + w$ for all $w,z\in\C$.\hfill (commutativity of addition)
\item $w + (z + \lambda) = (w + z) + \lambda$ for all $w,z,\lambda\in\C$.\hfill (associativity of addition)
\item $wz = zw$ for all $w,z\in\C$.\hfill (commutativity of multiplication)
\item $w(z\lambda) = (wz)\lambda$ for all $w,z,\lambda\in\C$.\hfill (associativity of multiplication)
\item $w(z + \lambda) = wz + w\lambda$ for all $w,z,\lambda\in\C$.\hfill (distributivity)
\end{enumerate}
\item Prove that $\RE{iz} = -\IM{z}$ for every $z\in\C$.
\item Show that for any $z_{1},z_{2}\in\COMPLEX$, we have $\CC{z_{1} + z_{2}} = \CC{z_{1}} + \CC{z_{2}}$ and $\CC{z_{1}z_{2}} = \CC{z_{1}}\cdot \CC{z_{2}}$.
\item Compute $3i^{11} + 6i^{3} + \frac{8}{i^{20}} + i^{-1}$.
\item If $z\in\C$ with $|z| = 1$ and $z\neq 1$, then check that $\RE{\dfrac{1}{1-z}} = \dfrac{1}{2}$.
\item Show that if $\big(\CC{z}\big)^2 = z^2$, then $z$ is either real or purely imaginary.
%\item Be sure to provide a complete justification when answering each of the following questions.
%\begin{enumerate}[(a)]
%\item Does there exist $a_{0},a_{1},a_{2},a_{3}\in \R$ such that
%\begin{center}
%$3(z-2i)(z+i)(z-2) = a_{0}(z-a_{1})(z - a_{2})(z - z_{3})$?
%\end{center}
%\item Does there exist $z_{0},z_{1},z_{2},z_{3},z_{4},z_{5}\in \C$ such that
%\begin{center}
%$3z^5 - 2iz^4 + \pi z^2 - i = z_{0}(z - z_{1})(z - z_{2})(z - z_{3})(z - z_{4})(z - z_{5})$?
%\end{center}
%\end{enumerate}
\item Find a polar coordinate representation for each of the following complex numbers.
\begin{multicols}{3}
\begin{enumerate}[(a)]
\item $-6$
\item $\sqrt{3} + i$
\item $-2-2i$
\end{enumerate}
\end{multicols}
\item The following ordered pairs are each polar coordinate representations of a complex number.  Write the corresponding complex number in the form $a + bi$.
\begin{multicols}{3}
\begin{enumerate}[(a)]
\item $(3,\pi/6)$
\item $(2,3\pi/2)$
\item $(1,7\pi/4)$
\end{enumerate}
\end{multicols}
\item Calculate $\dfrac{\big(\sqrt{2} + \sqrt{6}i\big)^5}{\big(1+i\big)^9}$.
\item Prove that for any $z\in\C$, $|z^{n}| = |z|^{n}$.
\item Is it true that $\PARG{z_1z_2} = \PARG{z_1} + \PARG{z_2}$ for every $z_{1},z_{2}\in\C$?
\item Find all the $8$th roots of unity, and sketch them in the complex plane.
\item Let $k\in\Z^{+}$ and suppose that $\zeta$ is a primitive $n$th root of unity.  Show that $\zeta^{k}$ is a primitive $n$th root of unity if and only if $\GCD{n,k} = 1$.
\end{enumerate}
\vsp

{\huge Chapter 3}
\vsp

\begin{enumerate}
\item Prove that if $a\overset{n}{\equiv}b$ and $c\overset{n}{\equiv}d$, then $a-c\overset{n}{\equiv}b-d$.
\item Prove that if $a\overset{n}{\equiv}b$, then $a^2 + b^2\overset{n^2}{\equiv}2ab$.
\item Which of the following congruences have solutions?  Find as many as you can.
\begin{enumerate}[(a)]
\item $x^2\overset{3}{\equiv}1$
\item $x^2\overset{7}{\equiv}2$
\item $x^2\overset{11}{\equiv}3$
\end{enumerate}
%\item Suppose $a,b\in\Z$ such that $a\overset{p}{\equiv}b$ for every prime $p$.  Prove that $a=b$.
\item Decide whether each of the following statements are true.  If they are true, then prove them.  Otherwise, provide a counter-example.
\begin{enumerate}[(a)]
\item If $a^2\overset{n}{\equiv}b^2$, then either $a\overset{n}{\equiv}b$ or $a\overset{n}{\equiv}-b$.
\item If $p$ is prime and $a^2\overset{p}{\equiv}b^2$, then either $a\overset{p}{\equiv}b$ or $a\overset{p}{\equiv}-b$.
\item If $[a] = [b]$ in $\ZN$, then $\GCD{a,n} = \GCD{b,n}$.
\end{enumerate}
\item Let $n\in\Z^{+}$.  If $a\in \Z$ such that $[a] = [1]$ in $\ZN$, prove that $\GCD{a,n} = 1$.  Demonstrate also that the converse fails to be true in general.
\item Let $A$ be a set and let $\mathcal{P}$ be a collection of subsets of $A$ satisfying the following two properties:
\begin{itemize}[(1)]
\item For any $a\in A$, there exists $U\in\mathcal{P}$ such that $a\in U$.  That is, $\mathcal{P}$ \textbf{covers} $A$.
\item For any $A,B\in\mathcal{P}$, either $A = B$ or $A\cap B = \emptyset$.  That is, $\mathcal{P}$ is \textbf{pairwise disjoint}.
\end{itemize}
\end{enumerate}
\end{document}





