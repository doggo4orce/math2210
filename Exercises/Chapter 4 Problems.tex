\documentclass[11pt,fleqn,dvipsnames,usenames]{article}

% to keep this file less overwhelming
% packages to include

\usepackage[dvipsnames, table]{xcolor}

\usepackage{
  amsthm,
  amsmath,
  amssymb, 
  arydshln, % for hyphenated lines in block matrices
  fancyhdr, % needed for header at top of each page
  graphicx, % to include pictures
  mathtools, % for a longer arrow
  multicol, % displaying enumerates and itemizes into multiple columns
  multirow, % for tables
  multido, % for TOC
  pgfplots, % for axis environment within tikz pictures
  systeme,
  tikz,
}

\usepackage[utf8]{inputenc}
\usepackage{color,soul}

\usepackage[inline, shortlabels]{enumitem}
\usepackage[hidelinks]{hyperref}


% global constants
\newcommand{\term}{Winter 2024}

% mathbb aliases
\newcommand{\COMPLEX}{\mathbb{C}}
\newcommand{\REAL}{\mathbb{R}}
\newcommand{\NATURAL}{\mathbb{N}}
\newcommand{\INTEGER}{\mathbb{Z}}

% for financial stuff
\newcommand{\dollar}{\mathrm{\$}}

% nicer looking trig functions
\newcommand{\SIN}[1]{\sin\left(#1\right)}
\newcommand{\COS}[1]{\cos\left(#1\right)}
\newcommand{\TAN}[1]{\tan\left(#1\right)}
\newcommand{\CSC}[1]{\csc\left(#1\right)}
\newcommand{\SEC}[1]{\sec\left(#1\right)}
\newcommand{\COT}[1]{\cot\left(#1\right)}

% automatically resize set brackets
\newcommand{\SET}[1]{\left\{#1\right\}}

% sums and products
\newcommand{\SUM}{\displaystyle\sum\limits}
\newcommand{\PROD}{\displaystyle\prod\limits}
\newcommand{\of}{\circ}
\newcommand{\restrict}[1]{\raisebox{-.5ex}{$|$}_{#1}}

% set intersection and union
\newcommand{\CAP}{\displaystyle\bigcap\limits}
\newcommand{\CUP}{\displaystyle\bigcup\limits}

% max and min
\newcommand{\MAX}[1]{\ensuremath{\max\left(#1\right)}}
\newcommand{\MIN}[1]{\ensuremath{\min\left(#1\right)}}

% for writing logic within mathematics environment
\newcommand{\FORALL}{\ensuremath{\text{ for all }}}
\newcommand{\FORSOME}{\ensuremath{\text{ for some }}}

% matrix notation
\newcommand{\MATRIX}[2]{\ensuremath{\left[\begin{array}{#1}#2\end{array}\right]}}
\newcommand{\COLUMN}[1]{\ensuremath{\left[\begin{array}{r}#1\end{array}\right]}}

% vector notation
%\newcommand{\vv}{\overset{\rightharpoonup}}
\newcommand{\vv}[1]{{\bf #1}}
\newcommand{\arr}{\overrightarrow}

% dot product
\newcommand{\dotp}{{\scriptstyle\bullet}}

% Text macros
\newcommand{\KER}[1]{\ensuremath{\text{ker}\left(#1\right)}}
\newcommand{\IMG}[1]{\ensuremath{\text{im}\left(#1\right)}}
\newcommand{\CHAR}[1]{\ensuremath{\text{char}\left(#1\right)}}
\newcommand{\BIGO}[1]{\ensuremath{\mathcal{O}\left(#1\right)}}
\newcommand{\TR}[1]{\ensuremath{\text{tr}\left(#1\right)}}

% abbreviations
\newcommand{\ds}{\displaystyle}
\newcommand{\md}{\mdseries}
\newcommand{\vsp}{\vspace{0.5cm}}
\newcommand{\smsp}{\vspace{0.25cm}}
\newcommand{\hsp}{\hspace{0.25cm}}

% new operators
\DeclareMathOperator\SPAN{Span}
\newcommand{\SPANOF}[1]{\ensuremath{\SPAN\left\{#1\right\}}}
\DeclareMathOperator\PROJ{proj}
\DeclareMathOperator\PERP{perp}

% quick abbreviations to avoid using latex environments
\newcommand{\answer}{\noindent \textbf{Answer:} }
\newcommand{\answers}{\noindent \textbf{Answers:} }
\newcommand{\application}{\noindent \textbf{Application:} }
\newcommand{\caution}{\noindent \textbf{Caution:} }
\newcommand{\conclusion}{\noindent \textbf{Conclusion:} }
\newcommand{\consequence}{\noindent \textbf{Consequence:} }
\newcommand{\defn}{\noindent \textbf{Definition:} }
\newcommand{\details}{\noindent \textbf{Details:} }
\newcommand{\example}{\noindent \textbf{Example:} }
\newcommand{\examples}{\noindent \textbf{Examples:} }
\newcommand{\exception}{\noindent \textbf{Exception:} }
\newcommand{\exercise}{\noindent \textbf{Exercise:} }
\newcommand{\exercises}{\noindent \textbf{Exercises:} }
\newcommand{\fact}{\noindent \textbf{Fact:} }
\newcommand{\facts}{\noindent \textbf{Facts:} }
\newcommand{\formula}{\noindent \textbf{Formula:} }
\newcommand{\goal}{\noindent \textbf{Goal:} }
\newcommand{\goals}{\noindent \textbf{Goals:} }
\newcommand{\hint}{\noindent \textbf{Hint:} }
\newcommand{\idea}{\noindent \textbf{Idea:} }
\newcommand{\illustration}{\noindent \textbf{Illustration:} }
\newcommand{\important}{\noindent \textbf{Important:} }
\newcommand{\midea}{\noindent \textbf{Main Idea:} }
\newcommand{\motivation}{\noindent \textbf{Motivation:} }
\newcommand{\nthm}[1]{\noindent \textbf{Theorem} (\textit{#1}):}
\newcommand{\notation}{\noindent \textbf{Notation:} }
\newcommand{\note}{\noindent \textbf{Note:} }
\newcommand{\notes}{\noindent \textbf{Notes:} }
\newcommand{\observation}{\noindent \textbf{Observation:} }
\newcommand{\observations}{\noindent \textbf{Observations:} }
\newcommand{\pict}{\noindent \textbf{Picture:} }
\newcommand{\plan}{\noindent \textbf{Plan:} }
\newcommand{\prf}{\noindent \textbf{Proof:} }
\newcommand{\problem}{\noindent \textbf{Problem:} }
\newcommand{\properties}{\noindent \textbf{Properties:} }
\newcommand{\question}{\noindent \textbf{Question:} }
\newcommand{\questions}{\noindent \textbf{Questions:} }
\newcommand{\recall}{\noindent \textbf{Recall:} }
\newcommand{\reason}{\noindent \textbf{Reason:} }
\newcommand{\remark}{\noindent \textbf{Remark:} }
\newcommand{\remarks}{\noindent \textbf{Remarks:} }
\newcommand{\reminder}{\noindent \textbf{Reminder:} }
\newcommand{\solution}{\noindent \textbf{Solution:} }
\newcommand{\nsolution}[1]{\noindent \textbf{Solution #1:} }
\newcommand{\strategy}{\noindent \textbf{Strategy:} }
\newcommand{\summary}{\noindent \textbf{Summary:} }
\newcommand{\terminology}{\noindent \textbf{Terminology:} }
\newcommand{\thm}{\noindent \textbf{Theorem:} }
\newcommand{\work}{\noindent \textbf{Work:} }


\usepackage[version=4]{mhchem}

% Where to look for pngs and jpegs
\graphicspath{{Images//}}

\usepackage[includehead, includefoot, left= 2cm, top =1.5cm, bottom = 1.5cm, textwidth=17.5cm]{geometry}

\usepackage{pifont, amsmath}

\pagestyle{fancy}
\fancyhf{}
\renewcommand{\headrulewidth}{1pt}
%\fancyhead[R]{\bfseries\sffamily\thepage}
%\fancyfoot[C]{\bfseries\sffamily\thepage}
\fancyhead[L]{\nouppercase{\bfseries\sffamily\leftmark}}

% used when adding fill-in-the-blanks for students
\newcommand{\blank}[1]{\underline{\hspace{#1}}}

% indents annoy me, and so does repeatedly typing \noindent
\newcommand{\p}{\noindent}

\begin{document}

\fancyhead[L]{\course}
\fancyhead[C]{\includegraphics[width=5cm, trim= 0 0.4cm 0 0]{TRU_logo}}
\fancyhead[R]{\term}
\renewcommand{\headrulewidth}{0.4pt}

{\huge \S4.1 Problems}
\vsp

\begin{enumerate}[1.]
\item Is $\Q$ a binary algebraic structure under the operation $*$ defined by
\begin{center}
$\dfrac{a}{b}*\dfrac{c}{d} = \dfrac{a}{d} + \dfrac{c}{b}$
\end{center}
for all $\dfrac{a}{b},\dfrac{c}{d}\in \Q$?    If not, describe the problem completely.
\item Let $\Q^{+} = \SET{x\in\Q:x > 0}$.  Define the binary operation on $\Q^{*}$ by
\begin{center}
$x*y = \sqrt{xy}$, for all $x,y\in\Q^{+}$.
\end{center}
Is $(\Q^{+},*)$ a binary algebraic structure?

\item Recall that $M_{2}(\R)$ is the set of all $2\times 2$ matrices of the form
\begin{center}
$\MATRIX{rr}{a & b\\c& d}$ for some $a,b,c,d\in\R$.
\end{center}
$M_{2}(\R)$ is a binary algebraic structure under both the addition and multiplication operations defined by
\begin{center}
$\MATRIX{rr}{a_{1} & b_{1}\\c_{1} & d_{1}} + \MATRIX{rr}{a_{2} & b_{2}\\c_{2} & d_{2}} = \MATRIX{rr}{a_{1} + a_{2} & b_{1} + b_{2}\\c_{1} + c_{2} & d_{1} + d_{2}}$ 
\end{center}
and 
\begin{center}
$\MATRIX{rr}{a_{1} & b_{1}\\c_{1} & d_{1}} * \MATRIX{rr}{a_{2} & b_{2}\\c_{2} & d_{2}} = \MATRIX{rr}{a_{1}a_{2} + b_{1}c_{2} & a_{1}b_{2} + b_{1}d_{2}\\c_{1}a_{2} + d_{1}c_{2} & c_{1}b_{2} + d_{1}d_{2}}$.
\end{center}
\begin{enumerate}[(a)]
\item Prove that $(M_{2}(\R), +)$ is both commutative and associative.
\item Prove that $(M_{2}(\R), *)$ is associative but not commutative.
\end{enumerate}
\end{enumerate}
\vsp

{\huge \S4.2 Problems}
\vsp

\begin{enumerate}[1.]
\item The set $2\Z := \SET{2n:n\in\Z} = \SET{0, \pm 2, \pm 4,\ldots}$ of even integers is a binary algebraic structure under addition.  Prove that $(\Z,+)\cong (2\Z,+)$.
%
\item Let $(S_{1},*_{1})$, $(S_{2}, *_{2})$, and $(S_{3}, *_{3})$ be binary algebraic structures such that
\begin{center}
$(S_{1},*_{1})\cong(S_{2},*_{2})$ and $(S_{2},*_{2})\cong(S_{3},*_{3})$.
\end{center}
Prove that $(S_{1},*_{1})\cong(S_{3},*_{3})$.
%
\item Let $(S,*)$ and $(S',*')$ be binary algebraic structures and suppose that $\varphi:S\to S'$ is an isomorphism.  Prove that $\varphi^{-1}:S'\to S$ is also an isomorphism.

\note This is why it doesn't matter whether we write $(S,*)\cong (S',*')$ or $(S',*')\cong (S,*)$!
%
\item Let $X$ be a non-empty set, and consider the following:
\begin{itemize}
\item The set $2^{X}$ of all functions $f:X\to\Z_{2}$ is a binary algebraic structure under the point-wise product $\cdot$, defined for any $f,g\in 2^{X}$ by
\begin{center}
$(f \cdot g)(x) = f(x)g(x)$ for all $x\in X$.
\end{center}
\item The set $\mathcal{P}(X)$ of all subsets of $X$ is a binary algebraic structure under the intersection operation $\cap$ defined by
\begin{center}
$A\cap B = \SET{x\in X:x\in A\text{ and }x\in B}$.
\end{center}
\end{itemize}
\begin{enumerate}[(a)]
\item What are the identity elements of $(2^{X}, \cdot)$ and $(\mathcal{P}(X), \cap)$?
\item Prove that $(2^{X}, \cdot)\cong (\mathcal{P}(X), \cap)$.
\end{enumerate}
%
\item Suppose $(S,*)$ and $(S',*')$ are binary algebraic structures, and that $\varphi:S\to S'$ is an isomorphism.
\begin{enumerate}[(a)]
\item Prove that if $(S,*)$ is commutative, then so is $(S',*')$.
\item Prove that if $(S,*)$ is associative, then so is $(S',*')$.
\item Prove that if $e\in S$ is an identity for $*$, then $\varphi(e)$ is an identity for $*'$.
\end{enumerate}
\end{enumerate}
\end{document}





