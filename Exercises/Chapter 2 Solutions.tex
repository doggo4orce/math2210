\documentclass[11pt,fleqn,dvipsnames,usenames]{article}

% to keep this file less overwhelming
% packages to include

\usepackage[dvipsnames, table]{xcolor}

\usepackage{
  amsthm,
  amsmath,
  amssymb, 
  arydshln, % for hyphenated lines in block matrices
  fancyhdr, % needed for header at top of each page
  graphicx, % to include pictures
  mathtools, % for a longer arrow
  multicol, % displaying enumerates and itemizes into multiple columns
  multirow, % for tables
  multido, % for TOC
  pgfplots, % for axis environment within tikz pictures
  systeme,
  tikz,
}

\usepackage[utf8]{inputenc}
\usepackage{color,soul}

\usepackage[inline, shortlabels]{enumitem}
\usepackage[hidelinks]{hyperref}


% global constants
\newcommand{\term}{Winter 2024}

% mathbb aliases
\newcommand{\COMPLEX}{\mathbb{C}}
\newcommand{\REAL}{\mathbb{R}}
\newcommand{\NATURAL}{\mathbb{N}}
\newcommand{\INTEGER}{\mathbb{Z}}

% for financial stuff
\newcommand{\dollar}{\mathrm{\$}}

% nicer looking trig functions
\newcommand{\SIN}[1]{\sin\left(#1\right)}
\newcommand{\COS}[1]{\cos\left(#1\right)}
\newcommand{\TAN}[1]{\tan\left(#1\right)}
\newcommand{\CSC}[1]{\csc\left(#1\right)}
\newcommand{\SEC}[1]{\sec\left(#1\right)}
\newcommand{\COT}[1]{\cot\left(#1\right)}

% automatically resize set brackets
\newcommand{\SET}[1]{\left\{#1\right\}}

% sums and products
\newcommand{\SUM}{\displaystyle\sum\limits}
\newcommand{\PROD}{\displaystyle\prod\limits}
\newcommand{\of}{\circ}
\newcommand{\restrict}[1]{\raisebox{-.5ex}{$|$}_{#1}}

% set intersection and union
\newcommand{\CAP}{\displaystyle\bigcap\limits}
\newcommand{\CUP}{\displaystyle\bigcup\limits}

% max and min
\newcommand{\MAX}[1]{\ensuremath{\max\left(#1\right)}}
\newcommand{\MIN}[1]{\ensuremath{\min\left(#1\right)}}

% for writing logic within mathematics environment
\newcommand{\FORALL}{\ensuremath{\text{ for all }}}
\newcommand{\FORSOME}{\ensuremath{\text{ for some }}}

% matrix notation
\newcommand{\MATRIX}[2]{\ensuremath{\left[\begin{array}{#1}#2\end{array}\right]}}
\newcommand{\COLUMN}[1]{\ensuremath{\left[\begin{array}{r}#1\end{array}\right]}}

% vector notation
%\newcommand{\vv}{\overset{\rightharpoonup}}
\newcommand{\vv}[1]{{\bf #1}}
\newcommand{\arr}{\overrightarrow}

% dot product
\newcommand{\dotp}{{\scriptstyle\bullet}}

% Text macros
\newcommand{\KER}[1]{\ensuremath{\text{ker}\left(#1\right)}}
\newcommand{\IMG}[1]{\ensuremath{\text{im}\left(#1\right)}}
\newcommand{\CHAR}[1]{\ensuremath{\text{char}\left(#1\right)}}
\newcommand{\BIGO}[1]{\ensuremath{\mathcal{O}\left(#1\right)}}
\newcommand{\TR}[1]{\ensuremath{\text{tr}\left(#1\right)}}

% abbreviations
\newcommand{\ds}{\displaystyle}
\newcommand{\md}{\mdseries}
\newcommand{\vsp}{\vspace{0.5cm}}
\newcommand{\smsp}{\vspace{0.25cm}}
\newcommand{\hsp}{\hspace{0.25cm}}

% new operators
\DeclareMathOperator\SPAN{Span}
\newcommand{\SPANOF}[1]{\ensuremath{\SPAN\left\{#1\right\}}}
\DeclareMathOperator\PROJ{proj}
\DeclareMathOperator\PERP{perp}

% quick abbreviations to avoid using latex environments
\newcommand{\answer}{\noindent \textbf{Answer:} }
\newcommand{\answers}{\noindent \textbf{Answers:} }
\newcommand{\application}{\noindent \textbf{Application:} }
\newcommand{\caution}{\noindent \textbf{Caution:} }
\newcommand{\conclusion}{\noindent \textbf{Conclusion:} }
\newcommand{\consequence}{\noindent \textbf{Consequence:} }
\newcommand{\defn}{\noindent \textbf{Definition:} }
\newcommand{\details}{\noindent \textbf{Details:} }
\newcommand{\example}{\noindent \textbf{Example:} }
\newcommand{\examples}{\noindent \textbf{Examples:} }
\newcommand{\exception}{\noindent \textbf{Exception:} }
\newcommand{\exercise}{\noindent \textbf{Exercise:} }
\newcommand{\exercises}{\noindent \textbf{Exercises:} }
\newcommand{\fact}{\noindent \textbf{Fact:} }
\newcommand{\facts}{\noindent \textbf{Facts:} }
\newcommand{\formula}{\noindent \textbf{Formula:} }
\newcommand{\goal}{\noindent \textbf{Goal:} }
\newcommand{\goals}{\noindent \textbf{Goals:} }
\newcommand{\hint}{\noindent \textbf{Hint:} }
\newcommand{\idea}{\noindent \textbf{Idea:} }
\newcommand{\illustration}{\noindent \textbf{Illustration:} }
\newcommand{\important}{\noindent \textbf{Important:} }
\newcommand{\midea}{\noindent \textbf{Main Idea:} }
\newcommand{\motivation}{\noindent \textbf{Motivation:} }
\newcommand{\nthm}[1]{\noindent \textbf{Theorem} (\textit{#1}):}
\newcommand{\notation}{\noindent \textbf{Notation:} }
\newcommand{\note}{\noindent \textbf{Note:} }
\newcommand{\notes}{\noindent \textbf{Notes:} }
\newcommand{\observation}{\noindent \textbf{Observation:} }
\newcommand{\observations}{\noindent \textbf{Observations:} }
\newcommand{\pict}{\noindent \textbf{Picture:} }
\newcommand{\plan}{\noindent \textbf{Plan:} }
\newcommand{\prf}{\noindent \textbf{Proof:} }
\newcommand{\problem}{\noindent \textbf{Problem:} }
\newcommand{\properties}{\noindent \textbf{Properties:} }
\newcommand{\question}{\noindent \textbf{Question:} }
\newcommand{\questions}{\noindent \textbf{Questions:} }
\newcommand{\recall}{\noindent \textbf{Recall:} }
\newcommand{\reason}{\noindent \textbf{Reason:} }
\newcommand{\remark}{\noindent \textbf{Remark:} }
\newcommand{\remarks}{\noindent \textbf{Remarks:} }
\newcommand{\reminder}{\noindent \textbf{Reminder:} }
\newcommand{\solution}{\noindent \textbf{Solution:} }
\newcommand{\nsolution}[1]{\noindent \textbf{Solution #1:} }
\newcommand{\strategy}{\noindent \textbf{Strategy:} }
\newcommand{\summary}{\noindent \textbf{Summary:} }
\newcommand{\terminology}{\noindent \textbf{Terminology:} }
\newcommand{\thm}{\noindent \textbf{Theorem:} }
\newcommand{\work}{\noindent \textbf{Work:} }


\usepackage[version=4]{mhchem}

% Where to look for pngs and jpegs
\graphicspath{{Images//}}

\usepackage[includehead, includefoot, left= 2cm, top =1.5cm, bottom = 1.5cm, textwidth=17.5cm]{geometry}

\usepackage{pifont, amsmath}

\pagestyle{fancy}
\fancyhf{}
\renewcommand{\headrulewidth}{1pt}
%\fancyhead[R]{\bfseries\sffamily\thepage}
%\fancyfoot[C]{\bfseries\sffamily\thepage}
\fancyhead[L]{\nouppercase{\bfseries\sffamily\leftmark}}

% used when adding fill-in-the-blanks for students
\newcommand{\blank}[1]{\underline{\hspace{#1}}}

% indents annoy me, and so does repeatedly typing \noindent
\newcommand{\p}{\noindent}

\begin{document}

\fancyhead[L]{\course}
\fancyhead[C]{\includegraphics[width=5cm, trim= 0 0.4cm 0 0]{TRU_logo}}
\fancyhead[R]{\term}
\renewcommand{\headrulewidth}{0.4pt}

{\huge Chapter 2 Solutions}
\vsp

\begin{enumerate}
\item Write $\ds{\frac{2 + 3i}{1 + 2i} + \frac{8+i}{2-i}}$ in the form $a+bi$ for some $a,b\in\R$.
\vsmsp

\solution
\begin{align*}
\frac{2 + 3i}{1 + 2i} + \frac{8+i}{2-i} &= \frac{2 + 3i}{1 + 2i}\cdot\frac{1 - 2i}{1-2i} + \frac{8+i}{2-i}\cdot \frac{2+i}{2+i}\\
&= \frac{(2+3i)(1-2i)}{(1^2 - (2i)^2)} + \frac{(8+i)(2+i)}{2^2 - (i)^2}\\
&= \frac{2 -4i +3i -6i^2}{1 + 4} + \frac{16 + 8i + 2i + i^2}{4 + 1}\\
&= \frac{8 - i}{5} + \frac{15 + 10i}{5}\\
&= \frac{23 + 9i}{5}\\
&= \frac{23}{5} + \frac{9}{5}i
\end{align*}

\item Prove the following properties hold in $\C$.  Assume that the corresponding properties hold in $\R$.
\begin{enumerate}[(a)]
\item $w + z = z + w$ for all $w,z\in\C$.\hfill (commutativity of addition)
\item $w + (z + \lambda) = (w + z) + \lambda$ for all $w,z,\lambda\in\C$.\hfill (associativity of addition)
\item $wz = zw$ for all $w,z\in\C$.\hfill (commutativity of multiplication)
\item $w(z\lambda) = (wz)\lambda$ for all $w,z,\lambda\in\C$.\hfill (associativity of multiplication)
\item $w(z + \lambda) = wz + w\lambda$ for all $w,z,\lambda\in\C$.\hfill (distributivity)
\end{enumerate}
\vsmsp

\solution 
\begin{enumerate}[(a)]
\item Write $w = (a_{1},b_{1})$ and $z = (a_{2},b_{2})$ for some $a_{1},a_{2},b_{1},b_{2}\in\R$.  Then
\begin{center}
$w + z = (a_{1},b_{1}) + (a_{2},b_{2}) = (a_{1} + a_{2}, b_{1} + b_{2}) = (a_{2} + a_{1}, b_{2} + b_{1}) = (a_{2},b_{2}) + (a_{1},b_{1}) = z + w$.
\end{center}

\item Write $w = (a_{1},b_{1}), z = (a_{2},b_{2})$, and $\lambda = (a_{3},b_{3})$ for some $a_{1},a_{2},a_{3},b_{1},b_{2},b_{3}\in\R$.  Then
\begin{align*}
w + (z + \lambda) &= (a_{1},b_{1}) + \big((a_{2},b_{2}) + (a_{3},b_{3})\big)\\
&= (a_{1},b_{1}) + (a_{2} + a_{3},b_{2}+b_{3})\\
&= \big(a_{1} + (a_{2} + a_{3}), b_{1} + (b_{2}+b_{3})\big)\\
&= \big((a_{1} + a_{2}) + a_{3}, (b_{1} + b_{2})+b_{3}\big)\\
&= (a_{1} + a_{2}, b_{1} + b_{2}) + (a_{3},b_{3})\\
&= (w + z) + \lambda.
\end{align*}

\item Write $w = (a_{1},b_{1})$ and $z = (a_{2},b_{2})$ for some $a_{1},a_{2},b_{1},b_{2}\in\R$.  Then
\begin{align*}
wz &= (a_{1},b_{1})\cdot (a_{2},b_{2})\\
&= (a_{1}a_{2} - b_{1}b_{2},a_{1}b_{2} + b_{1}a_{2})\\
&= (a_{2}a_{1} - b_{2}b_{1},a_{2}b_{1} + b_{2}a_{1})\\
&= (a_{2},b_{2})\cdot (a_{1},b_{1})\\
&= zw
\end{align*}

%\item \label{associativityofmult} Write $w = (a_{1},b_{1}), z = (a_{2},b_{2})$, and $\lambda = (a_{3},b_{3})$ for some $a_{1},a_{2},a_{3},b_{1},b_{2},b_{3}\in\R$.  In this case it is easier to work out the left and right hand sides separately, and check that they yield the same result:
%\vsmsp

%First we work out the left hand side:
%\begin{align*}
%w(z\lambda) &= (a_{1},b_{1})\cdot\big((a_{2},b_{2})\cdot (a_{3},b_{3})\big)\\
%&= (a_{1},b_{1})\cdot (a_{2}a_{3} - b_{2}b_{3},a_{2}b_{3} + b_{2}a_{3})\\
%&= \big(a_{1}(a_{2}a_{3}-b_{2}b_{3}) - b_{1}(a_{2}b_{3} + b_{2}a_{3}), a_{1}(a_{2}b_{3} + b_{2}a_{3}) + b_{1}(a_{2}a_{3} - b_{2}b_{3})\big)\\
%&= \big(a_{1}a_{2}a_{3} - a_{1}b_{2}b_{3} - b_{1}a_{2}b_{3} - b_{1}b_{2}a_{3}, a_{1}a_{2}b_{3} + a_{1}b_{2}a_{3} + b_{1}a_{2}a_{3} - b_{1}b_{2}b_{3}\big)
%\end{align*}

%Next we work out the right hand side:
%\begin{align*}
%(wz)\lambda &= \big((a_{1},b_{1})\cdot(a_{2},b_{2})\big)\cdot (a_{3},b_{3})\\
%&= (a_{1}a_{2} - b_{1}b_{2}, a_{1}b_{2} + b_{1}a_{2})\cdot (a_{3},b_{3})\\
%&= \big(a_{3}(a_{1}a_{2} - b_{1}b_{2}) - b_{3}(a_{1}b_{2} + b_{1}a_{2}), b_{3}(a_{1}a_{2} - b_{1}%b_{2}) + a_{3}(a_{1}b_{2} + b_{1}a_{2})\big)\\
%&= \big(a_{1}a_{2}a_{3} - b_{1}b_{2}a_{3} - a_{1}b_{2}a_{3} - b_{1}a_{2}b_{3}, a_{1}a_{2}b_{3} - b_{1}b_{2}b_{3}+a_{1}b_{2}a_{3} + b_{1}a_{2}a_{3}\big)\\
%&= \big(a_{1}a_{2}a_{3} - a_{1}b_{2}a_{3} - b_{1}a_{2}b_{3} - b_{1}b_{2}a_{3}, a_{1}a_{2}b_{3} + a_{1}b_{2}a_{3} + b_{1}a_{2}a_{3} - b_{1}b_{2}b_{3} \big)
%\end{align*}
%It follows that $w(z\lambda) = (wz)\lambda$, as required.

\item[(e)] Write $w = (a_{1},b_{1}), z = (a_{2},b_{2})$, and $\lambda = (a_{3},b_{3})$ for some $a_{1},a_{2},a_{3},b_{1},b_{2},b_{3}\in\R$.  Then
\begin{align*}
w(z+\lambda) &= (a_{1},b_{1})\cdot \big((a_{2},b_{2}) + (a_{3},b_{3})\big)\\
&= (a_{1},b_{1})\cdot (a_{2}+a_{3},b_{2} + b_{3})\\
&= \big(a_{1}(a_{2} + a_{3}) - b_{1}(b_{2} + b_{3}), a_{1}(b_{2}+b_{3}) + b_{1}(a_{2}+a_{3})\big)\\
&= \big(a_{1}a_{2} + a_{1}a_{3} - b_{1}b_{2} - b_{1}b_{3}, a_{1}b_{2} + a_{1}b_{3} + b_{1}a_{2} + b_{1}a_{3}\big)\\
&= (a_{1}a_{2} - b_{1}b_{2},a_{1}b_{2} + b_{1}a_{2}) + (a_{1}a_{3} - b_{1}b_{3}, a_{1}b_{3} + b_{1}a_{3})\\
&= (a_{1},b_{1})\cdot(a_{2},b_{2}) + (a_{1},b_{1})\cdot (a_{3},b_{3})\\
&= wz + w\lambda
\end{align*}
\end{enumerate}

\item Prove that $\RE{iz} = -\IM{z}$ for every $z\in\C$.
\vsmsp

\solution Let $z = a+bi\in\C$.  Then
\begin{center}
$\RE{iz} = \RE{i(a+bi)} = \RE{ai + bi^2} = \RE{-b + ai} = -b = -\IM{a+bi} = -\IM{z}$.
\end{center}

\item Show that for any $z_{1},z_{2}\in\COMPLEX$, we have $\CC{z_{1} + z_{2}} = \CC{z_{1}} + \CC{z_{2}}$ and $\CC{z_{1}z_{2}} = \CC{z_{1}}\cdot\CC{z_{2}}$.
\vsmsp

\solution For any $z_{1} = a_{1}+b_{1}i\in\C$ and $z_{2} = a_{2} +b_{2}i\in\C$ we have
\begin{align*}
\CC{z_{1} + z_{2}} &= \CC{(a_{1} + b_{1}i) + (a_{2} + b_{2}i)}\\
&= \CC{(a_{1} + a_{2}) + (b_{1} + b_{2})i}\\
&= (a_{1} + a_{2}) - (b_{1} + b_{2})i\\
&= a_{1} + a_{2} - b_{1}i - b_{2}i\\
&= (a_{1} - b_{1}i) + (a_{2} - b_{2}i)\\
&= \CC{a_{1} + b_{1}i} + \CC{a_{2} + b_{2}i}\\
&= \CC{z_{1}} + \CC{z_{2}},
\end{align*}
and calculating $\CC{z_{1}z_{2}}$ and $\CC{z_{1}}\cdot\CC{z_{2}}$ separately, we see they are equal since
\begin{center}
$\CC{z_{1}z_{2}} = \CC{(a_{1} + b_{1}i)(a_{2} + b_{2}i)} = \CC{(a_{1}a_{2} - b_{1}b_{2}) + (a_{1}b_{2} + b_{1}a_{2})i} = (a_{1}a_{2} - b_{1}b_{2}) - (a_{1}b_{2} + b_{1}a_{2})i$
\end{center}
and
\begin{center}
$\CC{z_{1}}\cdot\CC{z_{2}} = (a_{1}-b_{1}i)(a_{2}-b_{2}i) = a_{1}a_{2} - a_{1}b_{2}i - b_{1}a_{2}i + b_{1}b_{2}i^2 = (a_{1}a_{2} - b_{1}b_{2}) - (a_{1}b_{2} + b_{1}a_{2})i$.
\end{center}
\item Compute $\ds{3i^{11} + 6i^{3} + \frac{8}{i^{20}} + i^{-1}}$.
\vsmsp

\solution Since $i^{4} = 1$, it follows that $i^{8} = (i^4)^2 = 1$ and $i^{20} = (i^4)^5 = 1$.  Note also that $i^3 = -i$ and
\begin{center}
$\ds{i^{-1} = \frac{1}{i} = \frac{1}{i}\cdot \frac{i}{i} = \frac{i}{-1} = -i}$.
\end{center}
It follows that
\begin{center}
$\ds{3i^{11} + 6i^{3} + \frac{8}{i^{20}} + i^{-1}} = \ds{3i^{8}i^{3} + 6(-i) + \frac{8}{1} - i} = \ds{-3i + 6(-i) + 8 - i} = \ds{8 -10i}$.
\end{center}

\item If $z\in\C$ with $|z| = 1$ and $z\neq 1$, then check that $\RE{\dfrac{1}{1-z}} = \dfrac{1}{2}$.
\vsmsp

\solution Assume $z = a + bi\in\C$ with $|z| = 1$ but $z\neq 1$.  It follows that $\sqrt{a^2 + b^2} = 1$ and hence $a^{2}+b^{2} = 1$.  Then
\begin{align*}
\dfrac{1}{1-z} &= \dfrac{1}{1 - (a + bi)}\\
&= \dfrac{1}{(1 - a) - bi}\\
&= \dfrac{1}{(1 - a) - bi}\cdot\dfrac{(1 - a) + bi}{(1 - a) + bi}\\
&= \dfrac{(1 - a) + bi}{(1 - a)^2 + b^2}\\
&= \dfrac{(1 - a) + bi}{1^2 - 2a + a^2 + b^2}\\
&= \dfrac{(1 - a) + bi}{2 -2a}\\
&= \dfrac{1 - a}{2 - 2a} + \dfrac{b}{2 - 2a}i,
\end{align*}
and hence $\RE{\dfrac{1}{1-z}} = \dfrac{1 - a}{2 - 2a} = \dfrac{1-a}{2(1-a)} = \dfrac{1}{2}$.

\item Show that if $\big(\CC{z}\big)^2 = z^2$, then $z$ is either real or purely imaginary.
\vsmsp

\solution Suppose $z = a + bi\in C$ such that $\big(\CC{z}\big)^2 = z^2$.  Note that
\begin{center}
$z^2 = (a + bi)(a + bi) = a^2 + abi + bai + b^2i^2 = a^2 - b^2 + 2abi$
\end{center}
and
\begin{center}
$\big(\CC{z}\big)^2 = (a - bi)(a - bi) = a^2 - abi - bai + b^2i^2 = a^2 - b^2 - 2abi$.
\end{center}
By assumption, we must have $a^2 - b^2 + 2abi = a^2 - b^2 - 2abi$ which forces $2abi = -2abi$.  It follows that $4abi = 0$ which can only happen if $a = 0$ or $b=0$.  Since $\RE{z} = a$ and $\IM{z} = b$, this completes the proof.
%\item Be sure to provide a complete justification when answering each of the following questions.
%\begin{enumerate}[(a)]
%\item Does there exist $a_{0},a_{1},a_{2},a_{3}\in \R$ such that
%\begin{center}
%$3(z-2i)(z+i)(z-2) = a_{0}(z-a_{1})(z - a_{2})(z - z_{3})$?
%\end{center}
%\item Does there exist $z_{0},z_{1},z_{2},z_{3},z_{4},z_{5}\in \C$ such that
%\begin{center}
%$3z^5 - 2iz^4 + \pi z^2 - i = z_{0}(z - z_{1})(z - z_{2})(z - z_{3})(z - z_{4})(z - z_{5})$?
%\end{center}
%\end{enumerate}
\newpage

\item Find a polar coordinate representation for each of the following complex numbers.
\begin{multicols}{3}
\begin{enumerate}[(a)]
\item $-6$
\item $\sqrt{3} + i$
\item $-2-2i$
\end{enumerate}
\end{multicols}
\vsmsp
\begin{multicols}{2}
\solution
\begin{enumerate}[(a)]
\item $(6,\pi)$ is a polar coordinate representation of $-6$.
\item[\ ]
\item[\ ]
\item $(2,\pi/6)$ is a polar coordinate representation of $\sqrt{3} + i$.
\item[\ ]
\item[\ ]
\item $(2\sqrt{2},-3\pi/4)$ is a polar coordinate representation of $-2-2i$.
\item[\ ]
\item[\ ]
\end{enumerate}


\columnbreak

\hspace{1cm} \includegraphics[width=0.5\linewidth]{chapter2problem8.png}
\end{multicols}
\vsmsp

\item The following ordered pairs are each polar coordinate representations of a complex number.  Write the corresponding complex number in the form $a + bi$.
\begin{multicols}{3}
\begin{enumerate}[(a)]
\item $(3, \pi/6)$
\item $(2, 3\pi/2)$
\item $(1, 7\pi/4)$
\end{enumerate}
\end{multicols}
\vsmsp

\solution Note that if $(r,\theta)$ is a polar coordinate representation of $z\in \C$, then
\begin{center}
$z = re^{i\theta} = r\big(\cos(\theta) + i\sin(\theta)\big)$.
\end{center}

\begin{enumerate}[(a)]
\item $3e^{i(\pi/6)} = 3\big(\cos(\pi/6) + i\sin(\pi/6)\big) = 3\big(1/2 + i(\sqrt{3}/2)\big) = 3/2 + (3\sqrt{3}/2)i$
\item $2e^{i(3\pi/2)} = 2\big(\cos(3\pi/2) + i\sin(3\pi/2)\big) = 2\big(0 + (-1)i\big) = -2i$
\item $e^{i(7\pi/4)} = \cos(7\pi/4) + i\sin(7\pi/4)\big) = (-\sqrt{2}/2) + (-\sqrt{2}/2)i = -\sqrt{2}/2 - (\sqrt{2}/2)i$
\end{enumerate}

\item Calculate $\dfrac{\big(\sqrt{2} + \sqrt{6}i\big)^5}{\big(1+i\big)^9}$.
\vsmsp

\solution First note that we may write $\sqrt{2} + \sqrt{6}i$ and $1+i$ may be written in polar form as
\begin{center}
$\sqrt{2} + \sqrt{6}i = \sqrt{2}(1 + \sqrt{3}i) = \sqrt{2}\big(2e^{i(\pi/3)}\big) = 2\sqrt{2}e^{i(\pi/3)}$ and $1+i = \sqrt{2}e^{i(\pi/4)}$.
\end{center}
Then compute

\begin{align*}
\dfrac{\big(\sqrt{2} + \sqrt{6}i\big)^5}{\big(1+i\big)^9} = \dfrac{\big(2\sqrt{2}e^{i(\pi/3)}\big)^5}{\big(\sqrt{2}e^{i(\pi/4)}\big)^9} = \dfrac{128\sqrt{2}e^{i(5\pi/3)}}{16\sqrt{2}e^{i(9\pi/4)}} = 8e^{i(5\pi/3 - 9\pi/4} &= 8e^{i(20\pi - 27\pi)/4}\\
&= 8e^{i(-7\pi/4)}\\
&= 8e^{i(\pi/4)}\\
&= 8\big(\sqrt{2}/2 + (\sqrt{2}/2)i\big)\\
&= 4\sqrt{2} + 4\sqrt{2}i
\end{align*}

\item Prove that for any $z\in\C$, $|z^{n}| = |z|^{n}$.
\vsmsp

\solution Let $z = re^{i\theta}\in\C$.  Then $|z| = r$ and
\begin{center}
$|z^n| = \big|(re^{i\theta})^n\big| = \big|r^ne^{i(n\theta)}\big| = |r^n|\cdot {\color{blue}\underbrace{\color{black}\big|e^{i(n\theta)}\big|}_{=1}} = r^n  = |z|^n$.
\end{center}
\item Is it true that $\PARG{z_1z_2} = \PARG{z_1} + \PARG{z_2}$ for every $z_{1},z_{2}\in\C$?
\vsmsp

\solution No.  For example $\PARG{-i} = -\pi/2$, so
\begin{center}
$\PARG{-i} + \PARG{-i} = -\pi/2 + (-\pi/2) = -\pi$
\end{center}
but
\begin{center}
$\text{Arg}\big((-i)\cdot(-i)\big) = \PARG{-1} = \pi$.
\end{center}
\item Find all the $8$th roots of unity, and sketch them in the complex plane.
\item Let $k\in\Z^{+}$ and suppose that $\zeta$ is a primitive $n$th root of unity.  Show that $\zeta^{k}$ is a primitive $n$th root of unity if and only if $\GCD{n,k} = 1$.
\end{enumerate}
\end{document}





